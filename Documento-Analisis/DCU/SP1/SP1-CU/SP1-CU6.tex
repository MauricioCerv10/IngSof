\begin{UseCase}{SP1-CU6}{Registrar Unidad Temática}{El usuario podrá registrar una Unidad Temática correspondiente a una Unidad de Aprendizaje}
		\UCitem{Versión}{\color{Gray}1.0}
		\UCitem{Autor}{\color{Gray}Maldonado Carpio Jorge Enrique}
		\UCitem{Supervisa}{\color{Gray}Ramírez Martínez Janet Naibi}
		\UCitem{Actor}{\hyperlink{Usuario}{Usuario}}
		\UCitem{Propósito}{Poblar de Unidades Temáticas  a una Unidad de Aprendizaje.}
		\UCitem{Entradas}{Las entradas para el registro de tiempos de la Unidad de Aprendizaje serán:
          \begin{itemize}
          	\item Unidad de Aprendizaje.
          	\item Número de Unidad Tématica.
          	\item Nombre.
          	\item Unidad de Competencia.
          	\item Temas.
          	\item Estrategias de Aprendizaje.
          	\item Evaluación de los Aprendizajes.
          \end{itemize}
        }
		\UCitem{Origen}{Teclado y Mouse.}
		\UCitem{Salidas}{}
		\UCitem{Destino}{Pantalla.}
		\UCitem{Precondiciones}{Los siguientes catálogos no deben de estar vacios:
          \begin{itemize}
          	\item Unidad de Aprendizaje.
          \end{itemize}
          }
		\UCitem{Postcondiciones}{La Unidad Temáticas queda registrado en el sistema y asociada a una Unidad de Aprendizaje.}
		\UCitem{Errores}{}
		\UCitem{Estado}{Revisión.}
		\UCitem{Observaciones}{}
\end{UseCase}

%--------------------------- CU TRAYECTORIA PRINCIPAL -------------------------
\begin{UCtrayectoria}{Principal}
    
    \UCpaso[\UCactor] Presiona la opción \IUbutton{Registrar Unidad Temática } del menú lateral
    \UCpaso Verifica que los catálogos de Unidad de Aprendizaje contengan información.
    \UCpaso Carga la información de los catálogos.
    \UCpaso Muestra la interfaz de usuario \IUref{SP1-U3}{Registro de Unidad Temática}.
    \UCpaso[\UCactor] Selecciona la Unidad de Aprendizaje.
    \UCpaso[\UCactor] Ingresa el número de Unidad Tématica.
    \UCpaso[\UCactor] Ingresa el nombre de la Unidad Temática.
    \UCpaso[\UCactor] Ingresa la Unidad de Competencia de la Unidad Temática.
    \UCpaso[\UCactor] Presiona el botón \IUbutton{Registrar Tema}
    \UCpaso Abre un modal en el \UCref{SP1-CU7} [Trayectoria A]
    \UCpaso[\UCactor] Ingresa las Estrategias de Aprendizaje.
    \UCpaso[\UCactor] Presiona el botón \IUbutton{Registrar Evaluación de los Aprendizajes}
    \UCpaso Abre un modal en el \UCref{SP1-CU8}
    \UCpaso[\UCactor] Termina la operación presionando el botón \IUbutton{Guardar}. [Trayectoria B]
    \UCpaso Verifica que todos los campos marcados como obligatorios hayan sido llenados.[Trayectoria C]
    \UCpaso Guarda la información de la Unidad Temática.
    
    \UCpaso El sistema muestra el mensaje \MSGref{MSG5}{Registro finalizado exitosamente}.
    \UCpaso[\UCactor] Cierra el mensaje presionando el botón \IUbutton{Ok}. 
    
    
    
    \UCpaso Muestra la interfaz de usuario \IUref{IU1}{Página principal}.
\end{UCtrayectoria}

%------------------------ CU TRAYECTORIA ALTERNARTIVA A -------------------------

\begin{UCtrayectoriaA}{A}{El usuario desea agregar más de un tema.}

    \UCpaso Continua en el paso 9 de la trayectoria principal del \UCref{SP1-CU1}.
    
\end{UCtrayectoriaA}

\begin{UCtrayectoriaA}{B}{El usuario desea cancelar el registro del Programa Sintético.}

    \UCpaso[\UCactor] Presiona el botón \IUbutton{Cancelar}
    \UCpaso Muestra el \MSGref{MSG6}{¿Seguro que desea cancelar el registro?}.
    \UCpaso[\UCactor] Presiona el botón \IUbutton{Si} [Trayectoria B.1]
    \UCpaso Continua en el paso 19 de la trayectoria principal del \UCref{SP1-CU1}.
    
\end{UCtrayectoriaA}

\begin{UCtrayectoriaA}{B.1}{El usuario no desea cancelar el registro del Programa Sintético.}
    
    \UCpaso[\UCactor] Presiona el botón \IUbutton{No}
    \UCpaso Continua en el paso 15 de la trayectoria principal del \UCref{SP1-CU1}.
    
\end{UCtrayectoriaA}


\begin{UCtrayectoriaA}{C}{Uno o más campos obligatorios no fueron contestados.}
	\UCpaso Detecta uno o más campos sin contestar.
    \UCpaso Muestra el mensaje \MSGref{MSG4}{Los campos marcados con (*) son obligatorios}.
    \UCpaso[\UCactor] Cierra el mensaje presionando el botón \IUbutton{Aceptar}.
    \UCpaso Continua en el paso 5 de la trayectoria principal del \UCref{SP1-CU1}.
\end{UCtrayectoriaA}
