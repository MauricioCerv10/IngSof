\begin{UseCase}{SP2-CU6}{ Autorizar Sección Tiempos Asignados de Unidad de Aprendizaje}{El analista podrá visualizar la información de la sección de tiempos asignados de la unidad de aprendizaje para poder aprobarla.}
		\UCitem{Versión}{\color{Gray}1.0}
		\UCitem{Autor}{\color{Gray}Domínguez López Humberto}
		\UCitem{Supervisa}{\color{Gray}Parra Garcilazo Cinthya Dolores}
		\UCitem{Actor}{Analista}
		\UCitem{Propósito}{Que el analista conozca la sección de Tiempos Asignados de la propuesta de Unidad de Aprendizaje para determinar si necesita correcciones.}
		\UCitem{Entradas}{Clic en botones:
          \begin{itemize}
          	\item Finalizar Revisión.
          	\item Guardar.
            \item Cancelar.
            \item Nuevo Comentario.
            \item Subrayar.
            \item Eliminar subrayado.
            \item Editar Comentario.
            \item Eliminar Comentario.
          \end{itemize}
        }
		\UCitem{Origen}{Mouse.}
		\UCitem{Salidas}{
        	\begin{itemize}
        		\item MSG15. ¿Está seguro que desea Cancelar la revisión?
				Se perderán las Anotaciones que no hayas guardado anteriormente.
               \item MSG16. ¿Está seguro de Finalizar la Revisión?
               \item MSG17 Sección Aprobada.
                
               \item MSG18 Revisión de Sección Finalizada.
               \item MSG19 Anotaciones En Sección Guardadas Correctamente.


        	\end{itemize}
        }
		\UCitem{Destino}{Pantalla.}
		\UCitem{Precondiciones}{ Se llamó el caso de uso SP2-CU1}
		\UCitem{Postcondiciones}{Se habilita la llamada a los casos de uso SP2-CU11, SP2-CU14.}
		\UCitem{Errores}{}
		\UCitem{Estado}{Revisión.}
		\UCitem{Observaciones}{}
\end{UseCase}

%--------------------------- CU TRAYECTORIA PRINCIPAL -------------------------
\begin{UCtrayectoria}{Principal}

    \UCpaso[\UCactor] presiona el botón  \IUbutton{Tiempos Asignados} de la interfaz de usuario \IUref{SP2-IU-INICIO}{Sección Inicio}

    \UCpaso El sistema obtiene la información correspondiente a la sección de Tiempos Asignados.
    
    \UCpaso El sistema obtiene la bitácora de comentarios correspondientes a la sección Tiempos Asignados de la Unidad de Aprendizaje. 
    
    \UCpaso El sistema verifica que la sección Tiempos Asignados de la Unidad de Aprendizaje no haya sido aprobada anteriormente. \BRref{BR9}{Aprobación de Tareas Seccionadas.} \hyperlink{SP2-CU6-A1}{Trayectoria A1}. 
    
    \UCpaso El sistema muestra la interfaz de usuario  \IUref{SP2-IU-TIEMPOS}{Sección Tiempos Asignados}.
    
    \UCpaso[\UCactor] presiona el botón \IUbutton{Finalizar Revisión}. \hyperlink{SP2-CU6-A2}{Trayectoria A2}.
    \UCpaso El sistema muestra el \MSGref{MSG16}{¿Está seguro de Finalizar la Revisión?}.
    
    \UCpaso	El sistema verifica que no existan nuevos comentarios o subrayados para la sección de la Unidad de Aprendizaje.\BRref{BR8}{Aprobación de Tareas.} \hyperlink{SP2-CU6-A3}{Trayectoria A3}. 
    
    \UCpaso El sistema pone el estado de la sección Tiempos Asignados  en “Aprobado”.
    
    \UCpaso El sistema muestra el mensaje \MSGref{MSG17}{Sección Aprobada}.

    \UCpaso El sistema muestra la interfaz de usuario \IUref{SP2-IU-INICIO}{Sección Inicio}

\end{UCtrayectoria}

%------------------------ CU TRAYECTORIA ALTERNARIVA A1 -------------------------

\begin{UCtrayectoriaA}{A1}{La sección de la Unidad de Aprendizaje ya ha sido aprobada anteriormente.}

	\hypertarget{SP2-CU6-A1}{\UCpaso El sistema muestra la interfaz de usuario \IUref{SP2-IU-TIEMPOS}{Sección Tiempos Asignados}.}
    \UCpaso El sistema deshabilita los botones superiores: \IUbutton{Nuevo Comentario}, \IUbutton{Subrayar}, \IUbutton{Eliminar Subrayado}.
    \UCpaso El sistema deshabilita los botones inferiores: \IUbutton{Cancelar}, \IUbutton{Guardar}, \IUbutton{Finalizar Revisión}.
    \UCpaso El sistema deshabilita los botones laterales de: \IUbutton{Modificar Comentario}, \IUbutton{Eliminar Comentario}.
\end{UCtrayectoriaA}

%------------------------ CU TRAYECTORIA ALTERNARIVA A2 -------------------------
	
\begin{UCtrayectoriaA}{A2}{El analista no desea finalizar aun la revisión de la sección de la Unidad de Aprendizaje.}

    \hypertarget{SP2-CU6-A2}{\UCpaso[\UCactor] presiona el botón \IUbutton{Guardar} \hyperlink{SP2-CU6-A2.1}{Trayectoria A2.1}}. 
    \UCpaso El sistema guarda los nuevos comentaros y subrayados hechos durante esa sesión.
    \UCpaso El sistema muestra el mensaje \MSGref{MSG19}{Anotaciones En Sección Guardadas Correctamente}.
    \UCpaso El sistema muestra la interfaz de usuario \IUref{SP2-IU-INICIO}{Sección Inicio}
\end{UCtrayectoriaA}

%------------------------ CU TRAYECTORIA ALTERNARIVA A2.1 -----------------------
\begin{UCtrayectoriaA}{A2.1}{El analista desea cancelar todo lo que haya hecho en la sección de la Unidad de Aprendizaje durante esa sesión.}

	\hypertarget{SP2-CU6-A2.1}{\UCpaso[\UCactor] presiona el botón \IUbutton{Cancelar}}. 
    \UCpaso El sistema muestra el mensaje \MSGref{MSG15}{¿Está seguro que desea Cancelar la revisión? Se perderán las Anotaciones que no hayas guardado anteriormente.}
Se perderán las Anotaciones que no hayas guardado anteriormente. 
    \UCpaso El sistema elimina los nuevos comentaros y subrayados hechos durante esa sesión.
    \UCpaso El sistema muestra la interfaz de usuario \IUref{SP2-IU-INICIO}{Sección Inicio}.
\end{UCtrayectoriaA}

%------------------------ CU TRAYECTORIA ALTERNARIVA A3 -----------------------
	
\begin{UCtrayectoriaA}{A3}{El analista realizo comentarios y subrayados para su posterior corrección en la sección de la Unidad de Aprendizaje.} 

	\hypertarget{SP2-CU6-A3}{\UCpaso El sistema pone el estado de la sección Tiempos Asignados de Unidad de Aprendizaje en “Revisado”.}
    \UCpaso El sistema muestra el mensaje \MSGref{MSG18}{Revisión de Sección Finalizada}.
    \UCpaso El sistema muestra la interfaz de usuario \IUref{SP2-IU-INICIO}{Sección Inicio}.
\end{UCtrayectoriaA}