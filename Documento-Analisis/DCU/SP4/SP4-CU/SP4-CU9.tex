% EDITAR UNIDAD DE APRENDIZAJE.
\begin{UseCase}{SP4-CU9}{Editar Unidad de Aprendizaje}{El usuario Docente modifica los datos de las Unidades de Aprendizaje registradas en el sistema..}
		\UCitem{Versión}{\color{Gray}1.0}
		\UCitem{Autor}{\color{Gray}Cervantes Moreno Christian Andres}
		\UCitem{Supervisa}{\color{Gray} Evelyn Reyes}
		\UCitem{Actor}{\hyperlink{Usuario}{Docente}}
		\UCitem{Propósito}{Editar el nombre, créditos TEPIC, créditos SATCA, horas Teoría/Semana, horas Práctica/semana, Área de formación y semestre de una Unidad de Aprendizaje.}
		\UCitem{Entradas}{Las entradas para la modificación de la Unidad de Aprendizaje serán:
          \begin{itemize}
          	\item Nombre (Tipo carácter).
          	\item Créditos Totales TEPIC (Tipo double).
            \item Créditos Totales SATCA (Tipo double).
            \item Total horas / Teoría (Tipo entero).
            \item Total horas / Práctica (Tipo entero).
            \item Área de Formación.
            \item Semestre
          \end{itemize}
        }
		\UCitem{Origen}{Teclado, Mouse}
		\UCitem{Salidas}{
        	\begin{itemize}
        		\item \MSGref{MSG32}{Todos los campos son obligatorios.}
                \item \MSGref{MSG31}{Los cambios se guardaron exitosamente.}
                \item \MSGref{MSG29}{¿Está seguro que desea cancelar? Se perderán todos los avances sin guardar.}
                \item \MSGref{MSG7}{Los catálogos necesarios no se han cargado, favor de intentarlo más tarde.}
				\item \MSGref{MSG9}{Por el momento no se puede realizar el registro.}
        	\end{itemize}
        }
		\UCitem{Destino}{Pantalla.}
		\UCitem{Precondiciones}{ Debe de existir al menos una Unidad de Aprendizaje registrada en el sistema.}
		\UCitem{Postcondiciones}{La Unidad de Aprendizaje queda modificada en el sistema.}
		\UCitem{Errores}{
			  \begin{itemize}
				\item E1. Entrada invalida de un caracter no entero.
				\item E2. Entrada invalida de un caracter no double.
				\item E2. Entrada invalida de un carácter
			\end{itemize}
		}
		\UCitem{Estado}{Revisión.}
		\UCitem{Observaciones}{}
\end{UseCase}
%--------------------------- CU TRAYECTORIA PRINCIPAL -------------------------
\begin{UCtrayectoria}{Principal}
    %Usuario
    \UCpaso[\UCactor] Presiona el botón \IUbutton{Lápiz} de una Unidad de Aprendizaje en la Interfaz de usuario  \IUref{IU2.1-D}{Consultar Mapa Curricular}.
    \UCpaso El sistema carga el catálogo de Areas de Formación.[Trayectoria A]
    \UCpaso El sistema carga el catálogo de Semestres.[Trayectoria A]
    \UCpaso Muestra la interfaz de usuario \IUref{IU2.1.2-D}{Editar Unidad de Aprendizaje}.
    \UCpaso Carga los datos almacenados en el sistema.
    \UCpaso[\UCactor] Modifica los campos qeu desea.
    \UCpaso[\UCactor] Termina la operación presionando el botón \IUbutton{Guardar}. [Trayectoria B] [Trayectoria B.1]
    \UCpaso Verifica que todos los campos hayan sido completamente contestados. [Trayectoria C]
    \UCpaso Persiste los datos.
    \UCpaso El sistema muestra el mensaje \MSGref{MSG31}{Los cambios se guardaron exitosamente.}
    \UCpaso[\UCactor] Cierra el mensaje presionando el botón \IUbutton{Aceptar}.
    \UCpaso Muestra la interfaz de usuario \IUref{IU2.1-D}{Consultar Mapa Curricular}.
\end{UCtrayectoria}
%------------------------ CU TRAYECTORIA ALTERNARIVA A -------------------------
\begin{UCtrayectoriaA}{A}{El sistema no carga el catálogo de la Unidad de Aprendizaje}
	\UCpaso Muestra el mensaje \MSGref{MSG7}{Los catálogos necesarios no se han cargado, favor de intentarlo más tarde.}
	\UCpaso[\UCactor] Cierra el mensaje presionando el botón \IUbutton{Aceptar}.
	 \UCpaso Muestra la interfaz de usuario \IUref{IU2.1-D}{Consultar Mapa Curricular}
\end{UCtrayectoriaA}
%------------------------ CU TRAYECTORIA ALTERNARIVA B -------------------------
\begin{UCtrayectoriaA}{C}{El actor presiona el botón \IUbutton{Cancelar}}
	\UCpaso Muestra el mensaje \MSGref{MSG29}{¿Está seguro que desea cancelar? Se perderán todos los avances sin guardar.}.
	\UCpaso[\UCactor] Cierra el mensaje presionando el botón \IUbutton{Si}.
	\UCpaso Muestra la interfaz de usuario \IUref{IU2.1-D}{Consultar Mapa Curricular}
\end{UCtrayectoriaA}
%------------------------ CU TRAYECTORIA ALTERNARIVA B.1 -------------------------
\begin{UCtrayectoriaA}{D}{El actor presiona accidentalmente el botón \IUbutton{Cancelar}}
	\UCpaso Muestra el mensaje \MSGref{MSG29}{ ¿Está seguro que desea cancelar? Se perderán todos los avances sin guardar.}.
    \UCpaso Continúa en el paso 8 de la trayectoria principal del \UCref{SP4-CU9}.
\end{UCtrayectoriaA}
%------------------------ CU TRAYECTORIA ALTERNARIVA C-------------------------
\begin{UCtrayectoriaA}{E}{El sistema detecta uno o más campos sin contestar.}
	\UCpaso Muestra el mensaje \MSGref{MSG32}{Todos los campos son obligatorios.}
	\UCpaso[\UCactor] Cierra el mensaje presionando el botón \IUbutton{Aceptar}.
    \UCpaso Continúa en el paso 6 de la trayectoria principal del \UCref{SP4-CU9}.
\end{UCtrayectoriaA}