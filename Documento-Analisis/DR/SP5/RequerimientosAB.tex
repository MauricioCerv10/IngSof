% -------------- TABLA PARA REQUERIMIENTOS FUNCIONALES ---------------- % 
% Nomenclatura para la prioridad: 
%	A - Alta
%	M - Media
%	B - Baja

\textbf{Nota:} Se usa :
	\begin{itemize}
		\item \textbf{JDIC:} Jefe de departamento de Innovación curricular
		\item \textbf{JDIA:} Jefe de división de innovación académica
		\item \textbf{DES:} Directora de Educación Superior
	\end{itemize}

	\begin{requerimientos}
		
		\FRitem{SP5-F1}{Registrar empleados del JDIA}{El sistema permitirá al usuario JDIA registrar usuarios JDIC.}{A}{Sistema}

		\FRitem{SP5-F2}{Registrar empleados del JDIC}{El sistema permitirá al usuario JDCI registrar usuarios analistas y jefes de innovación educativa.}{A}{Usuario}

		\FRitem{SP5-F3}{Consultar empleados}{El sistema permitirá al JDIA y al JDIC consultar la información y trabajo activo de los analistas y jefes de innovación educativa.}{M}{Sistema}

		\FRitem{SP5-F4}{Eliminar empleados del JDIA}{El sistema permitirá al JDIA eliminar usuarios JDIC.}{B}{Usuario}

		\FRitem{SP5-F5}{Eliminar empleados del JDIC}{El sistema permitirá al JDCI eliminar usuarios analistas y jefes de innovación educativa.}{B}{Usuario}

		\FRitem{SP5-F6}{Consultar asignaciones del analista}{El sistema le mostrará al usuario analista las tareas  que tiene asignadas.}{A}{Sistema}

		\FRitem{SP5-F7}{Consultar mapas curriculares}{El JDIA y el JDIC consultarán la lista de los mapas curriculares por nombre de la unidad académica o todos los mapas curriculares registrados.}{M}{Sistema}

		\FRitem{SP5-F8}{Ver mapa curricular}{El sistema mostrará al JDIA y al JDIC la información de un mapa curricular previamente seleccionado.}{A}{Sistema}

		\FRitem{SP5-F9}{Asignar analista}{El JDIC seleccionará (de la lista de usuarios analista, él incluido) el empleado (visualizando su nombre y el número de tareas que tiene asignadas) que revisará el mapa curricular previamente elegido.}{M}{Usuario}

		\FRitem{SP5-F10}{Atender mapa curricular por parte del JDIC}{El sistema permitirá al JDIC analizar, en conjunto con los analistas, uno o más mapas curriculares recibidos por las unidades académicas.}{A}{Usuario}

		\FRitem{SP5-F11}{Comunicar observaciones del mapa curricular a los analistas}{El sistema permitirá al JDIC contactar al analista encargado de un mapa curricular para comunicarle observaciones, correcciones, notas, etcétera, respecto a éste.}{M}{Usuario}

		\FRitem{SP5-F12}{Realizar modificaciones al mapa curricular}{El sistema permitirá a la JDIC realizar correcciones y ajustes a un mapa curricular, en conjunto con su analista asignado, para llegar a un acuerdo.}{A}{Usuario}

		\FRitem{SP5-F13}{Verificar nombre de unidades de aprendizaje en el mapa curricular}{El sistema permitirá a la JDIC y a la JDIA aprobar o rechazar el nombre que se le asignó a las unidades de aprendizaje en un mapa curricular.}{A}{Sistema}

		\FRitem{SP5-F14}{Iniciar revisión del mapa curricular}{Un mapa curricular enviado por la unidad académica debe tener todos sus campos completos y llenos para que se pueda iniciar el revisado por la DES en el sistema.}{A}{Sistema}

		\FRitem{SP5-F15}{Realizar notas y comentarios}{El sistema permitirá al JDIC y a los analistas redactar una o más correcciones por cada campo de un mapa curricular.}{A}{Usuario}
		
		\FRitem{SP5-F16}{Autorizar campos}{El sistema permitirá al JDIC y a los analistas aprobar campos y secciones de un mapa curricular.}{A}{Sistema}
		
		\FRitem{SP5-F17}{Historial de notas y correcciones}{El sistema manejará un historial de correcciones de un mapa curricular, que son realizadas por el JDIC, el JDIA y los analistas.}{A}{Sistema}

		\FRitem{SP5-F18}{Imprimir notas y correcciones}{El sistema permitirá imprimir el historial de correcciones de un mapa curricular, con el formato establecido por la DES.}{B}{Sistema}

		\FRitem{SP5-F19}{Resaltar palabras erróneas}{El sistema permitirá a la JDIC y analistas subrayar o marcar una o varias palabras o secciones erróneas en cada campo de un mapa curricular.}{M}{Usuario}

		\FRitem{SP5-F20}{Estado de revisión}{El sistema administrará el estado de un mapa curricular. Puede estar en estado de revisión por los analistas, en revisión por el JDIC, en revisión por el JDIA, sin asignar analista, aprobado por el JDIC, etcétera.}{M}{Sistema}
		
		\FRitem{SP5-F21}{Enviar correcciones del mapa curricular}{ El sistema permitirá enviar un mapa curricular con observaciones hechas por la DES a la unidad académica, para su corrección en esta. Para los campos que estén correctos y aprobados, se bloqueará su edición.}{M}{Sistema}
    \caption{Requerimientos funcionales del sistema.}
    \label{tbl:RFAB}

	\end{requerimientos}
