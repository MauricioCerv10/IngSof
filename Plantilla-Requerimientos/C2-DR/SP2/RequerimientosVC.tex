% -------------- TABLA PARA REQUERIMIENTOS FUNCIONALES ---------------- % 
% Nomenclatura para la prioridad: 
%	MA - Muy Alta
%	A - Alta
%	M - Media
%	B - Baja
%	MB - Muy Baja
JDDIC : Jefe del Departamento de Desarrollo e Innovación Curricular.\\
JDIA: Jefe del Departamento de Innovación Académica. Es el jefe del JDDIC.\\
Propuesta de unidad de aprendizaje: documento con secciones las cuales se revisan y pueden tener comentarios o no.\\
Comentario u Observación: cuando no se cumple algún punto de una sección de la propuesta de unidad de aprendizaje se agrega un comentario explicando por qué no se cumple dicho punto y como cambiarlo.\\
Sección: parte del documento que agrupa información relacionada.\\

\begin{table}[htbp!]
	\begin{requerimientos}
		\FRitem{SP2-U1}{Visualización de propuestas}{El usuario analista, JDDIC y JDIA podrán visualizar  el contenido  de la propuesta de cada unidad de aprendizaje.}{A}{Usuario}
		\FRitem{SP2-U2}{Recibo de paquetes}{El usuario JDDIC y JDIA podrán recibir los paquetes de propuestas de unidades de aprendizaje del jefe de innovación educativa de la Unidad Académica correspondiente.}{A}{Usuario}
		\FRitem{SP2-U2.1}{Asignación de paquetes}{El usuario JDDIC podrá asignar los paquetes recibidos de la Unidad Académica correspondiente a sus analistas.}{A}{Usuario}
		\FRitem{SP2-U2.2}{Desasignar paquetes}{El usuario JDDIC podrá desasignar los paquetes recibidos de la Unidad Académica correspondiente previamente asignados a un analista.}{A}{Usuario}
		\FRitem{SP2-U3}{Envío de paquetes}{El usuario JDDIC podrá mandar los paquetes al jefe de innovación educativa de la unidad académica correspondiente.}{A}{Usuario}
		\FRitem{SP2-U4}{Creación de comentarios}{El usuario analista, JDIA y JDDIC  podrán agregar comentarios a las partes no aprobadas de la propuesta de cada unidad de aprendizaje.}{A}{Usuario}
		\FRitem{SP2-U4.1}{Eliminación de comentarios}{El usuario analista, JDIA y JDDIC  podrán eliminar los comentarios previamente realizados en la propuesta de cada unidad de aprendizaje que se le asignó.}{A}{Usuario}
		\FRitem{SP2-U4.2}{Modificación de comentarios}{El usuario analista, JDIA y JDDIC  podrán modificar los comentarios previamente realizados en la propuesta de cada unidad de aprendizaje que se le asignó.}{A}{Usuario}
		\FRitem{SP2-U4.3}{Agregar Subrayado de texto}{Los usuarios analista, JDIA y JDDIC podrán  agregar un subrayado a partes de la propuesta de cada unidad de aprendizaje que se le asignó que no han sido previamente aprobadas.}{M}{Usuario}
		\FRitem{SP2-U4.4}{Quitar Subrayado de texto}{Los usuarios analista, JDIA y JDDIC podrán quitar subrayados previamente agregados en el documento.}{M}{Usuario}
		\FRitem{SP2-U5}{Autorización de secciones}{El analista, JDIA y JDDIC podrá autorizar secciones de la propuesta de unidad de aprendizaje.}{M}{Usuario}
		\FRitem{SP2-U5.1}{Identificador de usuarios}{A los analistas, JDIA y el JDDIC se les agregará una matrícula única con las que se les identifica como autores de los comentarios en la propuesta de unidad de aprendizaje.}{M}{Usuario}
		\FRitem{SP2-U5.2}{Acceso de documentos asignados}{El usuario analista sólo podrá añadir comentarios a los documentos que le fueron asignados por el usuario JDDIC.}{M}{Usuario}
		\FRitem{SP2-NF5.3}{Bloqueo de secciones autorizadas}{El sistema impedirá agregar comentarios de las secciones del documento aprobadas anteriormente.}{B}{Origen}
		\FRitem{SP2-F5.4}{Registro de hora de actualización}{El sistema guardará la fecha y hora de cada modificación realizada en el área de comentarios por los usuarios analista y el JDDIC en el documento de propuesta de unidad de aprendizaje.}{B}{Origen}
	\end{requerimientos}
    \caption{Requerimientos funcionales del sistema.}
    \label{tbl:}
\end{table}
