

\begin{UseCase}{SP5-CU8}{Consultar usuarios del Jefe de División de Innovación Académica. }{El  Jefe de División de Innovación Académica podrá visualizar la información de los empleados.}
        \UCitem{Versión}{\color{Gray}2.0}
        \UCitem{Autor}{\color{Gray}Hernández Ruiz Rafael}
        \UCitem{Supervisa}{\color{Gray}Abigail Nicolás Sayago}
        \UCitem{Actor}{\hyperlink{JDIC}{JDIC}, \hyperlink{JDIA}{JDIA}, \hyperlink{JUA}{JUA}.}
        \UCitem{Propósito}{Ver los empleados que cada actor posee, así como realizar trabajos de gestión: añadir, editar y eliminar usuarios.}
        \UCitem{Entradas}{
          \begin{itemize}
            \item Selección del cargo de los empleados a buscar.
            \item Clic en botón buscar.
            \item Clic en botón x.
            \item Clic en botón Cancelar.
            \item Clic en botón Aceptar.
        \end{itemize}}
        \UCitem{Origen}{Mouse.}
        \UCitem{Salidas}{
            \begin{itemize}
                \item Lista de empleados de un cargo en específico con sus datos (cargo, nombre, matricula, titulo unidad académica). 
                \item \label{MSGR4} MSGR4. No se han cargado los catálogos.
                \item \label{MSGR2} MSGR2. No hay usuarios registrados con ese cargo.
                \item \label{MSGR3} MSGR3. ¿Seguro de eliminar al empleado ''Nombre'' con matrícula: ''Matrícula'' del sistema?. 
            \end{itemize}
        }
        \UCitem{Destino}{Pantalla.}
        \UCitem{Precondiciones}{ 1.- Debe existir por lo menos un registro en el catálogo de la BRR1.}
        \UCitem{Postcondiciones}{1.- Habilita la llamada a los casos de uso  SP5-CU10, SP5-CU11.}
        \UCitem{Errores}{ \begin{itemize}
        \item El catálogo de cargos no se cargo correctamente.
        \item Hubo un problema al conectarse con el servidor.
        \item Hubo un problema al conectarse con la base de datos. \end{itemize}}
        \UCitem{Estado}{Gestión.}
        \UCitem{Observaciones}{}
        \UCitem{Puntos de extensión}{Casos de uso  \UCref{SP5-CU10} , \UCref{SP5-CU11}. }
\end{UseCase}

\begin{UCtrayectoria}{Principal}
    
    \UCpaso[\UCactor] Presiona en la interfaz de usuario \IUref{IU1}{Menú} la opción de gestionar usuarios. 
    \UCpaso  El sistema verifica la existencia de registros del catalogo. [Trayectoria B] 
    \UCpaso El sistema carga la pantalla  \IUref{IU2}{Gestionar empleados}.
    \UCpaso[\UCactor] Selecciona el cargo de los empleados a buscar. 
    \UCpaso[\UCactor]  Presiona el botón de \IUbutton{Buscar}. [Trayectoria D] [Trayectoria C] [Trayectoria E]
    
    \UCpaso El sistema despliega la información de los empleados permitidos para el Jefe de División de Innovación Académica de acuerdo a la regla de negocio SP5-RNX (cargo, nombre, matricula, titulo, lugar de trabajo) en la parte inferior de la pantalla \IUref{IU2}{Gestionar empleados}. [Trayectoria Principal punto 2] [Trayectoria F]
\end{UCtrayectoria}

\begin{UCtrayectoriaA}{B}{No existen registros en el catálogo de cargos.}
    \UCpaso     El sistema muestra el \hyperref[MSGR4]{MSGR4. No se han cargado los catálogos}.
\end{UCtrayectoriaA}

\begin{UCtrayectoriaA}{C}{No existen  empleados con el cargo seleccionado.}
    \UCpaso     El sistema muestra el \hyperref[MSGR2]{MSGR2. No hay usuarios registrados con ese cargo}.
\end{UCtrayectoriaA}

\begin{UCtrayectoriaA}{D}{El actor es el Jefe de Departamento e Innovación Curricular.}
 \UCpaso El sistema despliega la información  de los empleados permitidos para el Jefe de Departamento e Innovación Curricular de acuerdo a la BRXX (cargo, nombre, matricula, titulo, lugar de trabajo) en la parte inferior de la pantalla \IUref{IU2}{Gestionar empleados}. [Trayectoria F]
  \UCpaso Continua en el paso 2 de la trayectoria principal del \UCref{SP5-CU8}.
\end{UCtrayectoriaA}

\begin{UCtrayectoriaA}{E}{El actor es el JUA.}
 \UCpaso El sistema despliega la información  de los empleados permitidos para el JUA de acuerdo a la BRXX (cargo, nombre, matricula, titulo, lugar de trabajo) en la parte inferior de la pantalla \IUref{IU2}{Gestionar empleados}.[Trayectoria F]
 \UCpaso Continua en el paso 2 de la trayectoria principal del \UCref{SP5-CU8}.
\end{UCtrayectoriaA}

\begin{UCtrayectoriaA}{F}{El actor presiona el botón \IUbutton{X}.}
    \UCpaso El sistema muestra el mensaje \hyperref[MSGR3]{MSGR3. ¿Seguro de eliminar al empleado ''Nombre'' con matrícula: ''Matrícula'' del sistema?} solicitando confirmación. [Trayectoria G]
    \UCpaso[\UCactor] El actor presiona el botón \IUbutton{Cancelar}.
\end{UCtrayectoriaA}

\begin{UCtrayectoriaA}{G}{El actor presiona el botón \IUbutton{Aceptar}.}
    \UCpaso     El sistema elimina al empleado.  
\end{UCtrayectoriaA}

