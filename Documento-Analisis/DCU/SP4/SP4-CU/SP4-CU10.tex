% EDITAR PLAN DE ESTUDIOS.
\begin{UseCase}{SP4-CU10}{Editar Plan de Estudios}{El usuario Docente modifica los datos del Plan de Estudios registrado en el sistema.}
		\UCitem{Versión}{\color{Gray}1.0}
		\UCitem{Autor}{\color{Gray}Cervantes Moreno Christian Andres}
		\UCitem{Supervisa}{\color{Gray} Evelyn Reyes}
		\UCitem{Actor}{\hyperlink{Usuario}{Docente}}
		\UCitem{Propósito}{Editar el año, modalidad, créditos totales TEPIC, créditos totales SATCA, Total Horas/Teoría, Total Horas/Práctica, del plan de estudio registrado en el sistema.}
		\UCitem{Entradas}{Las entradas para la modificación de la Unidad de Aprendizaje serán:
          \begin{itemize}
          	\item Programa Académico.
          	\item Año.
          	\item Modalidad.
          	\item Créditos Totales TEPIC.
            \item Créditos Totales SATCA.
            \item Total horas / Teoría.
            \item Total horas / Práctica.
          \end{itemize}
        }
		\UCitem{Origen}{Teclado, Mouse}
		\UCitem{Salidas}{
        	\begin{itemize}
        		\item \MSGref{MSG32}{Todos los campos son obligatorios}
                \item \MSGref{MSG31}{Los cambios se guardaron exitosamente.}
                \item \MSGref{MSG29}{¿Está seguro que desea cancelar? Se perderán todos los avances sin guardar.}
                \item \MSGref{MSG7}{Los catálogos necesarios no se han cargado, favor de intentarlo más tarde.}
				\item \MSGref{MSG9}{Por el momento no se puede realizar el registro.}
				\item \MSGref{MSG35}{Inconsistencia en los datos. Verifique los campos e intente de nuevo.}
        	\end{itemize}
        }
		\UCitem{Destino}{Pantalla.}
		\UCitem{Precondiciones}{ Debe de existir al menos un Plan de Estudios registrado en el sistema.}
		\UCitem{Postcondiciones}{El plan de Estudios queda modificado en el sistema.}
		\UCitem{Errores}{
			  \begin{itemize}
				\item E1. El catálogo de modalidad no se cargaron correctamente.
				\item E2. Hubo un problema al conectarse con la base de datos.
			\end{itemize}
		}
		\UCitem{Estado}{Revisión.}
		\UCitem{Observaciones}{}
\end{UseCase}
%--------------------------- CU TRAYECTORIA PRINCIPAL -------------------------
\begin{UCtrayectoria}{Principal}
    %Usuario
    \UCpaso[\UCactor] Presiona el botón \IUbutton{icono de lapiz}de la Interfaz de usuario \IUref{IU2.1-D}{Consultar Mapa Curricular}.
	%Sistema
    \UCpaso El sistema carga el catálogo del modalidades segun la \BRref{BR14}{Existen los catálogos}.[Trayectoria A]
    \UCpaso Muestra la interfaz de usuario \IUref{IU2.1.1-D}{Registrar Plan de Estudios}.
    \UCpaso[\UCactor] Camptura la información necesaria  que desea modificar empleando la \BRref{BR33}{Información del Plan de Estudios}.
    \UCpaso[\UCactor] Termina la operación presionando el botón \IUbutton{Registrar}. [Trayectoria B] [Trayectoria C]
    \UCpaso Verifica que todos los campos marcados como obligatorios hayan sido completamente contestados segun la \BRref{BR13}{Todos los datos solicitados son obligatorios}. [Trayectoria D]
    \UCpaso Guarda la información de del Plan de Estudios.  [Trayectoria E]
    \UCpaso El sistema muestra el mensaje \MSGref{MSG31}{Los cambios se guardaron exitosamente.}.
    \UCpaso[\UCactor] Cierra el mensaje presionando el botón \IUbutton{Aceptar}.
    \UCpaso Muestra la interfaz de usuario \IUref{IU2.1-D}{Consultar Mapa Curricular}.
\end{UCtrayectoria}
%------------------------ CU TRAYECTORIA ALTERNARIVA A -------------------------
\begin{UCtrayectoriaA}{A}{El sistema no carga el catálogo del Plan de Estudios}
	%\UCpaso[\UCactor] Presiona el botón \IUbutton{$\bigoplus$} que se encuentra a un lado del campo ``Nombre'' del formulario \IUref{IU2}{Registro de bibliografía}  [Trayectoria A.1]
	\UCpaso[\UCactor] Presiona el botón \IUbutton{Registrar}
	\UCpaso Muestra el mensaje \MSGref{MSG7}{Los catálogos necesarios no se han cargado, favor de intentarlo más tarde. }
	\UCpaso[\UCactor] Cierra el mensaje presionando el botón \IUbutton{Ok}.
	 \UCpaso Muestra la interfaz de usuario \IUref{IU2.1-D}{Consultar Mapa Curricular}
\end{UCtrayectoriaA}
%------------------------ CU TRAYECTORIA ALTERNARIVA B -------------------------
\begin{UCtrayectoriaA}{B}{El actor presiona el botón \IUbutton{Cancelar}}
	\UCpaso Muestra el mensaje \MSGref{MSG29}{¿Está seguro que desea cancelar? Se perderán todos los avances sin guardar.}.
	\UCpaso[\UCactor] Cierra el mensaje presionando el botón \IUbutton{Si}.
	\UCpaso Muestra la interfaz de usuario \IUref{IU2.1-D}{Consultar Mapa Curricular}
\end{UCtrayectoriaA}
%------------------------ CU TRAYECTORIA ALTERNARIVA C -------------------------
\begin{UCtrayectoriaA}{C}{El actor presiona accidentalmente el botón \IUbutton{Cancelar}}
	\UCpaso Muestra el mensaje \MSGref{MSG29}{¿Está seguro que desea cancelar? Se perderán todos los avances sin guardar.}.
	\UCpaso[\UCactor] Cierra el mensaje presionando el botón \IUbutton{No}.
	\UCpaso Continúa en el paso 9 de la trayectoria principal del \UCref{SP4-CU3}.
\end{UCtrayectoriaA}
%------------------------ CU TRAYECTORIA ALTERNARIVA D -------------------------
\begin{UCtrayectoriaA}{D}{El sistema detecta uno o más campos sin contestar.}
	\UCpaso Muestra el mensaje \MSGref{MSG32}{Todos los campos son obligatorios}.
	\UCpaso[\UCactor] Cierra el mensaje presionando el botón \IUbutton{Aceptar}.
	\UCpaso Continúa en el paso 2 de la trayectoria principal del \UCref{SP4-CU2}.
\end{UCtrayectoriaA}

%------------------------ CU TRAYECTORIA ALTERNARIVA E -------------------------
\begin{UCtrayectoriaA}{E}{La base de datos no está disponible, o hubo un error de conexión.}
	\UCpaso Muestra el mensaje \MSGref{MSG25}{Servicios no disponibles por el momento.}
	\UCpaso[\UCactor] Cierra el mensaje presionando el botón \IUbutton{Ok}.
	\UCpaso El sistema regresa a la interfaz de usuario \IUref{IU2.n-D}{Consultar Planes de Estudio}
\end{UCtrayectoriaA}