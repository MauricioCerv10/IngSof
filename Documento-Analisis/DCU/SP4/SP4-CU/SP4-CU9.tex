% EDITAR UNIDAD DE APRENDIZAJE.
\begin{UseCase}{SP4-CU9}{Editar Unidad de Aprendizaje}{El usuario Docente modifica los datos de las Unidades de Aprendizaje registradas en el sistema..}
		\UCitem{Versión}{\color{Gray}1.0}
		\UCitem{Autor}{\color{Gray}Cervantes Moreno Christian Andres}
		\UCitem{Supervisa}{\color{Gray} Evelyn Reyes}
		\UCitem{Actor}{\hyperlink{Usuario}{Docente}}
		\UCitem{Propósito}{Editar el nombre, créditos TEPIC, créditos SATCA, horas Teoría/Semana, horas Práctica/semana, Área de formación y semestre de una Unidad de Aprendizaje.}
		\UCitem{Entradas}{Las entradas para la modificación de la Unidad de Aprendizaje serán:
          \begin{itemize}
          	\item Nombre (Tipo carácter).
          	\item Créditos Totales TEPIC (Tipo double).
            \item Créditos Totales SATCA (Tipo double).
            \item Total horas / Teoría (Tipo entero).
            \item Total horas / Práctica (Tipo entero).
            \item Área de Formación.
            \item Semestre
          \end{itemize}
        }
		\UCitem{Origen}{Teclado, Mouse}
		\UCitem{Salidas}{
        	\begin{itemize}
        		\item \MSGref{MSG32}{Todos los campos son obligatorios.}
                \item \MSGref{MSG31}{Los cambios se guardaron exitosamente.}
                \item \MSGref{MSG29}{¿Está seguro que desea cancelar? Se perderán todos los avances sin guardar.}
                \item \MSGref{MSG7}{Los catálogos necesarios no se han cargado, favor de intentarlo más tarde.}
				\item \MSGref{MSG9}{Por el momento no se puede realizar el registro.}

        	\end{itemize}
        }
		\UCitem{Destino}{Pantalla.}
		\UCitem{Precondiciones}{ Debe de existir al menos una Unidad de Aprendizaje registrada en el sistema.}
		\UCitem{Postcondiciones}{La Unidad de Aprendizaje queda modificada en el sistema.}
		\UCitem{Errores}{
			  \begin{itemize}
				\item E1. Entrada invalida de un caracter no entero.
				\item E2. Entrada invalida de un caracter no double.
				\item E2. Entrada invalida de un carácter
			\end{itemize}
		}
		\UCitem{Estado}{Revisión.}
		\UCitem{Observaciones}{}
\end{UseCase}

%--------------------------- CU TRAYECTORIA PRINCIPAL -------------------------
\begin{UCtrayectoria}{Principal}

    %Usuario
    \UCpaso[\UCactor] Presiona el checkbox Editar de la Unidad de Aprendizaje que desea editar de la Interfaz de usuario  \IUref{IU2.1-D}{Consultar Mapa Curricular}.

	%Sistema
    \UCpaso El sistema carga el catálogo de la Unidad de Aprendizaje.[Trayectoria A]


    \UCpaso Muestra la interfaz de usuario \IUref{IU2.1.2-D}{Registrar Unidad de Aprendizaje}.
    \UCpaso[\UCactor] Elige los campos que desea modificar.[Trayectoria B].
    \UCpaso[\UCactor] Termina la operación presionando el botón \IUbutton{Guardar}. [Trayectoria B] [Trayectoria C]
    \UCpaso Verifica que todos los campos marcados como obligatorios hayan sido completamente contestados. [Trayectoria D]

    \UCpaso Guarda la información de la Unidad de Aprendizaje en la base de datos.

    \UCpaso El sistema muestra el mensaje \MSGref{MSG31}{Los cambios se guardaron exitosamente.}.

    \UCpaso[\UCactor] Cierra el mensaje presionando el botón \IUbutton{Aceptar}.

    \UCpaso Muestra la interfaz de usuario \IUref{IU2.1-D}{Consultar Mapa Curricular}.
\end{UCtrayectoria}

%------------------------ CU TRAYECTORIA ALTERNARIVA X -------------------------

\begin{comment}
\begin{UCtrayectoriaA}{A}{El sistema no encuentra ningún formulario para mostrar.}
	\UCpaso No encuentra ningún formulario para mostrar.
    \UCpaso El sistema muestra el mensaje \MSGref{MSG9}{Por el momento no se puede realizar el registro.}.
    \UCpaso[\UCactor] Cierra el mensaje presionando el botón \IUbutton{Aceptar}.
    \UCpaso Continua en el paso 1 de la trayectoria principal del \UCref{CU1}.
\end{UCtrayectoriaA}
\end{comment}

%------------------------ CU TRAYECTORIA ALTERNARIVA A -------------------------

\begin{UCtrayectoriaA}{A}{El sistema no carga el catálogo de la Unidad de Aprendizaje}
	\UCpaso[\UCactor] Presiona el botón \IUbutton{registrar}
	\UCpaso Muestra el mensaje \MSGref{MSG7}{Los catálogos necesarios no se han cargado, favor de intentarlo más tarde.}
	\UCpaso[\UCactor] Confirma la operación presionando el botón \IUbutton{Ok}.
	 \UCpaso Muestra la interfaz de usuario \IUref{IU2.1-D}{Consultar Mapa Curricular}

\end{UCtrayectoriaA}

%------------------------ CU TRAYECTORIA ALTERNARIVA B.1 -------------------------

\begin{UCtrayectoriaA}{B.1}{El usuario desea modificar el nombre de la Unidad de Aprendizaje.}
	\UCpaso[\UCactor] Borra el nombre de la Unidad de Aprendizaje.
	\UCpaso[\UCactor] Ingresa el nombre de la Unidad de Aprendizaje.
	\UCpaso Continúa en el paso 5 de la trayectoria principal del \UCref{SP4-CU9}.
\end{UCtrayectoriaA}

%------------------------ CU TRAYECTORIA ALTERNARIVA B.2 -------------------------

\begin{UCtrayectoriaA}{B.2}{El usuario desea modificar los créditos TEPIC de la Unidad de Aprendizaje.}
	\UCpaso[\UCactor] Borra los créditos TEPIC de la Unidad de Aprendizaje.
	\UCpaso[\UCactor] Ingresa los nuevos créditos TEPIC la Unidad de Aprendizaje.
	\UCpaso Continúa en el paso 5 de la trayectoria principal del \UCref{SP4-CU9}.
\end{UCtrayectoriaA}


%------------------------ CU TRAYECTORIA ALTERNARIVA B.3 -------------------------

\begin{UCtrayectoriaA}{B.3}{El usuario desea modificar los créditos SATCA de la Unidad de Aprendizaje.}
	\UCpaso[\UCactor] Borra los créditos SATCA de la Unidad de Aprendizaje.
	\UCpaso[\UCactor] Ingresa los nuevos créditos SATCA la Unidad de Aprendizaje.
	\UCpaso Continúa en el paso 5 de la trayectoria principal del \UCref{SP4-CU9}.
\end{UCtrayectoriaA}


%------------------------ CU TRAYECTORIA ALTERNARIVA B.4 -------------------------

\begin{UCtrayectoriaA}{B.4}{El usuario desea modificar las horas Teoría/Semana de la Unidad de Aprendizaje.}
	\UCpaso[\UCactor] Borra las horas Teoría/Semana de la Unidad de Aprendizaje.
	\UCpaso[\UCactor] Ingresa las nuevas horas Teoría/Semana SATCA la Unidad de Aprendizaje.
	\UCpaso Continúa en el paso 5 de la trayectoria principal del \UCref{SP4-CU9}.
\end{UCtrayectoriaA}



%------------------------ CU TRAYECTORIA ALTERNARIVA B.5 -------------------------

\begin{UCtrayectoriaA}{B.5}{El usuario desea modificar las horas Práctica/Semana de la Unidad de Aprendizaje.}
	\UCpaso[\UCactor] Borra las horas Práctica/Semana de la Unidad de Aprendizaje.
	\UCpaso[\UCactor] Ingresa las nuevas horas Práctica/Semana SATCA la Unidad de Aprendizaje.
	\UCpaso Continúa en el paso 5 de la trayectoria principal del \UCref{SP4-CU9}.
\end{UCtrayectoriaA}



%------------------------ CU TRAYECTORIA ALTERNARIVA B.6 -------------------------

\begin{UCtrayectoriaA}{B.6}{El usuario desea modificar el área de formación de la Unidad de Aprendizaje.}
	\UCpaso[\UCactor] Selecciona el área de formación al que pertenece la Unidad de Aprendizaje.
	\UCpaso Continúa en el paso 5 de la trayectoria principal del \UCref{SP4-CU9}.
\end{UCtrayectoriaA}


%------------------------ CU TRAYECTORIA ALTERNARIVA B.7 -------------------------

\begin{UCtrayectoriaA}{B.7}{El usuario desea modificar el área de formación de la Unidad de Aprendizaje.}
	\UCpaso[\UCactor] Selecciona el semestre al que pertenece la Unidad de Aprendizaje.
	\UCpaso Continúa en el paso 5 de la trayectoria principal del \UCref{SP4-CU9}.
\end{UCtrayectoriaA}



%------------------------ CU TRAYECTORIA ALTERNARIVA C -------------------------

\begin{UCtrayectoriaA}{C}{El actor quiere cancelar el registro de Unidad de Aprendizaje.}
	\UCpaso[\UCactor] Presiona el botón \IUbutton{Cancelar}.
	\UCpaso Muestra el mensaje \MSGref{MSG29}{¿Está seguro que desea cancelar? Se perderán todos los avances sin guardar.}.
	\UCpaso[\UCactor] Confirma la operación presionando el botón \IUbutton{Si}.
	\UCpaso Muestra la interfaz de usuario \IUref{IU2.1-D}{Consultar Mapa Curricular}
\end{UCtrayectoriaA}



%------------------------ CU TRAYECTORIA ALTERNARIVA D -------------------------

\begin{UCtrayectoriaA}{D}{El actor presiona accidentalmente el botón Cancelar}
	\UCpaso[\UCactor] Presiona el botón \IUbutton{Cancelar}
	\UCpaso Muestra el mensaje \MSGref{MSG29}{ ¿Está seguro que desea cancelar? Se perderán todos los avances sin guardar.}.
	\UCpaso[\UCactor] Presiona el botón \IUbutton{No}.
	\UCpaso Cierra el mensaje.
	\UCpaso Continúa en el paso 9 de la trayectoria principal del \UCref{SP4-CU3}.
\end{UCtrayectoriaA}



%------------------------ CU TRAYECTORIA ALTERNARIVA E -------------------------

\begin{UCtrayectoriaA}{E}{Uno o más campos obligatorios no fueron contestados.}
	\UCpaso Detecta uno o más campos sin contestar.
	\UCpaso Muestra el mensaje \MSGref{MSG32}{Todos los campos son obligatorios.}
	\UCpaso[\UCactor] Cierra el mensaje presionando el botón \IUbutton{Aceptar}.
	\UCpaso Continúa en el paso 2 de la trayectoria principal del \UCref{SP4-CU2}.
\end{UCtrayectoriaA}