% REGISTRAR BIBLIOGRAFÍA.
\begin{UseCase}{SP4-CU17}{Generar Tarea}{El usuario Jefe de Innovación Educativa registra nuevas Tareas en el sistema.}
        \UCitem{Versión}{\color{Gray}1.1}
        \UCitem{Autor}{\color{Gray}Rivas Rojas Arturo}
        \UCitem{Supervisa}{\color{Gray}}
        \UCitem{Actor}{\hyperlink{JDIE}{Jefe de Innovación Educativa}}
        \UCitem{Propósito}{Agregar Tareas de registro al sistema y asignárselas a un Docente.}
        \UCitem{Entradas}{Las entradas para el registro de una Tarea serán:
            \begin{itemize}
                \item Matrícula de Empleado.
                \item Tarea(tipo).
                \item Programa Académico.
                \item Unidad de Aprendizaje (de ser necesaria).
                \item Fecha de Entrega.
            \end{itemize}
        }
        \UCitem{Origen}{Teclado.}
        \UCitem{Salidas}{
            \begin{itemize}
                \item MSG1. Todos los campos son obligatorios.
                \item MSG2. Registro finalizado exitosamente.
                \item MSG3. La fecha que ha ingresado es inválida.
                \item MSG5. ¿Está seguro que desea cancelar?
                \item MSG6. Los catálogos necesarios no están disponibles por el momento, por favor inténtelo mas tarde.
                \item MSG7 Ocurrio un error, porfavor intentelo mas tarde.
            \end{itemize}
        }
        \UCitem{Destino}{Pantalla.}
        \UCitem{Precondiciones}{
            \begin{itemize}
                \item Debe de haber mínimo un Docente así como un Programa Académico previamente registrados en el sistema.
                \item El catálogo de Tarea.
            \end{itemize}
        }
        \UCitem{Postcondiciones}{La Tarea será asignada al Usuario y será posible consultar su estado.}
        \UCitem{Errores}{
            \begin{itemize}
                \item E1. No existen o no pudieron encontrarse los catálogo necesarios.
                \item E2. Problemas de conexión al persistir los datos.
            \end{itemize}
        }
        \UCitem{Estado}{Revisión.}
        \UCitem{Observaciones}{}
\end{UseCase}

%--------------------------- CU TRAYECTORIA PRINCIPAL -------------------------
\begin{UCtrayectoria}{Principal}

    \UCpaso[\UCactor] Presiona el botón \IUbutton{Generar Tarea} de la interfaz de usuario \IUref{IU2.3-J}{Asignar Tareas}
    \UCpaso Carga el catálogo de tipo de Tarea. [Trayectoria A]
    \UCpaso Carga las Matrículas de Empleado de los Docentes de la Unidad Académica. [Trayectoria A]
    \UCpaso Carga los Programas Académicos de la Unidad Académica. [Trayectoria A]
    \UCpaso Muestra la interfaz de usuario \IUref{IU2.3.2-J}{Generar Tarea}.
    \UCpaso[\UCactor] Selecciona la Matrícula del Empleado que al que desea asignar la Tarea.
    \UCpaso[\UCactor] Selecciona el Tipo de Tarea.[Trayectoria B]
    \UCpaso[\UCactor] Selecciona el Programa Académico.
    \UCpaso[\UCactor] Selecciona la Fecha de Entrega.
    \UCpaso[\UCactor] Termina la operación presionando el botón \IUbutton{Guardar}. [Trayectoria C]
    \UCpaso Verifica que todos los campos marcados como obligatorios hayan sido completamente contestados. [Trayectoria D]
    \UCpaso Verifica que la fecha sea válida. [Trayectoria E]
    \UCpaso Persiste los datos ingresados.
    \UCpaso Muestra el mensaje \MSGref{MSG2}{Registro finalizado exitosamente}.
    \UCpaso[\UCactor] Cierra el mensaje presionando el botón \IUbutton{Aceptar}.
    \UCpaso Muestra la interfaz de usuario \IUref{IU2.3-J}{Asignar Tareas}.
\end{UCtrayectoria}

%------------------------ CU TRAYECTORIA ALTERNARIVA A -------------------------

\begin{UCtrayectoriaA}{A}{Los catálogos no se pudieron cargar.}
    \UCpaso Muestra el mensaje \MSGref{MSG6}{Los catálogos necesarios no están disponibles por el momento, por favor inténtelo mas tarde}.
    \UCpaso[\UCactor] Cierra el mensaje presionando el botón \IUbutton{Aceptar}.
\end{UCtrayectoriaA}

%------------------------ CU TRAYECTORIA ALTERNARIVA B -------------------------

\begin{UCtrayectoriaA}{B}{El actor selecciona como tipo de Tarea Registrar Unidad de Aprendizaje.}
    \UCpaso Muestra el campo Unidad de Aprendizaje.
    \UCpaso[\UCactor] Selecciona la Unidad de Aprendizaje que se desea asignar.
    \UCpaso Continúa en el paso 8 de la trayectoria principal del \UCref{SP4-CU17}.
\end{UCtrayectoriaA}

%------------------------ CU TRAYECTORIA ALTERNARIVA C -------------------------

\begin{UCtrayectoriaA}{C}{El actor presiona el botón Cancelar}
    \UCpaso Muestra el mensaje \MSGref{MSG5}{¿Seguro que desea cancelar el registro?}.
    \UCpaso[\UCactor] ierra el mensaje presionando el botón \IUbutton{Si}.
    \UCpaso Muestra la interfaz de usuario \IUref{IU2.3-J}{Asignar Tareas}.
\end{UCtrayectoriaA}

%------------------------ CU TRAYECTORIA ALTERNARIVA C.1 -------------------------

\begin{UCtrayectoriaA}{C}{El actor presiona accidentalmente el botón Cancelar}
    \UCpaso Muestra el mensaje \MSGref{MSG5}{¿Está seguro que desea cancelar el registro?}.
    \UCpaso[\UCactor] Cierra el mensaje presionando el botón \IUbutton{No}.
    \UCpaso Continúa en el paso 10 de la trayectoria principal del \UCref{SP4-CU17}.
\end{UCtrayectoriaA}

%------------------------ CU TRAYECTORIA ALTERNARIVA D -------------------------

\begin{UCtrayectoriaA}{D}{Uno o más campos no fueron contestados.}
    \UCpaso Muestra el mensaje \MSGref{MSG1}{Todos los campos son obligatorios}.
    \UCpaso[\UCactor] Cierra el mensaje presionando el botón \IUbutton{Aceptar}.
    \UCpaso Continúa en el paso 6 de la trayectoria principal del \UCref{SP4-CU17}.
\end{UCtrayectoriaA}

%------------------------ CU TRAYECTORIA ALTERNARIVA E -------------------------

\begin{UCtrayectoriaA}{E}{La fecha que se ingresó es una fecha pasada.}
    \UCpaso Muestra el mensaje \MSGref{MSG3}{La fecha que ha ingresado es inválida}.
    \UCpaso[\UCactor] Cierra el mensaje presionando el botón \IUbutton{Aceptar}.
    \UCpaso Continúa en el paso 9 de la trayectoria principal del \UCref{SP4-CU17}.
\end{UCtrayectoriaA}

%------------------------ CU TRAYECTORIA ALTERNARIVA F -------------------------

\begin{UCtrayectoriaA}{F}{Ocurre un error al momento de persistir los datos.}
    \UCpaso Muestra el mensaje \MSGref{MSG7}{Ocurrio un error, porfavor intentelo mas tarde.}
    \UCpaso[\UCactor] Cierra el mensaje presionando el botón \IUbutton{Aceptar}.
    \UCpaso Muestra la interfaz de usuario \IUref{IU2.3-J}{Asignar Tareas}.
\end{UCtrayectoriaA}