% REGISTRAR BIBLIOGRAFÍA.
\begin{UseCase}{SP4-CU11}{Editar Recurso Humano}{El usuario Jefe de Innovación Educativa modifica la información general del Recurso Humano.}

		\UCitem{Versión}{\color{Gray}1.0}
		\UCitem{Autor}{\color{Gray}Plata García Josué Eliasaf}
		\UCitem{Supervisa}{\color{Gray}}
		\UCitem{Actor}{\hyperlink{Usuario}{Jefe de Innovación Educativa}}
		\UCitem{Propósito}{Cambiar los datos de un Recurso Humano.}
		\UCitem{Entradas}{Las entradas para Editar Recurso Humano:
          \begin{itemize}
          	\item Matrícula de Empleado (Tipo Alfanumérico)
          	\item Nombre (Tipo Alfanumérico)
          	\item Primer Apellido (Tipo Alfanumérico)
          	\item Segundo Apellido (Tipo Alfanumérico)
          	\item Título (Tipo Alfanumérico)
          	\item Cargo (Tipo Alfanumérico) 	
          	\item Lugar de Trabajo (Tipo Alfanumérico)
          	
          \end{itemize}
        }
		\UCitem{Origen}{Teclado.}
		\UCitem{Salidas}{
        	\begin{itemize}{
        		\item \MSGref {MSG4} {Los campos marcados con (*) son obligatorios.}
                \item \MSGref {MSG31}{ Los cambios se guardaron exitosamente.}
                \item \MSGref {MSG29} {¿Está seguro que desea cancelar? Se perderán todos los avances sin guardar.}
                \item \MSGref {MSG7} {Los catálogos necesarios no se han cargado, favor de intentarlo más tarde.}
				\item \MSGref {MSG9} {Por el momento no se puede realizar el registro.}
				\itam \MSGref {MSG20}{ Los campos no fueron contestados correctamente.}
                \item \MSGref {MSG26} {La matrícula ingresada no se encuentra registrada en el sistema.}
        	\end{itemize}
        }
		\UCitem{Destino}{Pantalla.}
		\UCitem{Precondiciones}{}
		\UCitem{Postcondiciones}{El Recurso Humano quedó actualizado en el sistema, permitiendo consultarlo y generar tareas de registro del mismo o quedó sin modificarse.}
		\UCitem{Errores}{E1. Algúna entrada  contiene uno o más caracteres no alfanuméricos.
		E2. No se encontró el Recurso Humano.
		}
		\UCitem{Estado}{Revisión.}
		\UCitem{Observaciones}{Ninguna matrícula de empleado es igual.}
\end{UseCase}

%--------------------------- CU TRAYECTORIA PRINCIPAL -------------------------
\begin{UCtrayectoria}{Principal}

    \UCpaso[\UCactor] Presiona el botón \IUbutton{Editar} de la interfaz de usuario \IUref{IU2.4-J}{Consultar Recurso Humano}

    \UCpaso Muestra la interfaz de usuario \IUref{IU2.4.1-J}{Editar Recurso Humano}.

	\UCpaso Ingresa la matrícula del Recurso Humano. [Trayectoria F]    
    
    \UCpaso Cargar datos del Recurso Humano correspondiente. [Trayectoria E]
    
    \UCpaso[\UCactor] Cambia los datos de Recurso Humano.

    \UCpaso[\UCactor] Termina la operación presionando el botón \IUbutton{Finalizar}. [Trayectoria A] [Trayectoria B]

    \UCpaso Verifica que todos los campos marcados como obligatorios hayan sido contestados. [Trayectoria C][Trayectoria D]

    \UCpaso Guarda la información del Recurso Humano correspondiente.

    \UCpaso El sistema muestra el mensaje \MSGref{MSG31}{ Los cambios se guardaron exitosamente.}.

    \UCpaso[\UCactor] Cierra el mensaje presionando el botón \IUbutton{Aceptar}.

    \UCpaso Muestra la interfaz de usuario \IUref{IU2.4-J}{Consultar Recurso Humano}
\end{UCtrayectoria}


%------------------------ CU TRAYECTORIA ALTERNARIVA A -------------------------


\begin{UCtrayectoriaA}{A}{El actor quiere cancelar la edición  del Recurso Humano correspondiente.}
	\UCpaso[\UCactor] Presiona el botón \IUbutton{Cancelar}.
    \UCpaso Muestra el mensaje \MSGref{MSG29}{¿Está seguro que desea cancelar? Se perderán todos los avances sin guardar.}.
    \UCpaso[\UCactor] Confirma la operación presionando el botón \IUbutton{Si}.
    \UCpaso Muestra la interfaz de usuario \IUref{IU2.4-J}{Consultar Recurso Humano}
\end{UCtrayectoriaA}


%------------------------ CU TRAYECTORIA ALTERNARIVA B -------------------------

\begin{UCtrayectoriaA}{B}{El actor presiona accidentalmente el botón Cancelar}
	\UCpaso[\UCactor] Presiona el botón \IUbutton{Cancelar}
    \UCpaso Muestra el mensaje \MSGref{MSG29}{¿Está seguro que desea cancelar? Se perderán todos los avances sin guardar.}.
    \UCpaso[\UCactor] Presiona el botón \IUbutton{No}.
    \UCpaso Cierra el mensaje.
    \UCpaso Continúa en el paso 4 de la trayectoria principal del \UCref{SP4-CU11}.
\end{UCtrayectoriaA}

%------------------------ CU TRAYECTORIA ALTERNARIVA C -------------------------

\begin{UCtrayectoriaA}{C}{Uno o más campos obligatorios no fueron contestados.}
	\UCpaso Detecta uno o más campos sin contestar.
    \UCpaso Muestra el mensaje \MSGref{MSG4}{Los campos marcados con (*) son obligatorios.}.
    \UCpaso[\UCactor] Cierra el mensaje presionando el botón \IUbutton{Aceptar}.
    \UCpaso Continúa en el paso 4 de la trayectoria principal del \UCref{SP4-CU11}.
\end{UCtrayectoriaA}
%------------------------ CU TRAYECTORIA ALTERNARIVA D -------------------------

\begin{UCtrayectoriaA}{D}{Alguna entrada  contiene caracteres no alfanuméricos}
	\UCpaso El Sistema detecta caracteres no alfanuméricos E1.
    \UCpaso Muestra el mensaje \MSGref{MSG20}{ Los campos no fueron contestados correctamente.}.
    \UCpaso Continúa en el paso 4 de la trayectoria principal del \UCref{SP4-CU11}.
\end{UCtrayectoriaA}
    %------------------------ CU TRAYECTORIA ALTERNARIVA E -------------------------

\begin{UCtrayectoriaA}{E}{No cargan los catálogos de Recurso Humano.}
    \UCpaso Muestra el mensaje \MSGref{MSG7}{Los catálogos necesarios no se han cargado, favor de intentarlo más tarde.}.
    \UCpaso[\UCactor] Cierra el mensaje presionando el botón \IUbutton{Aceptar}.
\UCpaso Muestra la interfaz de usuario Menú para Jefe de Innovación Educativa(Pantalla de Abi).
\end{UCtrayectoriaA}
    %------------------------ CU TRAYECTORIA ALTERNARIVA F -------------------------

\begin{UCtrayectoriaA}{E}{No existe ningun Recurso Humano con esa matrícula.}
    \UCpaso Muestra el mensaje \MSGref{MSG26}{La matrícula ingresada no se encuentra registrada en el sistema.}.
    \UCpaso[\UCactor] Cierra el mensaje presionando el botón \IUbutton{Aceptar}.
\UCpaso Continua en el paso  de búsqueda de matrícula.
\end{UCtrayectoriaA}