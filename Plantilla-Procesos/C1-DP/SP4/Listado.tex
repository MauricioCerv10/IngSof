\documentclass[12pt,letterpaper]{article}
\usepackage[latin1]{inputenc}
\usepackage[spanish]{babel}
\usepackage{amsmath}
\usepackage{amsfonts}
\usepackage{amssymb}
\usepackage{graphicx}
\usepackage[left=2cm,right=2cm,top=2cm,bottom=2cm]{geometry}

\author{Andres Cervantes}
 \author{ 
 Plata Garc�a Josu� Eliasaf\\ 
 \and 
 Rivas Rojas Arturo\\ 
  \and 
 Cervantes Andres\\ 
} 
\title{\Huge Lista de Requerimientos}
\date{}
%comentario
\begin{document}
\maketitle

{\scshape\Large Subproceso para la elaboraci�n del Plan de Estudios\par}

\begin{enumerate}
\item Contar con la aprobaci�n de los contenidos curriculares.
\item Analizar cada uno de los objetivos curriculares y de los distintos aspectos del perfil de egreso.
\item Identificar los contenidos curriculares.
\item Verificar los objetivos curriculares.
\item Identificar los ejes tem�ticos curriculares (Practicas, Conocimientos, Actitudes).
\item Identificar las �reas del plan de estudio.
\item Identificar las diferentes formas de las unidades de aprendizaje.
\item Identificar y estructurar los segmentos que van a conformar el plan de estudios.
\item Ubicar contenidos por �rea y eje tem�tico.
\item Organizar contenidos en asignaturas.
\item Tomar en cuenta la pertinencia, coherencia, integraci�n vertical para incorporar las asignaturas al plan de estudio.
\item Verificar la consistencia de la propuesta con en modelo acad�mico propuesto.


\end{enumerate}
\vfill
{\large \today\par}
\end{document}