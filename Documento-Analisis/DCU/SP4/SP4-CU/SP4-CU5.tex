\begin{UseCase}{SP4-CU5}{Consultar Programas Académicos}{El usuario visualiza la información de los Programas Académicos regsitrados.}
        \UCitem{Versión}{\color{Gray}1.0}
        \UCitem{Autor}{\color{Gray}Plata García Josué Eliasaf}
        \UCitem{Supervisa}{\color{Gray}Evelyn Reyes}
        \UCitem{Actor}{\hyperlink{Usuario}{Jefe de Innovación Educativa,Jefe de División de Innovación Académica,Jefe de Departamento de Desarrollo e Innovación Curricular
    }}
        \UCitem{Propósito}{El Usuario Jefe de Innovación Educativa puede visualizar una lista  de  los Programas Académicos almacenados en el Sistema asociados a su Unidad Académica.}
        \UCitem{Entradas}{Las entradas para la consulta del Programa Académico:
          \begin{itemize}
            \item Ninguna
          \end{itemize}
        }
        \UCitem{Origen}{Ninguno}
        \UCitem{Salidas}{
        \begin{itemize}
                \item \MSGref{MSG7}{Los catálogos necesarios no están disponibles por el momento, favor de intentarlo más tarde.}
        \end{itemize}
        }
        \UCitem{Destino}{Pantalla.}
        \UCitem{Precondiciones}{\begin{itemize}
                \item El catálogo de tipo Unidad Académica debe de estar cargado en el sistema.
                   \end{itemize}}
        \UCitem{Postcondiciones}{Ninguna}
        \UCitem{Errores}{Ningún}
        \UCitem{Puntos de Exteción}{
        \begin{itemize}
            \item \UCref{SP4-CU13}: Consultar Planes de Estudio.
            \item \UCref{SP4-CU1}: Registrar Programa Académico.
            \item \UCref{SP4-CU8}: Editar Programa Académico.
        \end{itemize}
    }
        \UCitem{Estado}{Revisión.}
        \UCitem{Observaciones}{}
\end{UseCase}
%--------------------------- CU TRAYECTORIA PRINCIPAL -------------------------
\begin{UCtrayectoria}{Principal}
    \UCpaso[\UCactor] Presiona el botón \IUbutton{Gestionar Programa Académico} de la interfaz de usuario menú para Jefe de Innovación Educativa.
    \UCpaso Muestra la interfaz de usuario \IUref{GPA-J}{Gestionar Programa Académico}.
    \UCpaso Carga el catálogo de Unidad Académica definido en la \BRref{BR14}{Catálogos existentes}. [Trayectoria A] y [Trayectoria C] Trayectorias lanzadas por el boton Gestionar Programa Académico.
    \UCpaso[\UCactor] Selecciona la Unidad Académica que desea como filtro. [Trayectoria B] Trayectoria Lanzada por el botón filtro.
    \UCpaso[\UCactor] Aplica el filtro presionando el botón \IUbutton{Buscar}.
    \UCpaso Muestra tabla con los datos de los Programas Académicos que cumplan con el filtro..
\end{UCtrayectoria}
%------------------------ CU TRAYECTORIA ALTERNARIVA A -------------------------
\begin{UCtrayectoriaA}{A}{Los catálogos no se pudieron cargar.}
    \UCpaso Muestra el mensaje \MSGref{MSG7}{Los catálogos necesarios no están disponibles por el momento, favor de intentarlo más tarde}.
    \UCpaso[\UCactor] Cierra el mensaje presionando el botón \IUbutton{Aceptar}.
    \UCpaso Muestra la interfaz de Inicio (abi) .
\end{UCtrayectoriaA}
%------------------------ CU TRAYECTORIA ALTERNARIVA B -------------------------
\begin{UCtrayectoriaA}{B}{El actor no selecciona ningúna Unidad Académica}
    \UCpaso Muestra una tabla con los datos de todos los Recursos Humanos.
    \UCpaso Continúa en el paso 3 de la trayectoria principal del \UCref{SP4-CU7}.
\end{UCtrayectoriaA}
%------------------------ CU TRAYECTORIA ALTERNARIVA C-------------------------
\begin{UCtrayectoriaA}{B}{El actor es el Jefe de Innovación Educativa}
    \UCpaso Continúa en el paso 6 de la trayectoria principal del \UCref{SP4-CU5} con filtro predeterminado de su Unidad Académica.
\end{UCtrayectoriaA}