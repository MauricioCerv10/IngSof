\begin{UseCase}{SP4-CU11}{Editar Recurso Humano}{El usuario Jefe de Innovación Educativa modifica la información general del Recurso Humano.}
		\UCitem{Versión}{\color{Gray}1.0}
		\UCitem{Autor}{\color{Gray}Plata García Josué Eliasaf}
		\UCitem{Supervisa}{\color{Gray}}
		\UCitem{Actor}{\hyperlink{Usuario}{Jefe de Innovación Educativa}}
		\UCitem{Propósito}{Cambiar los datos de un Recurso Humano.}
		\UCitem{Entradas}{Las entradas para Editar Recurso Humano:
            \begin{itemize}
                \item Nombre.
                \item Primer Apellido.
                \item Segundo Apellido.
                \item Cargo.
                \item Título.
                \item Lugar de Trabajo
            \end{itemize}
        }
		\UCitem{Origen}{Teclado.}
		\UCitem{Salidas}{
            \begin{itemize}
                \item \MSGref{MSG7}{Los catálogos necesarios no se han cargado, favor de intentarlo más tarde.}
                \item \MSGref{MSG29}{¿Está seguro que desea cancelar? Se perderán todos los avances sin guardar.}
                \item \MSGref{MSG31}{Los cambios se guardaron exitosamente.}
                \item \MSGref{MSG32}{Todos los campos son obligatorios.}
                \item \MSGref{MSG33}{El Recurso Humano con la Matrícula de Empleado [Número de Matrícula de Empleado] ya existe}
			    \item \MSGref{MSG35}{Inconsistencia en los datos. Verifique los campos e intente de nuevo.}
                \item \MSGref{MSG37}{Ha ocurrido un error al guardar los datos.}
        	\end{itemize}
        }
		\UCitem{Destino}{Pantalla.}
		\UCitem{Precondiciones}{}
		\UCitem{Postcondiciones}{El Recurso Humano quedó actualizado en el sistema, permitiendo consultarlo y generar tareas de registro del mismo o quedó sin modificarse.}
		\UCitem{Errores}{Ninguno
		}
		\UCitem{Estado}{Revisión.}
		\UCitem{Observaciones}{Ninguna}
\end{UseCase}
%--------------------------- CU TRAYECTORIA PRINCIPAL -------------------------
\begin{UCtrayectoria}{Principal}
    \UCpaso[\UCactor] Presiona el botón \IUbutton{Editar} de la interfaz de usuario \IUref{GRH-J}{Gestionar Recurso Humano} el cual tiene forma de lápiz.

    \UCpaso Muestra la interfaz de usuario \IUref{ERH-J}{Editar Recurso Humano}.
    \UCpaso Cargar datos del Recurso Humano correspondiente.
    \UCpaso Carga el catálogo de Cargo descrito en la \BRref{BR14}{Catálogos existentes}. [Trayectoria A]
    \UCpaso Carga el catálogo de Título descrito en la \BRref{BR14}{Catálogos existentes}. [Trayectoria A]
    \UCpaso Carga el catálogo de Lugar de Trabajo descrito en la \BRref{BR14}{Catálogos existentes}. [Trayectoria A]

    \UCpaso[\UCactor] Modifica los campos deseados.
    \UCpaso[\UCactor] Termina la operación presionando el botón \IUbutton{Guardar}.  Existen las trayectorias alternativas [Trayectoria B] y [Trayectoria B.1] al presionar el boton Cancelar.

    \UCpaso Verifica que todos los campos marcados como obligatorios hayan sido contestados conforme a la \BRref{BR13}{Todos los datos solicitados son obligatorios},  \BRref{BR30}{Información de los Usuarios y Recursos Humanos} y la \BRref{BR16}{Matrícula de empleado única} [Trayectoria C][Trayectoria D][Trayectoria F].

    \UCpaso Persiste los datos ingresados. [Trayectoria E]

    \UCpaso El sistema muestra el mensaje \MSGref{MSG31}{ Los cambios se guardaron exitosamente.}.
    \UCpaso[\UCactor] Cierra el mensaje presionando el botón \IUbutton{Aceptar}.
    \UCpaso Muestra la interfaz de usuario \IUref{GRH-J}{Gestionar Recurso Humano}
\end{UCtrayectoria}
    %------------------------ CU TRAYECTORIA ALTERNARIVA A -------------------------
\begin{UCtrayectoriaA}{A}{No cargan los catálogos de Recurso Humano.}
    \UCpaso Muestra el mensaje \MSGref{MSG7}{Los catálogos necesarios no se han cargado, favor de intentarlo más tarde.}.
    \UCpaso[\UCactor] Cierra el mensaje presionando el botón \IUbutton{Aceptar}.
\UCpaso Muestra la interfaz de usuario Menú para Jefe de Innovación Educativa(Pantalla de Abi).
\end{UCtrayectoriaA}
%------------------------ CU TRAYECTORIA ALTERNARIVA B -------------------------
\begin{UCtrayectoriaA}{B}{El actor presionael botón \IUbutton{Cancelar}.}
    \UCpaso Muestra el mensaje \MSGref{MSG29}{¿Está seguro que desea cancelar? Se perderán todos los avances sin guardar.}.
    \UCpaso[\UCactor] Cierra el mensaje presionando el botón \IUbutton{Si}.
    \UCpaso Muestra la interfaz de usuario \IUref{GRH-J}{Gestionar Recurso Humano}
\end{UCtrayectoriaA}
%------------------------ CU TRAYECTORIA ALTERNARIVA B -------------------------
\begin{UCtrayectoriaA}{B.1}{El actor presiona accidentalmente el botón \IUbutton{Cancelar}.}
    \UCpaso Muestra el mensaje \MSGref{MSG29}{¿Está seguro que desea cancelar? Se perderán todos los avances sin guardar.}.
    \UCpaso[\UCactor] Cierra el mensaje presionando el botón \IUbutton{No}.
    \UCpaso Continúa en el paso 4 de la trayectoria principal del \UCref{SP4-CU11}.
\end{UCtrayectoriaA}
%------------------------ CU TRAYECTORIA ALTERNARIVA C -------------------------
\begin{UCtrayectoriaA}{C}{El sistema detecta uno o más campos sin contestar.}
    \UCpaso Muestra el mensaje \MSGref{MSG32}{Todos los campos son obligatorios}.
    \UCpaso[\UCactor] Cierra el mensaje presionando el botón \IUbutton{Aceptar}.
    \UCpaso Continúa en el paso 4 de la trayectoria principal del \UCref{SP4-CU11}.
\end{UCtrayectoriaA}
%------------------------ CU TRAYECTORIA ALTERNARIVA D -------------------------
\begin{UCtrayectoriaA}{D}{El Sistema detecta caracteres no alfanuméricos.}
    \UCpaso Muestra el mensaje \MSGref{MSG35}{ Inconsistencia en los datos. Verifique los campos e intente de nuevo.}.
    \UCpaso Continúa en el paso 4 de la trayectoria principal del \UCref{SP4-CU11}.
\end{UCtrayectoriaA}
%------------------------ CU TRAYECTORIA ALTERNARIVA E -------------------------
\begin{UCtrayectoriaA}{E}{Ocurre un error al momento de persistir los datos.}
    \UCpaso Muestra el mensaje \MSGref{MSG37}{Ha ocurrido un error al guardar los datos.}
    \UCpaso[\UCactor] Cierra el mensaje presionando el botón \IUbutton{Aceptar}.
    \UCpaso Muestra la interfaz de usuario \IUref{GRH-J}{Gestionar Recursos Humanos}.
\end{UCtrayectoriaA}
%------------------------ CU TRAYECTORIA ALTERNARIVA F -------------------------
\begin{UCtrayectoriaA}{F}{La Matrícula de Empleado ingresada ya está registrada en el sistema incumpliendo la \BRref{BR16}{Matrícula de empleado única}.}
    \UCpaso Muestra el mensaje \MSGref{MSG33}{El Recurso Humano con la Matrícula de Empleado [Número de Matrícula de Empleado] ya existe}.
    \UCpaso[\UCactor] Cierra el mensaje presionando el botón \IUbutton{Aceptar}.
    \UCpaso Continúa en el paso 4 de la trayectoria principal del \UCref{SP4-CU11}.
\end{UCtrayectoriaA}