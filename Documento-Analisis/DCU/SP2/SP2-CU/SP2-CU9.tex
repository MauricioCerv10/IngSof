\chapter{SP2-CU9 Autorizar Sección Evaluación y Acreditación de Unidad de Aprendizaje}
\begin{UseCase}{SP2-CU9}{ Autorizar Sección Evaluación y Acreditación de Unidad de Aprendizaje}{El analista podrá visualizar la información de la sección de Evaluación y Acreditación de la unidad de aprendizaje para poder aprobarla.}
		\UCitem{Versión}{\color{Gray}1.0}
		\UCitem{Autor}{\color{Gray}Domínguez López Humberto}
		\UCitem{Supervisa}{\color{Gray}Parra Garcilazo Cinthya Dolores}
		\UCitem{Actor}{Analista}
		\UCitem{Propósito}{Que el analista conozca la sección de Evaluación y Acreditación de la propuesta de Unidad de Aprendizaje para determinar si necesita correcciones.}
		\UCitem{Entradas}{Clic en botones:
          \begin{itemize}
          	\item Finalizar Revisión.
          	\item Guardar.
            \item Cancelar.
            \item Nuevo Comentario.
            \item Subrayar.
            \item Eliminar subrayado.
            \item Editar Comentario.
            \item Eliminar Comentario.
          \end{itemize}
        }
		\UCitem{Origen}{Mouse.}
		\UCitem{Salidas}{
        	\begin{itemize}
        		\hypertarget{MSG15}{\item MSG15. ¿Está seguro que desea Cancelar la revisión?
				Se perderán las Anotaciones que no hayas guardado anteriormente.}
                \hypertarget{MSG16}{\item MSG16. ¿Está seguro de Finalizar la Revisión?}
                \hypertarget{MSG17}{\item MSG17 Sección Aprobada.}
                
               \hypertarget{MSG18}{ \item MSG18 Revisión de Sección Finalizada.}
                \hypertarget{MSG19}{\item MSG19 Anotaciones En Sección Guardadas Correctamente.}

        	\end{itemize}
        }
		\UCitem{Destino}{Pantalla.}
		\UCitem{Precondiciones}{ Se llamó el caso de uso SP2-CU1}
		\UCitem{Postcondiciones}{Se habilita la llamada a los casos de uso SP2-CU14, SP2-CU15, SP2-CU16, SP2-CU17.}
		\UCitem{Errores}{}
		\UCitem{Estado}{Revisión.}
		\UCitem{Observaciones}{}
\end{UseCase}

%--------------------------- CU TRAYECTORIA PRINCIPAL -------------------------
\begin{UCtrayectoria}{Principal}

    \UCpaso[\UCactor] presiona el botón  \IUbutton{Evaluación y Acreditación} de la interfaz de usuario \IUref{SP2-IU-INICIO}{Sección Inicio}

    \UCpaso El sistema obtiene la información correspondiente a la sección de Evaluación y Acreditación.
    
    \UCpaso El sistema obtiene la bitácora de comentarios correspondientes a la sección Evaluación y Acreditación de la Unidad de Aprendizaje. 
    
    \UCpaso El sistema verifica que la sección Evaluación y Acreditación de la Unidad de Aprendizaje no haya sido aprobada anteriormente. Regla de Negocio BR9. \hyperlink{SP2-CU9-A1}{Trayectoria A1}. 
    
    \UCpaso El sistema muestra la interfaz de usuario  \IUref{SP2-IU-EVALUACION}{Sección Evaluación y Acreditación}.
    
    \UCpaso[\UCactor] presiona el botón \IUbutton{Finalizar Revisión}. \hyperlink{SP2-CU9-A2}{Trayectoria A2}.
    \UCpaso El sistema muestra el \MSGref{MSG16}{¿Está seguro de Finalizar la Revisión?}.
    
    \UCpaso	El sistema verifica que no existan nuevos comentarios o subrayados para la sección de la Unidad de Aprendizaje. Regla de Negocio BR8. \hyperlink{SP2-CU9-A3}{Trayectoria A3}. 
    
    \UCpaso El sistema pone el estado de la sección Evaluación y Acreditación en “Aprobado”.
    
    \UCpaso El sistema muestra el mensaje \MSGref{MSG17}{Sección Aprobada}.

    \UCpaso El sistema muestra la interfaz de usuario \IUref{SP2-IU-INICIO}{Sección Inicio}

\end{UCtrayectoria}

%------------------------ CU TRAYECTORIA ALTERNARIVA A1 -------------------------

\begin{UCtrayectoriaA}{A1}{La sección de la Unidad de Aprendizaje ya ha sido aprobada anteriormente.}

	\hypertarget{SP2-CU9-A1}{\UCpaso El sistema muestra la interfaz de usuario \IUref{SP2-IU-EVALUACION}{Sección Evaluación y Acreditación}.}
    \UCpaso El sistema deshabilita los botones superiores: \IUbutton{Nuevo Comentario}, \IUbutton{Subrayar}, \IUbutton{Eliminar Subrayado}.
    \UCpaso El sistema deshabilita los botones inferiores: \IUbutton{Cancelar}, \IUbutton{Guardar}, \IUbutton{Finalizar Revisión}.
    \UCpaso El sistema deshabilita los botones laterales de: \IUbutton{Modificar Comentario}, \IUbutton{Eliminar Comentario}.
\end{UCtrayectoriaA}

%------------------------ CU TRAYECTORIA ALTERNARIVA A2 -------------------------
	
\begin{UCtrayectoriaA}{A2}{El analista no desea finalizar aun la revisión de la sección de la Unidad de Aprendizaje.}

    \hypertarget{SP2-CU9-A2}{\UCpaso[\UCactor] presiona el botón \IUbutton{Guardar} \hyperlink{SP2-CU9-A2.1}{Trayectoria A2.1}}. 
    \UCpaso El sistema guarda los nuevos comentaros y subrayados hechos durante esa sesión.
    \UCpaso El sistema muestra el mensaje \MSGref{MSG19}{Anotaciones En Sección Guardadas Correctamente}.
    \UCpaso El sistema muestra la interfaz de usuario \IUref{SP2-IU-INICIO}{Sección Inicio}
\end{UCtrayectoriaA}

%------------------------ CU TRAYECTORIA ALTERNARIVA A2.1 -----------------------
\begin{UCtrayectoriaA}{A2.1}{El analista desea cancelar todo lo que haya hecho en la sección de la Unidad de Aprendizaje durante esa sesión.}

	\hypertarget{SP2-CU9-A2.1}{\UCpaso[\UCactor] presiona el botón \IUbutton{Cancelar}}. 
    \UCpaso El sistema muestra el mensaje \MSGref{MSG15}{¿Está seguro que desea Cancelar la revisión? Se perderán las Anotaciones que no hayas guardado anteriormente.}
Se perderán las Anotaciones que no hayas guardado anteriormente. .
    \UCpaso El sistema elimina los nuevos comentaros y subrayados hechos durante esa sesión.
    \UCpaso El sistema muestra la interfaz de usuario \IUref{SP2-IU-INICIO}{Sección Inicio}.
\end{UCtrayectoriaA}

%------------------------ CU TRAYECTORIA ALTERNARIVA A3 -----------------------
	
\begin{UCtrayectoriaA}{A3}{El analista realizo comentarios y subrayados para su posterior corrección en la sección de la Unidad de Aprendizaje.} 

	\hypertarget{SP2-CU9-A3}{\UCpaso El sistema pone el estado de la sección Evaluación y Acreditación de Unidad de Aprendizaje en “Revisado”.}
    \UCpaso El sistema muestra el mensaje \MSGref{MSG18}{Revisión de Sección Finalizada}.
    \UCpaso El sistema muestra la interfaz de usuario \IUref{SP2-IU-INICIO}{Sección Inicio}.
\end{UCtrayectoriaA}

\chapter{Pantallas}
\IUfig[0.75]{SP2-Pantallas/Inicio}{SP2-IU-INICIO}{Sección Inicio}
\IUfig[0.75]{SP2-Pantallas/Evaluacion_Acreditacion}{SP2-IU-EVALUACION}{Sección Evaluación y Acreditación}

