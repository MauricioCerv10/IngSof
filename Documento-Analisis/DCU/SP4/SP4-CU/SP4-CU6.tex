% Consultar Mapa Curricular.
\begin{UseCase}{SP4-CU6}{Consultar Mapa Curricular}{El usuario Docente visualiza la totalidad de los contenidos del Mapa Curricular registrados.}
		\UCitem{Versión}{\color{Gray}1.0}
		\UCitem{Autor}{\color{Gray}Cervantes Moreno Christian Andres}
		\UCitem{Supervisa}{\color{Gray} Evelyn Reyes}
		\UCitem{Actor}{\hyperlink{Usuario}{Docente}}
		\UCitem{Propósito}{Visualizar el mapa curricular del Plan de Estudios registrado en el sistema.}
		\UCitem{Entradas}{Ninguna}
		\UCitem{Origen}{Teclado, Mouse}
		\UCitem{Salidas}{
        	\begin{itemize}
        		\item \MSGref{MSG36}{¿Seguro que desea eliminar el registro?}.
                \item \MSGref{MSG7}{Los catálogos necesarios no están disponibles por el momento, favor de intentarlo más tarde}.
        	\end{itemize}
        }
		\UCitem{Destino}{Pantalla.}
		\UCitem{Precondiciones}{ Debe de existir al menos un Plan de Estudios con Unidades de Aprendizaje registrado en el sistema.}
		\UCitem{Postcondiciones}{El Plan de Estudios y las Unidades de Aprendizaje son visualizadas en el sistema.}
		\UCitem{Errores}{
			  \begin{itemize}
				\item E1. El sistema no permite editar Plan de Estudios con las Unidades de Aprendizaje registradas.
				\item E2. El sistema no permite eliminar Plan de Estudios con las Unidades de Aprendizaje registradas.
			\end{itemize}
		}
	 \UCitem{Puntos de Exteción}{
		\begin{itemize}
			\item \UCref{SP4-CU9}: Editar Plan de Estudios.
			\item \UCref{SP4-CU10}: Editar Unidad de Aprendizaje.
		\end{itemize}
	}
		\UCitem{Estado}{Revisión.}
		\UCitem{Observaciones}{}

\end{UseCase}

%--------------------------- CU TRAYECTORIA PRINCIPAL -------------------------
\begin{UCtrayectoria}{Principal}

    %Usuario
    \UCpaso[\UCactor] Presiona el botón Consultar de la Interfaz de usuario  \IUref{IU2-D}{Consultar Tareas}.

	%Sistema
    \UCpaso El sistema carga el catálogo del Plan de Estudios. [Trayectoria A]


    \UCpaso Muestra la interfaz de usuario \IUref{IU2.1-D}{Consultar Mapa Curricular}.
    \UCpaso[\UCactor] Selecciona el semestre de la Unidades de Aprendizaje que desea consultar.
    \UCpaso Muestra las Unidades de Aprendizaje registradas en el sistema.[Trayectoria B][Trayectoria B.1]



\end{UCtrayectoria}

%------------------------ CU TRAYECTORIA ALTERNARIVA X -------------------------

\begin{comment}
\begin{UCtrayectoriaA}{A}{El sistema no encuentra ningún formulario para mostrar.}
	\UCpaso No encuentra ningún formulario para mostrar.
    \UCpaso El sistema muestra el mensaje \MSGref{MSG6}{Por el momento no se puede registrar la bibliografía}.
    \UCpaso[\UCactor] Cierra el mensaje presionando el botón \IUbutton{Aceptar}.
    \UCpaso Continua en el paso 1 de la trayectoria principal del \UCref{CU1}.
\end{UCtrayectoriaA}
\end{comment}

%------------------------ CU TRAYECTORIA ALTERNARIVA A -------------------------

\begin{UCtrayectoriaA}{A}{El sistema no carga el catálogo de la Unidad de Aprendizaje}
	%\UCpaso[\UCactor] Presiona el botón \IUbutton{$\bigoplus$} que se encuentra a un lado del campo ``Nombre'' del formulario \IUref{IU2}{Registro de bibliografía}  [Trayectoria A.1]
	\UCpaso[\UCactor] Presiona el botón \IUbutton{rconsultar}
	\UCpaso Muestra el mensaje \MSGref{MSG7}{Los catálogos necesarios no están disponibles por el momento, favor de intentarlo más tarde}.
	\UCpaso[\UCactor] Confirma la operación presionando el botón \IUbutton{Ok}.
	 \UCpaso Muestra la interfaz de usuario \IUref{IU2-D}{Consultar Tareas}

\end{UCtrayectoriaA}




%------------------------ CU TRAYECTORIA ALTERNARIVA B -------------------------

\begin{UCtrayectoriaA}{B}{El usuario desea eliminar la Unidad de Aprendizaje.}
	\UCpaso  Muestra el mensaje \MSGref{MSG36}{¿Seguro que desea eliminar el registro?}.
	\UCpaso[\UCactor] Presiona el botón \IUbutton{Aceptar}.
	\UCpaso Elimina la Unidad de Aprendizaje.
\end{UCtrayectoriaA}


%------------------------ CU TRAYECTORIA ALTERNARIVA B.1 -------------------------

\begin{UCtrayectoriaA}{B.1}{El actor presiona accidentalmente el botón Eliminar}
	\UCpaso Muestra el mensaje \MSGref{MSG36}{¿Seguro que desea eliminar el registro?}.
	\UCpaso[\UCactor] Presiona el botón \IUbutton{No}.
	\UCpaso Cierra el mensaje.

\end{UCtrayectoriaA}

