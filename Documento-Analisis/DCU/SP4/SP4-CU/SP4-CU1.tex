% REGISTRAR BIBLIOGRAFÍA.
\begin{UseCase}{SP4-CU1}{Registrar Programa Académico}{El usuario Jefe de Innovación Educativa ingresa los datos de un Programa Académico.}
		\UCitem{Versión}{\color{Gray}1.0}
		\UCitem{Autor}{\color{Gray}Plata García Josué Eliasaf}
		\UCitem{Supervisa}{\color{Gray}}
		\UCitem{Actor}{\hyperlink{Usuario}{Jefe de Innovación Educativa}}
		\UCitem{Propósito}{Registrar el nombre del Programa Académico en el sistema.}
		\UCitem{Entradas}{Las entradas para el registro de Programa Académico:
          \begin{itemize}
          	\item Nombre (Tipo Alfanumérico)
          \end{itemize}
        }
		\UCitem{Origen}{Teclado.}
		\UCitem{Salidas}{
        	\begin{itemize}
        		\item \MSGref{MSG32}{Todos los campos son obligatorios}
                \item \MSGref{MSG5}{Registro finalizado exitosamente.}
                \item \MSGref{MSG29}{¿Está seguro que desea cancelar? Se perderán todos los avances sin guardar.}
                \item \MSGref{MSG35}{Inconsistencia en los datos. Verique los campos e intente de nuevo.}
        	\end{itemize}
        }
		\UCitem{Destino}{Pantalla.}
		\UCitem{Precondiciones}{}
		\UCitem{Postcondiciones}{El Programa Académico quedó registrado en el sistema, permitiendo consultarlo y generar tareas de registro del mismo.}
		\UCitem{Errores}{E1. El nombre del Programa Académico contiene uno o más caracteres no alfanuméricos.}
		\UCitem{Estado}{Revisión.}
		\UCitem{Observaciones}{El nombre de un Programa Académico puede ser igual a otro.}
\end{UseCase}
%--------------------------- CU TRAYECTORIA PRINCIPAL -------------------------
\begin{UCtrayectoria}{Principal}
    \UCpaso[\UCactor] Presiona el botón \IUbutton{Registrar Programa Académico} de la interfaz de usuario \IUref{IU2.2-J}{Consultar Programas Académicos}
    \UCpaso Muestra la interfaz de usuario \IUref{IU2.2.1-J}{Registrar Programa Académico}.
    \UCpaso[\UCactor] Ingresa el nombre del Programa Académico.
    \UCpaso[\UCactor] Termina la operación presionando el botón \IUbutton{Guardar}. [Trayectoria A] [Trayectoria B]
    \UCpaso Verifica que todos los campos marcados como obligatorios hayan sido contestados. [Trayectoria C][Trayectoria D]
    \UCpaso Guarda la información del Programa Académico correspondiente.
    \UCpaso El sistema muestra el mensaje \MSGref{MSG5}{Registro finalizado exitosamente}.
    \UCpaso[\UCactor] Cierra el mensaje presionando el botón \IUbutton{Aceptar}.
    \UCpaso Muestra la interfaz de usuario \IUref{IU2.2-J}{Consultar Programas Académicos}.
\end{UCtrayectoria}
%------------------------ CU TRAYECTORIA ALTERNARIVA A -------------------------
\begin{UCtrayectoriaA}{A}{El actor quiere cancelar el registro de la bibliografía.}
	\UCpaso[\UCactor] Presiona el botón \IUbutton{Cancelar}.
    \UCpaso Muestra el mensaje \MSGref{MSG3}{¿Seguro que desea cancelar el registro?}.
    \UCpaso[\UCactor] Confirma la operación presionando el botón \IUbutton{Si}.
    \UCpaso Muestra la interfaz de usuario \IUref{IU2.2-J}{Consultar Programas Académicos}
\end{UCtrayectoriaA}
%------------------------ CU TRAYECTORIA ALTERNARIVA B -------------------------
\begin{UCtrayectoriaA}{B}{El actor presiona accidentalmente el botón Cancelar}
	\UCpaso[\UCactor] Presiona el botón \IUbutton{Cancelar}
    \UCpaso Muestra el mensaje \MSGref{MSG29}{¿Está seguro que desea cancelar? Se perderán todos los avances sin guardar.}.
    \UCpaso[\UCactor] Presiona el botón \IUbutton{No}.
    \UCpaso Cierra el mensaje.
    \UCpaso Continúa en el paso 4 de la trayectoria principal del \UCref{SP4-CU1}.
\end{UCtrayectoriaA}
%------------------------ CU TRAYECTORIA ALTERNARIVA C -------------------------
\begin{UCtrayectoriaA}{C}{Uno o más campos obligatorios no fueron contestados.}
	\UCpaso Detecta uno o más campos sin contestar.
    \UCpaso Muestra el mensaje \MSGref{MSG32}{Todos los campos son obligatorios}.
    \UCpaso[\UCactor] Cierra el mensaje presionando el botón \IUbutton{Aceptar}.
    \UCpaso Continúa en el paso 3 de la trayectoria principal del \UCref{SP4-CU1}.
\end{UCtrayectoriaA}
%------------------------ CU TRAYECTORIA ALTERNARIVA D -------------------------
\begin{UCtrayectoriaA}{D}{El Nombre de Programa Académico contiene caracteres no alfanuméricos}
	\UCpaso El Sistema detecta caracteres no alfanuméricos E1.
    \UCpaso Muestra el mensaje \MSGref{MSG35}{Inconsistencia en los datos. Verique los campos e intente de nuevo.}.
    \UCpaso Continúa en el paso 1 de la trayectoria principal del \UCref{SP4-CU1}.
\end{UCtrayectoriaA}