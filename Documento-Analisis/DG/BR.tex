\chapter{Especificación de Reglas del Negocio}
\section{Reglas de Operación}

\begin{BussinesRule}{BR}{Jefes de Departamento e Innovación curricular }
     \BRitem[Tipo: Estructural]
    \BRitem[Clase: Habilitadora]
    \BRitem[Nivel: Control]
    \BRitem[Descripción: Solo puede existir un único Jefe de Departamento e Innovación curricular.]
    \BRitem[Motivación: Restringir el numero de Usuarios capaces de Gestionar Usuarios]
\end{BussinesRule}

\begin{BussinesRule}{BR}{Jefes de Innovación Educativa }
    \BRitem[Tipo: Estructura]
    \BRitem[Clase: Habilitadora]
    \BRitem[Nivel: Control]
    \BRitem[Descripción: Solo puede existir un único Jefe de Innovación Educativa por Unidad Académica.]
    \BRitem[Motivación: Tener un único responsable por Unidad Académica.]
\end{BussinesRule}

\begin{BussinesRule}{BR}{Máquina de Estados de una Tarea.}
    \BRitem[Tipo: Flujo]
    \BRitem[Clase: Habilitadora]
    \BRitem[Nivel: Control]
    \BRitem[Descripción: Según el estado de la Tarea son los permisos de quien puede modificar la información relacionada con esta.]
    \BRitem[Motivación: Tener un mayor control sobre la información]
    \BRitem[Ejemplo Positivo:La Tarea de registrar Mapa Curricular se aprobó después de que el analista y el jefe la revisaran.]
    \BRitem[Ejemplo Negativo: La Tarea de registrar Unidad de Aprendizaje se empezó a revisar antes de que terminara su registro.]
\end{BussinesRule}

\begin{BussinesRule}{BR}{Tareas de Registro.}
    \BRitem[Tipo: Flujo]
    \BRitem[Clase: Habilitadora]
    \BRitem[Nivel: Control]
    \BRitem[Descripción: Solo los Docentes pueden registrar Mapas Curriculares y Unidades de Aprendizaje.]
    \BRitem[Motivación: Cuestiones de jerarquía dentro de la Unidad Académica]
    \BRitem[Ejemplo Positivo:El usuario Docente registra un Mapa Curricular o una Unidad de Aprendizaje.]
    \BRitem[Ejemplo Negativo:El usuario Analista registra un Mapa Curricular o una Unidad de Aprendizaje]
\end{BussinesRule}

\begin{BussinesRule}{BR}{Revisión de Tareas.}
    \BRitem[Tipo: Flujo]
    \BRitem[Clase: Habilitadora]
    \BRitem[Nivel: Control]
    \BRitem[Descripción: Solo los analistas y los jefes  pueden revisar Mapas Curriculares o Unidades de Aprendizaje]
    \BRitem[Motivación: Cuestiones de jerarquía dentro de la Unidad Académica.]
    \BRitem[Ejemplo Positivo:El Analista revisa un Mapa Curricular y sus respectivas Unidades de Aprendizaje.]
    \BRitem[Ejemplo Negativo: El Docente revisa un Mapa Curricular y Unidades de Aprendizaje.]
\end{BussinesRule}

\begin{BussinesRule}{BR}{Aprobación de Tareas.}
    \BRitem[Tipo: Flujo]
    \BRitem[Clase: Habilitadora]
    \BRitem[Nivel: Control]
    \BRitem[Descripción: Una Tarea es aprobada si al terminar de revisar no contiene comentarios]
    \BRitem[Motivación: Poder saber cuando una tarea a finalizado.]
    \BRitem[Ejemplo Positivo:El Jefe de Innovación Educativa no agrego comentarios por lo que se aprueba.]
    \BRitem[Ejemplo Negativo: El Jefe de Innovación Educativa agrega comentario y aun así se aprueba el documento.]
\end{BussinesRule}

\begin{BussinesRule}{BR}{Rechazo de Tareas.}
    \BRitem[Tipo: Flujo]
    \BRitem[Clase: Habilitadora]
    \BRitem[Nivel: Control]
    \BRitem[Descripción: Una Tarea es rechazada si al terminar de revisar contiene comentarios]
    \BRitem[Motivación: Poder saber caundo uan tarea debe volver al estado de Registor.]
    \BRitem[Ejemplo Positivo:El Jefe de Innovación Educativa agrego comentarios por lo que se rechaza.]
    \BRitem[Ejemplo Negativo: Ni el Jefe de Innovación Educativa ni el Analista agregaron comentarios y aun así se apruebó.]
\end{BussinesRule}

\begin{BussinesRule}{BR}{Analistas revisando una Tarea}
    \BRitem[Tipo: Relación]
    \BRitem[Clase:Habilitadora]
    \BRitem[Nivel: Control]
    \BRitem[Descripción: Una Tarea debe tener solo un analista asignado para su revisión, y un analista puede tener muchos Tareas asignados para su revisión.]
    \BRitem[Motivación: Distribuir los procesos de revisión en la DES de forma óptima y eficiente, para evitar carga de trabajo.]
    \BRitem[Ejemplo Positivo: Mapa curricular Plan 2018: analista 1 asignado, Unidad de Aprendizaje "Ingenieria de Software" Plan 2018: analista 1 asignado.]
    \BRitem[Ejemplo Negativo: Mapa curricular Plan 2018: 3 analista asignados]
\end{BussinesRule}

\begin{BussinesRule}{BR}{Planes de Estudios en rediseño por Programa Académico}
    \BRitem[Tipo:] Relación.
    \BRitem[Clase:] Habilitadora.
    \BRitem[Nivel:] Control.
    \BRitem[Descripción:] Un Programa Académico puede tener unicamente un Plan de Estudios en rediseño.
    \BRitem[Motivación: Una Prograama Académico puede tener varios Planes de Estudios activos pero solo se rediseña el mas actual pues este fue el rediseño de los anterior] .
    \BRitem[Ejemplo Positivo:] Ingeniería en Sistemas Computacionales: Plan 2019.
    \BRitem[Ejemplo Negativo:] Licenciatura en Economía: Plan 2000, Plan 2009, Plan 2014, Plan 2018.
\end{BussinesRule}

\begin{BussinesRule}{BR}{Existen los catálogos }
    \BRitem[Tipo: Regla de inferencia de un hecho]
    \BRitem[Clase: Habilitadora]
    \BRitem[Nivel: Control]
    \BRitem[Descripción: Existen catálogos de :
Modalidades
    Escolarizada
    Mixta
    No Escolarizada
Tipo de Usuario
    Docente
    Analista
Tareas
    Registrar Mapa Curricular
    Registrar Unidad de Aprendizaje]
    \BRitem[Motivación: Mantener los datos en un estado consistente.]
\end{BussinesRule}

\begin{BussinesRule}{BR}{La matrícula del empleado es única.}
    \BRitem[Tipo: Relación]
    \BRitem[Clase: Habilitadora]
    \BRitem[Nivel: Control]
    \BRitem[Descripción:Cada una de las matrículas de los empleados registrados en el sistema cuentan con una matrícula única.]
    \BRitem[Motivación: Poder identificar a los usuarios.]
    \BRitem[Ejemplo Positivo:El Empleado Juan Perez tiene una matricula 2014000990  y El Empleado Armando López Doriga  tiene una matricula 2014000991 .]
    \BRitem[Ejemplo Negativo: El Empleado Juan Perez tiene una matricula 2014000990 y El Empleado Armando López Doriga  tiene una matricula 2014000990 .]
\end{BussinesRule}

\begin{BussinesRule}{BR} {El nombre de la Unidad de Aprendizaje es único.}
    \BRitem[Tipo: Relación]
    \BRitem[Clase: Habilitadora]
    \BRitem[Nivel: Control]
    \BRitem[Descripción: Cada  una de las Unidades de Aprendizaje dentro de un mismo Programa Académico tienen un nombre único]
    \BRitem[Motivación:  Que no exista redundancia en la información y la base de datos este siempre en un estado consistente.]
    \BRitem[Ejemplo Positivo:En un Programa Académico  tenemos la materia Ingeniería de Software y Administración de Proyectos.]
    \BRitem[Ejemplo Negativo: En un Programa Académico  tenemos la materia Ingeniería de Software y  Ingeniería de Software.]
\end{BussinesRule}

\begin{BussinesRule}{BR} {El nombre del Programa Académico es único.}
    \BRitem[Tipo: Relación]
    \BRitem[Clase: Habilitadora]
    \BRitem[Nivel: Control]
    \BRitem[Descripción: El nombre de los Programas Académicos es único dentro de la Unidad Académica]
    \BRitem[Motivación:  Que no exista redundancia en la información y la base de datos este siempre en un estado consistente.]
    \BRitem[Ejemplo Positivo:En una Unidad Académica existen los Programas Académicos Ingenieria en Sistemas Computacionales y Ingenieria en Comunicaciones y Electrónica.]
    \BRitem[Ejemplo Negativo: En una Unidad Académica existen los Programas Académicos Ingenieria en Sistemas Computacionales y Ingenieria en Sistemas Computacionales.]
\end{BussinesRule}
