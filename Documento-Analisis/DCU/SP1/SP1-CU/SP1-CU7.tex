\begin{UseCase}{SP1-CU7}{Registrar Temas de la Unidad de Aprendizaje}{El usuario podrá registrar temas a la Unidad de Aprendizaje.}
    \UCitem{Versión}{\color{Gray}1.0}
    \UCitem{Autor}{\color{Gray}López Rivera Aiko Dallane}
    \UCitem{Supervisa}{\color{Gray}Ramírez Martínez Janet Naibi}
    \UCitem{Actor}{\hyperlink{Usuario}{Usuario}}
    \UCitem{Propósito}{Servir como marco de referencia para el registro de Temas de la Unidad de Aprendizaje.}
    \UCitem{Entradas}{Las entradas para el registro del Perfil Docente serán:
          \begin{itemize}
            \item Número de tema. 
            \item Nombre del Tema.
            \item Horas con docente teoricas.
            \item Horas con docente prácticas.
            \item Horas de Aprendizaje Autonomo.
            \item Subtemas.
          \end{itemize}
        }
    \UCitem{Origen}{Teclado y Mouse.}
    \UCitem{Salidas}{
    }
    \UCitem{Destino}{Pantalla.}
    \UCitem{Precondiciones}{
    \item Se debe haber elaborado previamente el Programa Sintético.}
    \UCitem{Postcondiciones}{Los temas de la Unidad de Aprendizaje quedan registrados.}
    \UCitem{Errores}{}
    \UCitem{Estado}{Revisión.}
    \UCitem{Observaciones}{}
\end{UseCase}

%--------------------------- CU TRAYECTORIA PRINCIPAL -------------------------
\begin{UCtrayectoria}{Principal}

    \UCpaso[\UCactor] Presiona el botón \IUbutton{Registrar Temas de la Unidad de Aprendizaje} de la interfaz de usuario. \IUref{IU.03}{Página Principal}
    \UCpaso El sistema verifica que el programa sintectico haya sido registrado. [Trayectoria A]
    \UCpaso Muestra la interfaz de usuario \IUref{MIU3.01}{Registro de Temas de la Unidad de Aprendizaje}.
    \UCpaso[\UCactor] Ingresa el no. de tema.
    \UCpaso[\UCactor] Ingresa el nombre del tema.
    \UCpaso[\UCactor] Selecciona las horas con docente teoricas.
    \UCpaso[\UCactor] Selecciona las horas con docente prácticas.
    \UCpaso[\UCactor] Selecciona las horas con docente autónomas.
    \UCpaso[\UCactor] Presiona el botón \IUbutton{Agregar Temas} [Trayectoria B] 
    \UCpaso[\UCactor] Presiona el botón \IUbutton{Agregar Subtemas} [Trayectoria C]
    \UCpaso Abre un modal en el \UCref{SMIU3.01} [Trayectoria D]
    \UCpaso[\UCactor] Termina la operación presionando el botón \IUbutton{Guardar}. 
    \UCpaso Verifica que todos los campos marcados como obligatorios hayan sido llenados.[Trayectoria E]
    \UCpaso Guarda la información de los temas correspondientes a la Unidad de Aprendizaje.
    \UCpaso El sistema muestra el mensaje \MSGref{MSG5}{Registro finalizado exitosamente}.
    \UCpaso[\UCactor] Cierra el mensaje presionando el botón \IUbutton{Ok}.
    \UCpaso Muestra la interfaz de usuario \IUref{IU.03}{Página principal}.
\end{UCtrayectoria}

%------------------------ CU TRAYECTORIA ALTERNATIVA A -------------------------

\begin{UCtrayectoriaA}{A}{No se ha realizado la elaboración del Programa Sintético}.
	\UCpaso No muestra los datos provenientes del \UCref{SP1-CU1}.
	\UCpaso Muestra el mensaje \MSGref{MSG41}{Debe llenar el Programa Sintético para realizar este registro}.
	\UCpaso[\UCactor] Cierra el mensaje presionando el botón \IUbutton{Aceptar}.
	\UCpaso Muestra la interfaz de usuario \IUref{SP1-IU}{Principal}.
\end{UCtrayectoriaA}


%------------------------ CU TRAYECTORIA ALTERNATIVA B -------------------------

\begin{UCtrayectoriaA}{B}{El usuario requiere registrar otr tema}.
	\UCpaso El sistema genera los pasos 4-8 nuevamente de la trayectoria principal del \UCref{SP1-CU7}..
    \UCpaso[\UCactor] El usuario ingresa los datos correspondientes a la tema.
    \UCpaso{}[\UCactor] EL usuario continua con el paso 9 de la trayectoria principal del \UCref{SP1-CU7}.
    \IUref{SP1-IU}{Principal}.
\end{UCtrayectoriaA}

%------------------------ CU TRAYECTORIA ALTERNATIVA C -------------------------

\begin{UCtrayectoriaA}{D}{El docente quiere registrar Subtemas}.
	\UCpaso[\UCactor] Presiona el botón \BtnModal que se encuentra a un lado del campo ``Subtemas'' de la \IUref{SMIU3.O1}{Registrar Subtemas}.
	\UCpaso Muestra el modal \IUref{SMIU3.01}{Registrar Subtemas}.
	\UCpaso Continua en el paso X de la trayectoria principal de \UCref{SP1-CUXX}
\end{UCtrayectoriaA}

%------------------------ CU TRAYECTORIA ALTERNATIVA D -------------------------

\begin{UCtrayectoriaA}{E}{Uno o más campos obligatorios no fueron contestados.}
	\UCpaso Detecta uno o más campos sin contestar.
    \UCpaso Muestra el mensaje \MSGref{MSG4.}{Los campos marcados con (*) son obligatorios.}
    \UCpaso[\UCactor] Cierra el mensaje presionando el botón \IUbutton{Aceptar}.
    \UCpaso Continua en el paso 14 de la trayectoria principal del \UCref{SP1-CU7}.
\end{UCtrayectoriaA}
