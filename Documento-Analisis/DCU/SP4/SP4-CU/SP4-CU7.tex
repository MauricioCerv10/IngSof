\begin{UseCase}{SP4-CU7}{Consultar Recursos Humanos}{El usuario Jefe de Innovación Educativa visualiza los Recursos Humanos Registrados de su Unidad Académica.}
        \UCitem{Versión}{\color{Gray}1.1}
        \UCitem{Autor}{\color{Gray}Rivas Rojas Arturo}
        \UCitem{Supervisa}{\color{Gray}}
        \UCitem{Actor}{\hyperlink{JDIE}{Jefe de Innovación Educativa}}
        \UCitem{Propósito}{Visualizar los Recursos Humanos registrados en el sistema.}
        \UCitem{Entradas}{Las entradas para la consulta de Recursos Humanos serán:
          \begin{itemize}
            \item Cargo.
          \end{itemize}
        }
        \UCitem{Origen}{Teclado.}
        \UCitem{Salidas}{
            \begin{itemize}
                \item MSG1. Los catálogos necesarios no están disponibles por el momento, por favor inténtelo más tarde.
                \item MSG2. ¿Está seguro que desea Eliminar al Usuario con Matricula [no.matricula]?
            \end{itemize}
        }
        \UCitem{Destino}{Pantalla.}
        \UCitem{Precondiciones}{
            \begin{itemize}
                \item El catálogo de Cargo debe de estar cargado en el sistema.
            \end{itemize}
        }
        \UCitem{Postcondiciones}{Los Recursos Humanos desplegados en pantalla podrán ser editados o eliminados.}
        \UCitem{Errores}{
            \begin{itemize}
                \item E1. No existen o no pudieron cargarse los catálogo necesarios.
            \end{itemize}
        }
        \UCitem{Puntos de Exteción}{
            \begin{itemize}
                \item \UCref{SP4-CU4}: Registrar Recurso Humano.
                \item \UCref{SP4-CU10}: Editar Recurso Humano.
                \item \UCref{SP4-CU15}: Eliminar Recurso Humano.
            \end{itemize}
        }
        \UCitem{Estado}{Revisión.}
        \UCitem{Observaciones}{}
\end{UseCase}

%--------------------------- CU TRAYECTORIA PRINCIPAL -------------------------
\begin{UCtrayectoria}{Principal}

    \UCpaso[\UCactor] Presiona la opción de Gestionar Recursos Humanos del menú del la Interfaz de Inicio (abi).
    \UCpaso Carga el catálogo de Cargo. [Trayectoria A]
    \UCpaso Muestra la interfaz de usuario \IUref{IU2.4-J}{Gestionar Recursos Humanos}.
    \UCpaso[\UCactor] Selecciona el Cargo que desea como filtro. [Trayectoria B]
    \UCpaso[\UCactor] Aplica el filtro presionando el botón \IUbutton{Buscar}.
    \UCpaso Muestra tabla con los datos de los Recursos Humanos que cumplan con el filtro.
    \UCpaso Repite la trayectoria principal del \UCref{SP4-CU7} desde el paso 3. [Trayectoria C] [Trayectoria C.1]
\end{UCtrayectoria}

%------------------------ CU TRAYECTORIA ALTERNARIVA A -------------------------

\begin{UCtrayectoriaA}{A}{Los catálogos no se pudieron cargar.}
    \UCpaso Muestra el mensaje \MSGref{MSG1}{Los catálogos necesarios no están disponibles por el momento, por favor inténtelo más tarde}.
    \UCpaso[\UCactor] Cierra el mensaje presionando el botón \IUbutton{Aceptar}.
    \UCpaso Muestra la interfaz de Inicio (abi) .
\end{UCtrayectoriaA}

%------------------------ CU TRAYECTORIA ALTERNARIVA B -------------------------

\begin{UCtrayectoriaA}{B}{El actor no selecciona ningún Cargo}
    \UCpaso Muestra una tabla con los datos de todos los Recursos Humanos.
    \UCpaso Repite la trayectoria principal del \UCref{SP4-CU7} desde el paso 3.

\end{UCtrayectoriaA}

%------------------------ CU TRAYECTORIA ALTERNARIVA C -------------------------

\begin{UCtrayectoriaA}{C}{El actor presiona la casilla "Eliminar" de un Recurso Humano desplegado}
    \UCpaso Muestra el mensaje \MSGref{MSG2}{¿Está seguro que desea Eliminar al Recurso Humano con la Matricula de Empleado [no.matricula]?}.
    \UCpaso[\UCactor] Cierra el mensaje presionando el botón \IUbutton{Si}.
    \UCpaso Repite la trayectoria principal del \UCref{SP4-CU7} desde el paso 3.
\end{UCtrayectoriaA}
%------------------------ CU TRAYECTORIA ALTERNARIVA C.1 -------------------------

\begin{UCtrayectoriaA}{C.1}{El actor presiona por accidente la casilla "Eliminar" de un Recurso Humano desplegado}
    \UCpaso Muestra el mensaje \MSGref{MSG2}{¿Está seguro que desea Eliminar al Recurso Humano con la Matricula de Empleado [no.matricula]?}.
    \UCpaso[\UCactor] Cierra el mensaje presionando el botón \IUbutton{No}.
    \UCpaso Repite la trayectoria principal del \UCref{SP4-CU7} desde el paso 3.
\end{UCtrayectoriaA}

