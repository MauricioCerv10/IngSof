\begin{UseCase}{SP5-CU5}{Editar usuario.}{Los jefes (\hyperlink{JDIC}{JDIC}, \hyperlink{JDIA}{JDIA}, \hyperlink{JIE}{JIE}) podrán realizar modificaciones a la información de los empleados   a su cargo}
        \UCitem{Versión}{\color{Gray}2.0}
        \UCitem{Autor}{\color{Gray}Hernández Ruiz Rafael}
        \UCitem{Supervisa}{\color{Gray}Abigail Nicolás Sayago}
        \UCitem{Actor}{\hyperlink{JDIC}{JDIC}, \hyperlink{JDIA}{JDIA}, \hyperlink{JIE}{JIE}.}
        \UCitem{Propósito}{Poder realizar modificaciones en la información de los empleados por motivos de actualización o errores }
        \UCitem{Entradas}{
          \begin{itemize}
            	
            \item Nombre
            \item Apellido Paterno.
            \item Apellido Materno.
            \item Titulo.
            \item Cargo.
            \item Lugar de trabajo.
            \item Contraseña.
            \item Correo electrónico.
            \item Clic en botón finalizar.
        \end{itemize}}
        \UCitem{Origen}{Mouse.}
        \UCitem{Salidas}{
            \begin{itemize}
                \item Información del empleado (nombre, apellido paterno, apellido materno, correo , titulo, cargo, lugar de trabajo y contraseña).
                \item \MSGref{MSG7}{Los catálogos necesarios no se han cargado, favor de intentarlo más tarde}
                \item \MSGref{MSG27}{Empleado modificado.}
                \item \MSGref{MSG29}{¿Está seguro que desea cancelar? Se perderán todos los avances sin guardar.}
                \item \MSGref{MSG35}{Escriba información valida.}
                \item \MSGref{MSG3X}{El campo es requerido.}
           
            \end{itemize}
        }
        \UCitem{Destino}{Pantalla.}
        \UCitem{Precondiciones}{ \begin{itemize}
            \item Debe existir por lo menos un registro en el catálogo de cargos.
            \item Debe existir por lo menos un registro en el catalogo de lugares de trabajo.
        \end{itemize} }
        \UCitem{Postcondiciones}{}
        \UCitem{Errores}{ \begin{itemize}
        \item El catálogo de cargos no se cargo correctamente.
        \item Hubo un problema al conectarse con el servidor.
        \item Hubo un problema al conectarse con la base de datos. \end{itemize}}
        \UCitem{Estado}{Gestión.}
        \UCitem{Observaciones}{}

\end{UseCase}

\begin{UCtrayectoria}{Principal}
    
    \UCpaso[\UCactor] Presiona en la interfaz de usuario \IUref{IU2}{Gestionar empleados} la opción de editar un usuario. 
    \UCpaso  El sistema verifica la existencia de registros en los catálogos cargos  y  lugares de trabajo de la la \BRref{BR14}{Existen los catálogos} . [Trayectoria B] 
    \UCpaso El sistema carga la información del catalogo de empleado  según la \BRref{BR3}{Gestión de Usuarios} para el actor.
    \UCpaso El sistema carga la pantalla  \IUref{IU7}{Editar empleados}. [Trayectoria G] 
    \UCpaso[\UCactor] Edita la información de los empleados.[Trayectoria C] [Trayectoria D]
    \UCpaso[\UCactor]  Presiona el botón de \IUbutton{Finalizar}.
    \UCpaso El sistema verifica la consistencia de la información. [Trayectoria E] [Trayectoria I]
    \UCpaso El sistema realiza la modificación de la información del empleado. [Trayectoria F] 
    \UCpaso  El sistema muestra el \MSGref{MSG27}{Empleado modificado}.    
    \UCpaso El sistema regresa a la interfaz de usuario \IUref{IU2}{Gestionar empleados}
\end{UCtrayectoria}

\begin{UCtrayectoriaA}{B}{No existen registros en los catálogos.}
    \UCpaso     El sistema muestra el \MSGref{MSG7}{Los catálogos necesarios no se han cargado, favor de intentarlo más tarde}.
    \UCpaso[\UCactor] Cierra el mensaje presionando el botón \IUbutton{Aceptar}.
    \UCpaso El sistema regresa a la interfaz de usuario \IUref{IU2}{Gestionar empleados}.
\end{UCtrayectoriaA}

\begin{UCtrayectoriaA}{C}{Cambio en el cargo de un usuario.}
    \UCpaso     El sistema ajustara las opciones de zona de trabajo de acuerdo con el cargo y el actor según la \BRref{BR4}{Cada Usuario debe estar relacionado a una zona de trabajo.}
    \UCpaso     Continua en el paso 5 de la trayectoria principal del \UCref{SP5-CU5}.
\end{UCtrayectoriaA}


\begin{UCtrayectoriaA}{D}{El actor presiono el botón \IUbutton{Cancelar}.}
	\UCpaso El sistema muestra el mensaje \MSGref{MSG29}{¿Está seguro que desea cancelar? Se perderán todos los avances sin guardar.}
	\UCpaso[\UCactor] Cierra el mensaje presionando el boton \IUbutton{Aceptar} . [Trayectoria H]
	\UCpaso El sistema regresa a la interfaz de usuario \IUref{IU2}{Gestionar empleados}.
\end{UCtrayectoriaA}

\begin{UCtrayectoriaA}{E}{Se incumple la \BRref{BR3}{Gestión de Usuarios} o la \BRref{BR4}{Cada Usuario debe estar relacionado a una zona de trabajo}  o la  \BRref{BR30}{Información de los usuarios} en la definición de cada campo   .}
	\UCpaso El sistema muestra el \MSGref{MSG35}{Escriba información valida.}
	\UCpaso[\UCactor] Cierra el mensaje presionando el botón \IUbutton{Aceptar}.
	\UCpaso Continúa en el paso 5 de la trayectoria principal del \UCref{SP5-CU5}.
\end{UCtrayectoriaA}



%------------------------ CU TRAYECTORIA ALTERNARIVA H -------------------------

\begin{UCtrayectoriaA}{F}{La base de datos no está disponible, o hubo un error de conexión.}
    \UCpaso El sistema muestra el mensaje \MSGref{MSG25}{Servicios no disponibles por el momento.}.
    \UCpaso[\UCactor] Cierra el mensaje presionando el botón \IUbutton{Aceptar}.
    \UCpaso El sistema regresa a la interfaz de usuario \IUref{IU2}{Gestionar empleados}.
\end{UCtrayectoriaA}

\begin{UCtrayectoriaA}{G}{El sistema no encuentra la interfaz a mostrar o existe un problema con el servidor.}
    \UCpaso No se puede mostrar la página solicitada.
    \UCpaso El sistema muestra el \MSGref{MSG24}{Por el momento la página solicitada no esta disponible}.
    \UCpaso[\UCactor] Cierra el mensaje presionando el botón \IUbutton{Aceptar}.
    \UCpaso El sistema regresa a la interfaz de usuario \IUref{IU2}{Gestionar empleados}.
\end{UCtrayectoriaA}
\begin{UCtrayectoriaA}{H}{El actor presiono el botón \IUbutton{Cancelar}.}
	\UCpaso Cierra el mensaje y continúa en el paso 5 de la trayectoria principal del \UCref{SP5-CU5}.	
\end{UCtrayectoriaA}
\begin{UCtrayectoriaA}{I}{Uno o mas campos de la {BR30}{Información de los usuarios}  están vacíos }
	\UCpaso El sistema muestra el {MSG3X}{El campo es requerido} debajo de cada campo que no fue llenado. 
	\UCpaso	Continúa en el paso 5 de la trayectoria principal del \UCref{SP5-CU5}.
\end{UCtrayectoriaA}