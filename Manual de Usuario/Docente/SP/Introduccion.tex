\chapter{Introducción}


\section{Presentación}
El presente manual tiene como propósito contar con una guía clara y específica que garantice la óptima operación y desarrollo de las diferentes actividades del \textbf{Sistema Gestor de Programas Académicos}, relacionados a la gestión de Programas Academicos, Planes de Estudios y Recursos Humanos de las diversas Unidades Académicas del Instituto Politécnico Nacional, que están bajo el cargo del Docente de cada unidad, así como el de servir como un instrumento de apoyo y mejora institucional.\\

Comprende en forma ordenada, secuencial y detallada las operaciones de los procedimientos a seguir para cada actividad laboral, promoviendo el buen desarrollo administrativo de la Unidad Académica y dando cumplimiento con ello a las normativas del instituto.\\

Contempla la red de procesos de gestión del personal de la unidad, así como la funcionalidad de cada uno de las pantalla pertenecientes a éste módulo.\\

Es importante señalar, que este documento está sujeto a actualización en la medida que se presenten variaciones en la ejecución de los procedimientos, en la normatividad establecida, en la estructura orgánica de la Unidad Académica y el instituto en general, o bien en algún otro aspecto que influya en la operatividad del mismo, con el fin de cuidar su vigencia operativa.

\subsection{Objetivos del Manual}
Orientar a los Docentes el uso del módulo del Sistema Gestor de Programas Académicos perteneciente al trabajo inter institucional de las Unidades Académicas del Instituto Politécnico Nacional, más específicamente la gestión y control de los planes de estudios que posee.

\subsection{Objetivos particulares}
Los objetivos particulares son que el usuario conozca y pueda dominar las siguientes funcióne:
\begin{itemize}
	\item Mostrar como consultar los Planes de Estudios bajo el cargo del Docente de una unidad académica.
	\item Mostrar como editar los Planes de Estudios bajo el cargo del Docente de una unidad académica.
	\item Mostrar como eliminar los Planes de Estudios bajo el cargo del Docente de una unidad académica.
	\item Mostrar como agregar nuevos Planes de Estudios bajo el cargo del Docente de una unidad académica.

	
	
\end{itemize}
