% REGISTRAR PLAN DE ESTUDIOS.
\begin{UseCase}{SP4-CU2}{Registrar Plan de Estudios}{El usuario Docente ingresa los datos del Plan de Estudios, acorde a la entidad Plan de Estudios del Modelo de Datos.}
		\UCitem{Versión}{\color{Gray}1.0}
		\UCitem{Autor}{\color{Gray}Cervantes Moreno Christian Andres}
		\UCitem{Supervisa}{\color{Gray} Evelyn Reyes}
		\UCitem{Actor}{\hyperlink{Usuario}{Docente}}
		\UCitem{Propósito}{Registrar el año, modalidad, total de créditos (TEPIC / SATCA) y horas totales (Prácticas / Teóricas) a un Plan de Estudios.}
		\UCitem{Entradas}{Las entradas para el registro del Plan de Estudios serán:
          \begin{itemize}
          	\item Año (Tipo entero).
          	\item Créditos Totales TEPIC (Tipo double).
            \item Créditos Totales SATCA (Tipo double).
            \item Total horas / Teoría (Tipo entero).
            \item Total horas / Práctica (Tipo entero).
            \item Modalidad.
          \end{itemize}
        }
		\UCitem{Origen}{Teclado, Mouse}
		\UCitem{Salidas}{
        	\begin{itemize}
        		\item MSG1. Los campos marcados con (*) son obligatorios
                \item MSG2. Registro finalizado exitosamente.
                \item MSG3. ¿Seguro que desea cancelar?.
                \item MSG4. Los catálogos necesarios no están disponibles por el momento, favor de intentarlo más tarde.
                \item MSG5. Entrada no válida.

        	\end{itemize}
        }
		\UCitem{Destino}{Pantalla.}
		\UCitem{Precondiciones}{Debe de existir al menos un Programa Académico en el sistema.}
		\UCitem{Postcondiciones}{El Plan de Estudios queda registrado en el sistema.}
		\UCitem{Errores}{
			  \begin{itemize}
				\item E1. Entrada inválida de un caracter no entero.
				\item E2. Entrada inválida de un caracter no double.
			\end{itemize}
		}
		\UCitem{Estado}{Revisión.}
		\UCitem{Observaciones}{}
\end{UseCase}

%--------------------------- CU TRAYECTORIA PRINCIPAL -------------------------
\begin{UCtrayectoria}{Principal}

    %Usuario
    \UCpaso[\UCactor] Presiona el checkbox que se encuentra en la columna consultar de la interfaz de usuario \IUref{IU2-D}{Consultar Tareas}

	%Sistema
    \UCpaso El sistema carga el catálogo de modalidad.[Trayectoria A]


    \UCpaso Muestra la interfaz de usuario \IUref{IU2.1.1-D}{Registrar Plan de Estudios}.
    \UCpaso[\UCactor] Ingresa el año del Plan de Estudios que desea registrar.
    \UCpaso[\UCactor] Selecciona la modalidad del Plan de Estudios.
    \UCpaso[\UCactor] Ingresa los créditos totales TEPIC del Plan de Estudios.
    \UCpaso[\UCactor] Ingresa los créditos totales SATCA del Plan de Estudios.
    \UCpaso[\UCactor] Ingresa el total de Horas/Teorías del Plan de Estudios.
    \UCpaso[\UCactor] Ingresa el total de Horas/Prácticas del Plan de Estudios.
    \UCpaso[\UCactor] Termina la operación presionando el botón \IUbutton{Guardar}. [Trayectoria B] [Trayectoria C]
    \UCpaso Verifica que todos los campos marcados como obligatorios hayan sido completamente contestados. [Trayectoria D]

    \UCpaso Guarda la información del Plan de Estudios en la base de datos.

    \UCpaso El sistema muestra el mensaje \MSGref{MSG2}{Registro finalizado exitosamente}.

    \UCpaso[\UCactor] Cierra el mensaje presionando el botón \IUbutton{Aceptar}.

    \UCpaso Muestra la interfaz de usuario \IUref{IU2.1-D}{Consultar Mapa Curricular}.
\end{UCtrayectoria}

%------------------------ CU TRAYECTORIA ALTERNARIVA X -------------------------

\begin{comment}
\begin{UCtrayectoriaA}{A}{El sistema no encuentra ningún formulario para mostrar.}
	\UCpaso No encuentra ningún formulario para mostrar.
    \UCpaso El sistema muestra el mensaje \MSGref{MSG6}{Por el momento no se puede registrar la bibliografía}.
    \UCpaso[\UCactor] Cierra el mensaje presionando el botón \IUbutton{Aceptar}.
    \UCpaso Continua en el paso 1 de la trayectoria principal del \UCref{CU1}.
\end{UCtrayectoriaA}
\end{comment}

%------------------------ CU TRAYECTORIA ALTERNARIVA A -------------------------

\begin{UCtrayectoriaA}{A}{El sistema no carga el catálogo de modalidad}
	%\UCpaso[\UCactor] Presiona el botón \IUbutton{$\bigoplus$} que se encuentra a un lado del campo ``Nombre'' del formulario \IUref{IU2}{Registro de bibliografía}  [Trayectoria A.1]
	\UCpaso[\UCactor] Presiona el botón \IUbutton{registrar}
	\UCpaso Muestra el mensaje \MSGref{MSG4}{.Los catálogos necesarios no están disponibles por el momento, favor de intentarlo más tarde. }
	\UCpaso[\UCactor] Confirma la operación presionando el botón \IUbutton{Ok}.
	 \UCpaso Muestra la interfaz de usuario \IUref{IU2.1-D}{Consultar Mapa Curricular}

\end{UCtrayectoriaA}


%------------------------ CU TRAYECTORIA ALTERNARIVA B -------------------------

\begin{UCtrayectoriaA}{B}{El actor quiere cancelar el registro del Plan de Estudio.}
	\UCpaso[\UCactor] Presiona el botón \IUbutton{Cancelar}.
	\UCpaso Muestra el mensaje \MSGref{MSG3}{¿Seguro que desea cancelar el registro?}.
	\UCpaso[\UCactor] Confirma la operación presionando el botón \IUbutton{Si}.
	\UCpaso Muestra la interfaz de usuario \IUref{IU2.1-D}{Consultar Mapa Curricular}
\end{UCtrayectoriaA}



%------------------------ CU TRAYECTORIA ALTERNARIVA C -------------------------

\begin{UCtrayectoriaA}{C}{El actor presiona accidentalmente el botón Cancelar}
	\UCpaso[\UCactor] Presiona el botón \IUbutton{Cancelar}
	\UCpaso Muestra el mensaje \MSGref{MSG3}{¿Seguro que desea cancelar el registro?}.
	\UCpaso[\UCactor] Presiona el botón \IUbutton{No}.
	\UCpaso Cierra el mensaje.
	\UCpaso Continúa en el paso 8 de la trayectoria principal del \UCref{SP4-CU2}.
\end{UCtrayectoriaA}



%------------------------ CU TRAYECTORIA ALTERNARIVA D -------------------------

\begin{UCtrayectoriaA}{D}{Uno o más campos obligatorios no fueron contestados.}
	\UCpaso Detecta uno o más campos sin contestar.
	\UCpaso Muestra el mensaje \MSGref{MSG1}{Los campos marcados con (*) son obligatorios}.
	\UCpaso[\UCactor] Cierra el mensaje presionando el botón \IUbutton{Aceptar}.
	\UCpaso Continúa en el paso 2 de la trayectoria principal del \UCref{SP4-CU2}.
\end{UCtrayectoriaA}
