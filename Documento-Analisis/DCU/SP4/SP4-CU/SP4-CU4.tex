\begin{UseCase}{SP4-CU4}{Registrar Recurso Humano}{El usuario Jefe de Innovación Educativa ingresa los datos de los Recursos Humanos de su Unidad Académica.}
        \UCitem{Versión}{\color{Gray}1.1}
        \UCitem{Autor}{\color{Gray}Rivas Rojas Arturo}
        \UCitem{Supervisa}{\color{Gray}}
        \UCitem{Actor}{\hyperlink{JDIE}{Jefe de Innovación Educativa}}
        \UCitem{Propósito}{Ingresar al sistema la información necesaria de los Recursos Humanos.}
        \UCitem{Entradas}{Las entradas para el registro de un Recurso Humano serán:
          \begin{itemize}
            \item Matríacula de Empleado.
            \item Cargo.
            \item Título.
            \item Nombre.
            \item Apellido Paterno.
            \item Apellido Materno.
          \end{itemize}
        }
        \UCitem{Origen}{Teclado.}
        \UCitem{Salidas}{
            \begin{itemize}
                \item MSG32 Todos los campos son obligatorios
                \item MSG5 Registro finalizado exitosamente.
                \item MSG33 El Recurso Humano con la Matrícula de Empleado [Número de Matrícula de Empleado] ya existe.
                \item MSG29 ¿Está seguro que desea cancelar? Se perderán todos los avances sin guardar.
                \item MSG7 Los catálogos necesarios no se han cargado, favor de intentarlo más tarde.
                \item MSG25 Ha ocurrido un error con la base de datos.
            \end{itemize}
        }
        \UCitem{Destino}{Pantalla.}
        \UCitem{Precondiciones}{
            \begin{itemize}
                \item El Recurso Humano a registrar, no debe existir previamente en el sistema.
                \item El catálogo de Cargo debe de estar cargado en el sistema.
            \end{itemize}
        }
        \UCitem{Postcondiciones}{El Recurso Humano quedará registrado en el sitema, permitiendo consultarlo y relacionarlo a una Unidad de Aprendizaje.}
        \UCitem{Errores}{
            \begin{itemize}
                \item E1. No existen o no pudieron cargarse los catálogo necesarios.
            \end{itemize}
        }
        \UCitem{Estado}{Revisión.}
        \UCitem{Observaciones}{}
\end{UseCase}

%--------------------------- CU TRAYECTORIA PRINCIPAL -------------------------
\begin{UCtrayectoria}{Principal}

    \UCpaso[\UCactor] Presiona el botón \IUbutton{(+)Registrar Recurso Humano} de la interfaz de usuario \IUref{IU2.4-J}{Gestionar Recursos Humanos}
    \UCpaso Carga el catálogo de Cargo. [Trayectoria A]
    \UCpaso Carga el catálogo de Título. [Trayectoria A]
    \UCpaso Muestra la interfaz de usuario \IUref{IU2.4.1-J}{Registrar Recurso Humano}.
    \UCpaso[\UCactor] Ingresa la Matrícula de Empleado.
    \UCpaso[\UCactor] Selecciona el Cargo.
    \UCpaso[\UCactor] Selecciona el Título.
    \UCpaso[\UCactor] Ingresa el Nombre.
    \UCpaso[\UCactor] Ingresa el Apellido Paterno.
    \UCpaso[\UCactor] Ingresa el Apellido Materno.
    \UCpaso[\UCactor] Termina la operación presionando el botón \IUbutton{Guardar}. [Trayectoria B] [Trayectoria B.1]
    \UCpaso Verifica que todos los campos hayan sido contestados. [Trayectoria C]
    \UCpaso Valida que no exista un Recurso Humano con la misma Matrícula. [Trayectoria D]
    \UCpaso Persiste los datos ingresados. [Trayectoria E]
    \UCpaso Muestra el mensaje \MSGref{MSG5}{Registro finalizado exitosamente}.
    \UCpaso[\UCactor] Cierra el mensaje presionando el botón \IUbutton{Aceptar}.
    \UCpaso Muestra la interfaz de usuario \IUref{IU2.4-J}{Gestionar Recursos Humanos}.
\end{UCtrayectoria}

%------------------------ CU TRAYECTORIA ALTERNARIVA A -------------------------

\begin{UCtrayectoriaA}{A}{Los catálogos no se pudieron cargar.}
    \UCpaso Muestra el mensaje \MSGref{MSG7}{Los catálogos necesarios no se han cargado, favor de intentarlo más tarde}.
    \UCpaso[\UCactor] Cierra el mensaje presionando el botón \IUbutton{Aceptar}.
\end{UCtrayectoriaA}

%------------------------ CU TRAYECTORIA ALTERNARIVA B -------------------------

\begin{UCtrayectoriaA}{B}{El actor presiona el botón Cancelar}
    \UCpaso Muestra el mensaje \MSGref{MSG29}{¿Está seguro que desea cancelar? Se perderán todos los avances sin guardar}.
    \UCpaso[\UCactor] Cierra el mensaje presionando el botón \IUbutton{Si}.
    \UCpaso Muestra la interfaz de usuario \IUref{IU2.4-J}{Gestionar Recursos Humanos}.
\end{UCtrayectoriaA}

%------------------------ CU TRAYECTORIA ALTERNARIVA B.1 -------------------------

\begin{UCtrayectoriaA}{B.1}{El actor presiona accidentalmente el botón Cancelar}
    \UCpaso Muestra el mensaje \MSGref{MSG5}{¿Está seguro que desea cancelar el registro?}.
    \UCpaso[\UCactor] Cierra el mensaje presionando el botón \IUbutton{No}.
    \UCpaso Continúa en el paso 11 de la trayectoria principal del \UCref{SP4-CU4}.
\end{UCtrayectoriaA}

%------------------------ CU TRAYECTORIA ALTERNARIVA C -------------------------

\begin{UCtrayectoriaA}{C}{Uno o más campos no fueron contestados.}
    \UCpaso Muestra el mensaje \MSGref{MSG32}{Todos los campos son obligatorios}.
    \UCpaso[\UCactor] Cierra el mensaje presionando el botón \IUbutton{Aceptar}.
    \UCpaso Continúa en el paso 5 de la trayectoria principal del \UCref{SP4-CU4}.
\end{UCtrayectoriaA}

%------------------------ CU TRAYECTORIA ALTERNARIVA D -------------------------

\begin{UCtrayectoriaA}{D}{La Matrícula de Empleado ingresada ya está registrada en el sistema.}
    \UCpaso Muestra el mensaje \MSGref{MSG33}{El Recurso Humano con la Matrícula de Empleado [Número de Matrícula de Empleado] ya existe}.
    \UCpaso[\UCactor] Cierra el mensaje presionando el botón \IUbutton{Aceptar}.
    \UCpaso Continúa en el paso 5 de la trayectoria principal del \UCref{SP4-CU4}.
\end{UCtrayectoriaA}

%------------------------ CU TRAYECTORIA ALTERNARIVA E -------------------------

\begin{UCtrayectoriaA}{E}{Ocurre un error al momento de persistir los datos.}
    \UCpaso Muestra el mensaje \MSGref{MSG25}{Ha ocurrido un error con la base de datos.}
    \UCpaso[\UCactor] Cierra el mensaje presionando el botón \IUbutton{Aceptar}.
    \UCpaso Muestra la interfaz de usuario \IUref{IU2.4-J}{Gestionar Recursos Humanos}.
\end{UCtrayectoriaA}