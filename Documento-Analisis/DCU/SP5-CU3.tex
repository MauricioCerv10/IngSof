
\begin{UseCase}{SP5-CU3}{Consultar usuarios}{El \hyperlink{JDIC}{JDIC}, \hyperlink{JDIA}{JDIA} y  \hyperlink{JIE}{JIE}  podrá visualizar la información de los empleados.}
	\UCitem{Versión}{\color{Gray}3.0}
	\UCitem{Autor}{\color{Gray}Hernández Ruiz Rafael}
	\UCitem{Supervisa}{\color{Gray}Abigail Nicolás Sayago}
	\UCitem{Actor}{\hyperlink{JDIC}{JDIC}, \hyperlink{JDIA}{JDIA}, \hyperlink{JIE}{JIE}.}
	\UCitem{Propósito}{Ver los empleados que cada actor posee, así como realizar trabajos de gestión: editar y eliminar usuarios.}
	\UCitem{Entradas}{
		\begin{itemize}
			\item Selección del cargo de los empleados a buscar.
			\item Clic en botón buscar.
			\item Clic en botón x.
			\item Clic en botón Cancelar.
			\item Clic en botón Aceptar.
	\end{itemize}}
	\UCitem{Origen}{Mouse.}
	\UCitem{Salidas}{
		\begin{itemize}
			\item Lista de empleados de un cargo en específico con sus datos (cargo, nombre, apellido paterno, apellido materno correo, zona de trabajo). 
			\item \MSGref{MSG7}{Los catálogos necesarios no se han cargado, favor de intentarlo más tarde.}
			\item \MSGref{MSG21}{No hay usuarios registrados con ese cargo.}
			\item \MSGref{MSG22}{¿Seguro de eliminar al empleado: ''Nombre'' con correo: ''Correo'' del sistema?}
		\end{itemize}
	}
	\UCitem{Destino}{Pantalla.}
	\UCitem{Precondiciones}{ 1.- Debe existir por lo menos un registro en el catálogo de la cargos y lugares de trabajo de la \BRref{BR14}{Existen los catálogos} .}
	\UCitem{Postcondiciones}{1.- Habilita la llamada a los casos de uso  SP5-CU10, SP5-CU11.}
	\UCitem{Errores}{ \begin{itemize}
			\item El catálogo de cargos no se cargo correctamente.
			\item Hubo un problema al conectarse con el servidor.
			\item Hubo un problema al conectarse con la base de datos. \end{itemize}}
	\UCitem{Estado}{Gestión.}
	\UCitem{Observaciones}{}
	\UCitem{Puntos de extensión}{Casos de uso  \UCref{SP5-CU5}}
\end{UseCase}

\begin{UCtrayectoria}{Principal}
	
	\UCpaso[\UCactor] Presiona en la interfaz de usuario \IUref{IU1}{Menú} la opción de gestionar usuarios. 
	\UCpaso  El sistema verifica la existencia de registros del catalogo de la \BRref{BR14}{Existen los catálogos} para cargos  y carga los indicados para el actor según la \BRref{BR3}{Gestión de Usuarios} . [Trayectoria B] 
	\UCpaso El sistema carga la pantalla  \IUref{IU2}{Gestionar empleados}. [Trayectoria G] 
	\UCpaso[\UCactor] Selecciona el cargo de los empleados a buscar. 
	\UCpaso[\UCactor]  Presiona el botón de \IUbutton{Buscar}.  [Trayectoria C] 
	
	\UCpaso El sistema despliega la información de los empleados (cargo, nombre, correo, titulo, lugar de trabajo) en la parte inferior de la pantalla \IUref{IU2}{Gestionar empleados}. [Trayectoria Principal punto 4] [Trayectoria D] [Trayectoria F] 
\end{UCtrayectoria}

\begin{UCtrayectoriaA}{B}{No existen registros en el catálogo de cargos.}
	\UCpaso     El sistema muestra el \MSGref{MSG7}{Los catálogos necesarios no se han cargado, favor de intentarlo más tarde}.
	\UCpaso[\UCactor] Cierra el mensaje presionando el botón \IUbutton{Aceptar}.
	\UCpaso El sistema regresa a la interfaz de usuario \IUref{IU1}{Menú}.
\end{UCtrayectoriaA}

\begin{UCtrayectoriaA}{C}{No existen  empleados con el cargo seleccionado.}
	\UCpaso     El sistema muestra el \MSGref{MSG21}{No hay usuarios registrados con ese cargo}.
\end{UCtrayectoriaA}


\begin{UCtrayectoriaA}{D}{El actor presiona el botón \IUbutton{X}.}
	\UCpaso El sistema muestra el mensaje \MSGref{MSG22}{¿Seguro de eliminar al empleado: ''Nombre'' con correo: ''Correo'' del sistema?} solicitando confirmación. [Trayectoria E]
	\UCpaso[\UCactor] El actor presiona el botón \IUbutton{Cancelar}.
\end{UCtrayectoriaA}

\begin{UCtrayectoriaA}{E}{El actor presiona el botón \IUbutton{Aceptar}.}
	\UCpaso     El sistema elimina al empleado. [Trayectoria F]   
\end{UCtrayectoriaA}

%------------------------ CU TRAYECTORIA ALTERNARIVA F -------------------------

\begin{UCtrayectoriaA}{F}{La base de datos no está disponible, o hubo un error de conexión.}
	\UCpaso El sistema muestra el mensaje \MSGref{MSG25}{Servicios no disponibles}.
	\UCpaso[\UCactor] Cierra el mensaje presionando el botón \IUbutton{Aceptar}.
	\UCpaso Continua en el paso 2 de la trayectoria principal del \UCref{SP5-CU1}.
\end{UCtrayectoriaA}

%------------------------ CU TRAYECTORIA ALTERNARIVA G -------------------------

\begin{UCtrayectoriaA}{G}{El sistema no encuentra la interfaz a mostrar o existe un problema con el servidor.}
	\UCpaso No se puede mostrar la página solicitada.
	\UCpaso El sistema muestra el \MSGref{MSG24}{Por el momento la página solicitada no esta disponible}.
	\UCpaso[\UCactor] Cierra el mensaje presionando el botón \IUbutton{Aceptar}.
	\UCpaso El sistema regresa a la interfaz de usuario \IUref{IU2}{Gestionar empleados}.
\end{UCtrayectoriaA}



