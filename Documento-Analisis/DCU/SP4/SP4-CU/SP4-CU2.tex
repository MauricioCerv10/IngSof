% REGISTRAR PLAN DE ESTUDIOS.
\begin{UseCase}{SP4-CU2}{Registrar Plan de Estudios}{El usuario Jefe de Innovación ingresa los datos del Plan de Estudios, acorde a la entidad Plan de Estudios del Modelo de Datos.}
		\UCitem{Versión}{\color{Gray}1.0}
		\UCitem{Autor}{\color{Gray}Cervantes Moreno Christian Andres}
		\UCitem{Supervisa}{\color{Gray} Evelyn Reyes}
		\UCitem{Actor}{\hyperlink{Usuario}{Jefe de Innovación}}
		\UCitem{Propósito}{Registrar el año, modalidad, total de créditos (TEPIC / SATCA) y horas totales (Prácticas / Teóricas) a un Plan de Estudios.}
		\UCitem{Entradas}{Las entradas para el registro del Plan de Estudios serán:
          \begin{itemize}
          	\item Programa Academico.
          	\item Año.
          	\item Créditos Totales TEPIC.
            \item Créditos Totales SATCA.
            \item Total horas / Teoría.
            \item Total horas / Práctica.
            \item Modalidad.
          \end{itemize}
        }
		\UCitem{Origen}{Teclado, Mouse}
		\UCitem{Salidas}{
        	\begin{itemize}
        		\item \MSGref{MSG32}{ Todos los campos son obligatorios}
                \item \MSGref{MSG5}{Registro finalizado exitosamente.}
                \item \MSGref{MSG29}{ ¿Está seguro que desea cancelar? Se perderán todos los avances sin guardar.}
                \item \MSGref{MSG7}{Los catálogos necesarios no se han cargado, favor de intentarlo más tarde.}
                \item \MSGref{MSG35}{Escribe información valida.}
                \item \MSGref{MSG25}{Servicios no disponibles por el momento.}
        	\end{itemize}
        }
		\UCitem{Destino}{Pantalla.}
		\UCitem{Precondiciones}{Debe de existir al menos un Programa Académico en el sistema.}
		\UCitem{Postcondiciones}{El Plan de Estudios queda registrado en el sistema.}
		\UCitem{Errores}{
			  \begin{itemize}
				\item E1. Los catálogos de modalidad y programa académico no se cargaron correctamente.
				\item E2. Hubo un problema al conectarse con la base de datos.
			\end{itemize}
		}
		\UCitem{Estado}{Revisión.}
		\UCitem{Observaciones}{}
\end{UseCase}
%--------------------------- CU TRAYECTORIA PRINCIPAL -------------------------
\begin{UCtrayectoria}{Principal}
    %Usuario
    \UCpaso[\UCactor] Presiona el boton \IUbutton{+} que se encuentra de la interfaz de usuario \IUref{GPE-J}{Gestionar Planes de Estudio}
	%Sistema
	\UCpaso  El sistema verifica la existencia de registros en el catálogo modalidad de la \BRref{BR14}{Existen los catálogos} . [Trayectoria A]
    \UCpaso El sistema carga el catálogo de modalidad descrito en la \BRref{BR14}.[Trayectoria B]
    \UCpaso Muestra la interfaz de usuario \IUref{RPE-J}{Registrar Plan de Estudios}.
    \UCpaso[\UCactor] Captura la información necesaria del Plan de Estudios según la \BRref{BR33}{Información del Plan de Estudios}.
    \UCpaso[\UCactor] Termina la operación presionando el botón \IUbutton{Registrar}. [Trayectoria C] [Trayectoria D]
    \UCpaso Verifica la consistencia de la información conforme a la \BRref{BR13}{Todos los datos solicitados son obligatorios} y la \BRref{BR33}{Información de Planes de Estudios}. [Trayectoria E][Trayectoria F]
    \UCpaso Guarda la información del Plan de Estudios. [Trayectoria B]
    \UCpaso El sistema muestra el mensaje \MSGref{MSG5}{Registro finalizado exitosamente}.
    \UCpaso[\UCactor] Cierra el mensaje presionando el botón \IUbutton{Aceptar}.
    \UCpaso Muestra la interfaz de usuario \IUref{GPE-J}{Gestionar Planes de Estudios}.
\end{UCtrayectoria}
%------------------------ CU TRAYECTORIA ALTERNARIVA A -------------------------
\begin{UCtrayectoriaA}{A}{No existen registros en los catálogos.}
	\UCpaso Muestra el mensaje \MSGref{MSG7}{ Los catálogos necesarios no se han cargado, favor de intentarlo más tarde. }
	\UCpaso[\UCactor] Cierra el mensaje presionando el botón \IUbutton{Ok}.
	\UCpaso El sistema regresa a la interfaz de usuario \IUref{GPE-J}{Gestionar Planes de Estudio}
\end{UCtrayectoriaA}
%------------------------ CU TRAYECTORIA ALTERNARIVA B -------------------------
\begin{UCtrayectoriaA}{B}{La base de datos no está disponible, o hubo un error de conexión.}
	\UCpaso Muestra el mensaje \MSGref{MSG25}{Servicios no disponibles por el momento.}
	\UCpaso[\UCactor] Cierra el mensaje presionando el botón \IUbutton{Ok}.
    \UCpaso El sistema regresa a la interfaz de usuario \IUref{GPE-J}{Gestionar Planes de Estudio}
\end{UCtrayectoriaA}
%------------------------ CU TRAYECTORIA ALTERNARIVA C -------------------------
\begin{UCtrayectoriaA}{C}{El actor presiona el botón \IUbutton{Cancelar}.}
	\UCpaso Muestra el mensaje \MSGref{MSG29}{¿Está seguro que desea cancelar? Se perderán todos los avances sin guardar}.
	\UCpaso[\UCactor] Confirma la operación presionando el botón \IUbutton{Si}.
	\UCpaso Muestra la interfaz de usuario \IUref{GPE-J}{Gestionar Planes de Estudio}
\end{UCtrayectoriaA}
%------------------------ CU TRAYECTORIA ALTERNARIVA D -------------------------
\begin{UCtrayectoriaA}{D}{El actor presiona accidentalmente el botón \IUbutton{Cancelar}.}
	\UCpaso[\UCactor] Presiona el botón \IUbutton{Cancelar}
	\UCpaso Muestra el mensaje \MSGref{MSG29}{¿Está seguro que desea cancelar? Se perderán todos los avances sin guardar}.
	\UCpaso[\UCactor] Presiona el botón \IUbutton{No}.
	\UCpaso Cierra el mensaje.
	\UCpaso Continúa en el paso 5 de la trayectoria principal del \UCref{SP4-CU2}.
\end{UCtrayectoriaA}
%------------------------ CU TRAYECTORIA ALTERNARIVA E -------------------------
\begin{UCtrayectoriaA}{E}{El sistema detecta uno o más campos sin contestar.}
	\UCpaso Muestra el mensaje \MSGref{MSG44}{Este campo es requerido} debajo de cada campo obligatorio sin contestar.
	\UCpaso Continúa en el paso 5 de la trayectoria principal del \UCref{SP4-CU2}.
\end{UCtrayectoriaA}
%------------------------ CU TRAYECTORIA ALTERNARIVA F -------------------------
\begin{UCtrayectoriaA}{F}{El Sistema detecta que algun campo no cumple con la expresión regular.}
	\UCpaso Muestra el mensaje \MSGref{MSG35}{Escribe información valida} del campo que incumplió con la \BRref{BR33}{Información de Planes de Estudios}.
	\UCpaso Continúa en el paso 5 de la trayectoria principal del \UCref{SP4-CU2}.
\end{UCtrayectoriaA}