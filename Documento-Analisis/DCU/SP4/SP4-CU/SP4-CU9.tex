% EDITAR UNIDAD DE APRENDIZAJE.
\begin{UseCase}{SP4-CU9}{Editar Unidad de Aprendizaje}{El usuario Docente modifica los datos de las Unidades de Aprendizaje registradas en el sistema..}
		\UCitem{Versión}{\color{Gray}1.0}
		\UCitem{Autor}{\color{Gray}Cervantes Moreno Christian Andres}
		\UCitem{Supervisa}{\color{Gray} Evelyn Reyes}
		\UCitem{Actor}{\hyperlink{Usuario}{Jefe de Innovación}}
		\UCitem{Propósito}{Editar el nombre, créditos TEPIC, créditos SATCA, horas Teoría/Semana, horas Práctica/semana, Área de formación y semestre de una Unidad de Aprendizaje.}
		\UCitem{Entradas}{Las entradas para la modificación de la Unidad de Aprendizaje serán:
          \begin{itemize}
          		\item Nombre.
          	\item Créditos Totales TEPIC.
          	\item Créditos Totales SATCA.
          	\item  Horas Teoría/Semana
          	\item  Horas Práctica/Semana
          	\item Área de Formación.
          	\item Semestre
          \end{itemize}
        }
		\UCitem{Origen}{Teclado, Mouse}
		\UCitem{Salidas}{
        	\begin{itemize}
        		\item \MSGref{MSG32}{Todos los campos son obligatorios.}
        		\item \MSGref{MSG5}{Registro finalizado exitosamente.}
        		\item \MSGref{MSG29}{¿Está seguro que desea cancelar? Se perderán todos los avances sin guardar.}
        		\item \MSGref{MSG7}{Los catálogos necesarios no se han cargado, favor de intentarlo más tarde.}
        		\item \MSGref{MSG35}{Inconsistencia en los datos. Verifique los campos e intente de nuevo.}
        		\item \MSGref{MSG25}{Servicios no disponibles por el momento.}
                \item \MSGref{MSG31}{Los cambios se guardaron exitosamente.}
        	\end{itemize}
        }
		\UCitem{Destino}{Pantalla.}
		\UCitem{Precondiciones}{ Debe de existir al menos una Unidad de Aprendizaje registrada en el sistema.}
		\UCitem{Postcondiciones}{La Unidad de Aprendizaje queda modificada en el sistema.}
		\UCitem{Errores}{
			  \begin{itemize}
				\item E1. Los catálogos de tipo de usuario no se cargaron correctamente.
				\item E2. Hubo un problema al conectarse con la base de datos.
			\end{itemize}
		}
		\UCitem{Estado}{Revisión.}
		\UCitem{Observaciones}{}
\end{UseCase}
%--------------------------- CU TRAYECTORIA PRINCIPAL -------------------------
\begin{UCtrayectoria}{Principal}
    %Usuario
    \UCpaso[\UCactor] Presiona el botón \IUbutton{Lápiz} de una Unidad de Aprendizaje en la Interfaz de usuario  \IUref{CMC-J}{Consultar Mapa Curricular}.
    \UCpaso El sistema verifica la existencia de registros en el catálogo tipo de usuario de la \BRref{BR14}{Existen los catálogos} .
   \UCpaso El sistema carga el catálogo de tipo de usuario.[Trayectoria A]
    \UCpaso Muestra la interfaz de usuario \IUref{EUA-J}{Editar Unidad de Aprendizaje}.
 	\UCpaso[\UCactor] Captura la información que desea modificar de la Unidad de Aprendizaje según la \BRref{BR34}{Información de la Unidad de Aprendizaje}.
 	\UCpaso[\UCactor] Termina la operación presionando el botón \IUbutton{Registrar}. [Trayectoria B] [Trayectoria C]
 	\UCpaso Verifica la consistencia de la información conforme a la \BRref{BR13}{Todos los datos solicitados son obligatorios}. [Trayectoria D]
 	\UCpaso Guarda la información de la Unidad de Aprendizaje. [Trayectoria E]
 	\UCpaso El sistema muestra el mensaje \MSGref{MSG5}{Registro finalizado exitosamente}.
 	\UCpaso[\UCactor] Cierra el mensaje presionando el botón \IUbutton{Aceptar}.
 	\UCpaso Muestra la interfaz de usuario \IUref{CMC-J}{Consultar Mapa Curricular}.
\end{UCtrayectoria}
%------------------------ CU TRAYECTORIA ALTERNARIVA A -------------------------
\begin{UCtrayectoriaA}{A}{El sistema no carga el catálogo de tipo de usuario}
	\UCpaso Muestra el mensaje \MSGref{MSG7}{Los catálogos necesarios no se han cargado, favor de intentarlo más tarde.}
	\UCpaso[\UCactor] Cierra el mensaje presionando el botón \IUbutton{Aceptar}.
	 \UCpaso Muestra la interfaz de usuario \IUref{CMC-J}{Consultar Mapa Curricular}
\end{UCtrayectoriaA}
%------------------------ CU TRAYECTORIA ALTERNARIVA B -------------------------
\begin{UCtrayectoriaA}{B}{El actor presiona el botón \IUbutton{Cancelar}}
	\UCpaso Muestra el mensaje \MSGref{MSG29}{¿Está seguro que desea cancelar? Se perderán todos los avances sin guardar.}.
	\UCpaso[\UCactor] Cierra el mensaje presionando el botón \IUbutton{Si}.
	\UCpaso Muestra la interfaz de usuario \IUref{CMC-J}{Consultar Mapa Curricular}
\end{UCtrayectoriaA}
%------------------------ CU TRAYECTORIA ALTERNARIVA B.1 -------------------------
\begin{UCtrayectoriaA}{C}{El actor presiona accidentalmente el botón \IUbutton{Cancelar}}
	\UCpaso Muestra el mensaje \MSGref{MSG29}{ ¿Está seguro que desea cancelar? Se perderán todos los avances sin guardar.}.
    \UCpaso Continúa en el paso 8 de la trayectoria principal del \UCref{SP4-CU9}.
\end{UCtrayectoriaA}
%------------------------ CU TRAYECTORIA ALTERNARIVA C-------------------------
\begin{UCtrayectoriaA}{D}{El sistema detecta uno o más campos sin contestar.}
	\UCpaso Muestra el mensaje \MSGref{MSG32}{Todos los campos son obligatorios.}
	\UCpaso[\UCactor] Cierra el mensaje presionando el botón \IUbutton{Aceptar}.
    \UCpaso Continúa en el paso 6 de la trayectoria principal del \UCref{SP4-CU9}.
\end{UCtrayectoriaA}
%------------------------ CU TRAYECTORIA ALTERNARIVA E -------------------------
\begin{UCtrayectoriaA}{E}{La base de datos no está disponible, o hubo un error de conexión.}
	\UCpaso Muestra el mensaje \MSGref{MSG25}{Servicios no disponibles por el momento.}
	\UCpaso[\UCactor] Cierra el mensaje presionando el botón \IUbutton{Ok}.
	\UCpaso El sistema regresa a la interfaz de usuario \IUref{EUA-J}{Editar Unidad de Aprendizaje}
\end{UCtrayectoriaA}