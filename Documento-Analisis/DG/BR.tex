\chapter{Especificación de Reglas del Negocio}

\begin{BussinesRule}{BR1}{Jefe de División de Innovación Académica y Jefe de Departamento e Innovación curricular.}
    \BRitem[Tipo:] Estructural.
    \BRitem[Clase:] Habilitadora.
    \BRitem[Nivel:] Control.
    \BRitem[Descripción:] Solo pueden existir un Jefe de División de Innovación Académica y un Jefe de Departamento e Innovación curricular.
    \BRitem[Motivación:] Restringir el numero de Usuarios capaces de Gestionar Usuarios.
\end{BussinesRule}

\begin{BussinesRule}{BR2}{Jefes de Innovación Educativa.}
    \BRitem[Tipo:] Estructura.
    \BRitem[Clase:] Habilitadora.
    \BRitem[Nivel:] Control.
    \BRitem[Descripción:] Solo puede existir un único Jefe de Innovación Educativa por Unidad Académica.
    \BRitem[Motivación:] Tener un único responsable por Unidad Académica.
\end{BussinesRule}

\begin{BussinesRule}{BR3}{Gestión de Usuarios}
    \BRitem[Tipo:] Autorización.
    \BRitem[Clase:] Ejecutiva.
    \BRitem[Nivel:] Influencia.
    \BRitem[Descripción:] Solo los jefes (Jefe de División de Innovación Académica, Jefe de Departamento de Desarrollo e Innovación curricular y jefe de Innovación Educativa ) pueden registrar, consultar y cambiar  la información de los empleados que estén a su cargo, la organización para saber que usuario controla  la información de otro  será la siguiente:
    \begin{itemize}
        \item Jefe de División de Innovación Académica: Jefe de Departamento de Desarrollo e Innovación curricular, jefe de Innovación Educativa y analista de la DES.
        \item Jefe de Departamento de Desarrollo e Innovación curricular: ]efe de Innovación Educativa y analista de la DES.
        \item Jefe de Innovación Educativa: Docente y analista de su unidad académica.
    \end{itemize}
    \BRitem[Motivación:] Permitir a los Jefes tener un control sobre el acceso al sistema.
    \BRitem[Ejemplo Positivo:] El Jefe de División de Innovación Académica  actualiza el título de un analista.
    \BRitem[Ejemplo Negativo:] Un analista modifica  el nombre de otro analista.
\end{BussinesRule}

\begin{BussinesRule}{BR4}{Cada Usuario debe estar relacionado a una zona de trabajo.}
    \BRitem[Tipo:] Integradora.
    \BRitem[Clase:] Habilitadora.
    \BRitem[Nivel:] Control.
    \BRitem[Descripción:] La zona de trabajos de un Usuario se determinará de la siguiente manera:
    \begin{itemize}
        \item Jefe de División de Innovación Académica: DES.
        \item Jefe de Departamento de Desarrollo e Innovación curricular: DES.
        \item Jefe de Innovación Educativa: Unidad Académica.
        \item Docente: Unidad Académica de su Jefe de Innovación Educativa.
        \item Analista:  DES o Unidad Académica de su Jefe de Innovación Educativa.
    \end{itemize}
    \BRitem[Motivación:] Identificar que usuarios pertenecen a cada área y así saber que permisos tienen y a que informació pueden acceder.
    \BRitem[Ejemplo Positivo:] La zona de trabajo de un Docente creado por el Jefe de Innovación Educativa de la ESCOM es la ESCOM.
    \BRitem[Ejemplo Negativo:] La zona de trabajo del Jefe de Innovación Educativa de la ESCOM es la DES.
\end{BussinesRule}

\begin{BussinesRule}{BR5}{Máquina de Estados de una Tarea.}
    \BRitem[Tipo:] Flujo.
    \BRitem[Clase:] Habilitadora.
    \BRitem[Nivel:] Control.
    \BRitem[Descripción:] Según el estado de la Tarea son los permisos de quien puede modificar la información relacionada con esta.
    \BRitem[Motivación:] Tener un mayor control sobre la información.
    \BRitem[Ejemplo Positivo:] La Tarea de registrar Mapa Curricular se aprobó después de que el analista y el jefe la revisaran.
    \BRitem[Ejemplo Negativo:] La Tarea de registrar Unidad de Aprendizaje se empezó a revisar antes de que terminara su registro.
\end{BussinesRule}

\begin{BussinesRule}{BR6}{Tareas de Registro.}
    \BRitem[Tipo:] Flujo.
    \BRitem[Clase:] Habilitadora.
    \BRitem[Nivel:] Control.
    \BRitem[Descripción:] Solo los Docentes pueden registrar Mapas Curriculares y Unidades de Aprendizaje.
    \BRitem[Motivación:] Cuestiones de jerarquía dentro de la Unidad Académica.
    \BRitem[Ejemplo Positivo:] El usuario Docente registra un Mapa Curricular o una Unidad de Aprendizaje.
    \BRitem[Ejemplo Negativo:] El usuario Analista registra un Mapa Curricular o una Unidad de Aprendizaje.
\end{BussinesRule}

\begin{BussinesRule}{BR7}{Revisión de Tareas.}
    \BRitem[Tipo:] Flujo.
    \BRitem[Clase:] Habilitadora.
    \BRitem[Nivel:] Control.
    \BRitem[Descripción:] Solo los analistas y los jefes  pueden revisar Mapas Curriculares o Unidades de Aprendizaje.
    \BRitem[Motivación:] Cuestiones de jerarquía dentro de la Unidad Académica.
    \BRitem[Ejemplo Positivo:] El Analista revisa un Mapa Curricular y sus respectivas Unidades de Aprendizaje.
    \BRitem[Ejemplo Negativo:] El Docente revisa un Mapa Curricular y Unidades de Aprendizaje.
\end{BussinesRule}

\begin{BussinesRule}{BR8}{Aprobación de Tareas.}
    \BRitem[Tipo:] Flujo.
    \BRitem[Clase:] Habilitadora.
    \BRitem[Nivel:] Control.
    \BRitem[Descripción:] Una Tarea es aprobada si al terminar de revisar no contiene comentarios.
    \BRitem[Motivación:] Poder saber cuando una tarea a finalizado.
    \BRitem[Ejemplo Positivo:] El Jefe de Innovación Educativa no agrego comentarios por lo que se aprueba.
    \BRitem[Ejemplo Negativo:] El Jefe de Innovación Educativa agrega comentario y aun así se aprueba el documento.
\end{BussinesRule}

\begin{BussinesRule}{BR9}{Aprobación de Tareas Seccionada.}
    \BRitem[Tipo:] Condición.
    \BRitem[Clase:] Habilitadora.
    \BRitem[Nivel:] Control.
    \BRitem[Descripción:]
    \BRitem[Sentencia:] Una tarea puede ser aprobada parcialmente y las secciones aprobadas no puden ser modificadas.
    \BRitem[Motivación:] Evitar modificaciones en una sección ya aprobada.
    \BRitem[Ejemplo Positivo:] Una sección ya ha sido aprobada y no se hacen modificaciones.
    \BRitem[Ejemplo Negativo:] Una sección es aprobada y fue modificada.
\end{BussinesRule}

\begin{BussinesRule}{BR10}{Rechazo de Tareas.}
    \BRitem[Tipo:] Flujo.
    \BRitem[Clase:] Habilitadora.
    \BRitem[Nivel:] Control.
    \BRitem[Descripción:] Una Tarea es rechazada si al terminar de revisar contiene comentarios.
    \BRitem[Motivación:] Poder saber cuando una tarea debe volver al estado de Registro.
    \BRitem[Ejemplo Positivo:] El Jefe de Innovación Educativa agrego comentarios por lo que se rechaza.
    \BRitem[Ejemplo Negativo:] Ni el Jefe de Innovación Educativa ni el Analista agregaron comentarios y aun así se aprobó.
\end{BussinesRule}

\begin{BussinesRule}{BR11}{Asignar tareas.}
    \BRitem[Tipo:] Autorización.
    \BRitem[Clase:] Ejecutiva.
    \BRitem[Nivel:] Control.
    \BRitem[Descripción:] Solo los jefes pueden asignar tareas a los usuarios.
    \BRitem[Sentencia:]
    \BRitem[Motivación:] Llevar un orden y un control para la asignación de tareas.
    \BRitem[Ejemplo Positivo:] El Jefe de Departamento e Innovación curricular le asigna una tarea a un analista.
    \BRitem[Ejemplo Negativo:] El analista le asigna una tarea al Jefe de Departamento e Innovación curricular.
\end{BussinesRule}

\begin{BussinesRule}{BR12}{Tiempos de entregas.}
    \BRitem[Tipo:] Integridad.
    \BRitem[Clase:] Cronometrado.
    \BRitem[Nivel:] Control.
    \BRitem[Descripción:] Una tarea debe entregarse en la fecha indicada por el jefe de desarrollo e innovación curricular.
    \BRitem[Motivación:] Se tiene un control en las entregas de tareas.
    \BRitem[Ejemplo Positivo:] Una propuesta de unidad de aprendizaje puede ser revisada por un analista solo en el tiempo establecido.
    \BRitem[Ejemplo Negativo:] Una propuesta de unidad de aprendizaje puede ser revisada por un analista en cualquier momento.
\end{BussinesRule}

\begin{BussinesRule}{BR13}{Todos los datos solicitados son obligatorios.}
    \BRitem[Tipo:] Regla de Operación.
    \BRitem[Clase:] Habilitadora.
    \BRitem[Nivel:] Control.
    \BRitem[Descripción:] Los campos solicitados, no se pueden ser dejados en blanco.
    \BRitem[Sentencia:]
    \BRitem[Motivación: ]Que la base de datos esté siempre en un estado consistente.
    \BRitem[Ejemplo Positivo:] El usuario ingresa todos los datos solicitados y prosigue con su operación.
    \BRitem[Ejemplo Negativo: ]El usuario deja campos vacios y el sistema le permite continuar.
\end{BussinesRule}

\begin{BussinesRule}{BR14}{Existen los catálogos.}
    \BRitem[Tipo: ]Regla de inferencia de un hecho.
    \BRitem[Clase: ]Habilitadora.
    \BRitem[Nivel: ]Control.
    \BRitem[Descripción: ]Existen catálogos de:
    \begin{itemize}
        \item Modalidades\\
            Escolarizada\\
            Mixta\\
            No Escolarizada
        \item Tipo de Usuario\\
            Docente\\
            Analista\\
            Jefe de División de Innovación Académica\\
            Jefe de Departamento de Desarrollo e Innovación Curricular\\
            Jefe de Innovación Educativa
        \item Tareas\\
            Revisar Mapa Curricular\\
            Aprobar Mapa Curricular\\
            Registrar Unidad de Aprendizaje\\
            Revisar Unidad de Aprendizaje\\
            Aprobar Unidad de Aprendizaje
        \item Cargo\\
            Subdirector académico\\
            Director
        \item Titulo\\
            Abogado/a\\
            Arquitecto/a\\
            Contador público titulado\\
            Doctor/a\\
            Ingeniero/a\\
            Licenciado/a\\
            Máster || Magíster\\
            Maestro || Ministro\\
            Maestra || Ministra\\
            Técnico
    \end{itemize}
    \BRitem[Motivación: ]Mantener los datos en un estado consistente.
\end{BussinesRule}

\begin{BussinesRule}{BR15}{Planes de Estudios en re diseño por Programa Académico.}
    \BRitem[Tipo:] Relación.
    \BRitem[Clase:] Habilitadora.
    \BRitem[Nivel:] Control.
    \BRitem[Descripción:] Un Programa Académico puede tener unicamente un Plan de Estudios en rediseño.
    \BRitem[Motivación: ]Una Prograama Académico puede tener varios Planes de Estudios activos pero solo se rediseña el mas actual pues este fue el rediseño de los anterior.
    \BRitem[Ejemplo Positivo:] Ingeniería en Sistemas Computacionales: ]Plan 2019.
    \BRitem[Ejemplo Negativo:] Licenciatura en Economía: ]Plan 2000, Plan 2009, Plan 2014, Plan 2018.
\end{BussinesRule}

\begin{BussinesRule}{BR16}{La matrícula del empleado es única.}
    \BRitem[Tipo: ]Relación.
    \BRitem[Clase: ]Habilitadora.
    \BRitem[Nivel: ]Control.
    \BRitem[Descripción:] Cada una de las matrículas de los empleados registrados en el sistema cuentan con una matrícula única.
    \BRitem[Motivación: ]Poder identificar a los usuarios.
    \BRitem[Ejemplo Positivo:] El Empleado Juan Perez tiene una matricula 2014000990  y El Empleado Armando López Doriga  tiene una matricula 2014000991.
    \BRitem[Ejemplo Negativo: ]El Empleado Juan Perez tiene una matricula 2014000990 y El Empleado Armando López Doriga  tiene una matricula 2014000990.
\end{BussinesRule}

\begin{BussinesRule}{BR17} {El nombre de la Unidad de Aprendizaje es único.}
    \BRitem[Tipo: ]Relación.
    \BRitem[Clase: ]Habilitadora.
    \BRitem[Nivel: ]Control.
    \BRitem[Descripción: ]Cada  una de las Unidades de Aprendizaje dentro de un mismo Programa Académico tienen un nombre único.
    \BRitem[Motivación: ] Que no exista redundancia en la información y la base de datos este siempre en un estado consistente.
    \BRitem[Ejemplo Positivo:] En un Programa Académico  tenemos la materia Ingeniería de Software y Administración de Proyectos.
    \BRitem[Ejemplo Negativo: ]En un Programa Académico  tenemos la materia Ingeniería de Software y  Ingeniería de Software.
\end{BussinesRule}

\begin{BussinesRule}{BR18}{ Longitud máxima de Unidad de aprendizaje.}
    \BRitem[Tipo: ]Regla de inferencia de un hecho.
    \BRitem[Clase: ]Habilitadora.
    \BRitem[Nivel: ]Control.
    \BRitem[Descripción: ]El nombre de la Unidad de Aprendizaje tiene un tamaño máximo de 80 caracteres.
    \BRitem[Sentencia:]
    \BRitem[Motivación: ]Que la base de datos esté siempre en un estado consistente.
    \BRitem[Ejemplo Positivo:] El usuario llena los datos marcados con (*) y prosigue con su operación.
    \BRitem[Ejemplo Negativo: ]El usuario deja campos obligatorios y el sistema le permite continuar.
\end{BussinesRule}

\begin{BussinesRule}{BR19} {El nombre del Programa Académico es único.}
    \BRitem[Tipo: ]Relación.
    \BRitem[Clase: ]Habilitadora.
    \BRitem[Nivel: ]Control.
    \BRitem[Descripción: ]El nombre de los Programas Académicos es único dentro de la Unidad Académica.
    \BRitem[Motivación: ] Que no exista redundancia en la información y la base de datos este siempre en un estado consistente.
    \BRitem[Ejemplo Positivo:] En una Unidad Académica existen los Programas Académicos Ingeniería en Sistemas Computacionales y Ingeniería en Comunicaciones y Electrónica.
    \BRitem[Ejemplo Negativo: ]En una Unidad Académica existen los Programas Académicos Ingeniería en Sistemas Computacionales y Ingeniería en Sistemas Computacionales.
\end{BussinesRule}

\begin{BussinesRule}{BR20}{Las horas totales deben de estar entre 350 y 450 horas.}
    \BRitem[Tipo: ]Regla de operación.
    \BRitem[Clase: ]Habilitadora.
    \BRitem[Nivel: ]Control.
    \BRitem[Descripción: ]Un Plan de Estudios debe tener entre 350 y 450 horas.
    \BRitem[Motivación:] Distribuir las horas entre las unidades de aprendizaje.
    \BRitem[Ejemplo Positivo:] El usuario llena los dato de total de horas con 400 horas.
    \BRitem[Ejemplo Negativo: ] El usuario llena los dato de total de horas con 40000 horas.
\end{BussinesRule}

\begin{BussinesRule}{BR21}{Un libro tiene un ISBN único.}
    \BRitem[Tipo:] Regla de integridad referencial.
    \BRitem[Clase:] Habilitadora.
    \BRitem[Nivel:] Control.
    \BRitem[Descripción:] Cada uno de los libros registrados en el sistema cuentan con un ISBN único que los diferencia de otro libro aunque éstos coincidan en el nombre.
    \BRitem[Motivación:] Que no exista redundancia en la información y la base de datos esté siempre en un estado consistente.
    \BRitem[Ejemplo Positivo:] Un libro cuyo nombre es ``Ingeniería de Software'' conlleva un ISBN que es 978-92-95055-02-5, y existe otro libro con el mismo nombre ``Ingeniería de Software'' sin embargo su ISBN es 969-83-59550-10-3. Entonces se cumple la regla
    \BRitem[Ejemplo Negativo:] Para el planteamiento anterior no cumple la regla que el ISBN sea el mismo para ambos libros, puesto que uno es escrito por un autor, y el otro es escrito por otro autor.
\end{BussinesRule}

\begin{BussinesRule}{BR22}{Longitud del ISBN.}
    \BRitem[Tipo:] Regla de integridad estructural.
    \BRitem[Clase:] Habilitadora.
    \BRitem[Nivel:] Control.
    \BRitem[Descripción:] El ISBN se compone actualmente de trece dígitos agrupados en cinco elementos, mismos que deben estar separados por guiones de la siguiente manera:
    \begin{itemize}
        \item Prefijo Internacional (978)
        \item Identificador de grupo o Grupo de registro (607)
        \item Prefijo de editor o de Agente editor (0000)
        \item Identificador de título o publicación (00)
        \item Dígito de control o de comprobación (0)
    \end{itemize}
El resultado finalmente será: ] ISBN: ]978 - 607 - 0000 - 00 - 0
    \BRitem[Sentencia:]
    \BRitem[Motivación:] Apegarnos a las normas de la \emph{Secretaría de Cultura} y el \emph{Instituto Nacional del Derecho de Autor} para así evitar sanciones y/o incongruencias en el sistema.
    \BRitem[Ejemplo Positivo:]
    \BRitem[Ejemplo Negativo:]
\end{BussinesRule}

\begin{BussinesRule}{BR23}{Un libro con más de 3 autores es sólo citado por los primeros 3.}
    \BRitem[Tipo:] Regla de integridad estructural.
    \BRitem[Clase:] Habilitadora.
    \BRitem[Nivel:] Control.
    \BRitem[Descripción:] Se tomará en cuenta solamente los tres primeros autores que son referenciados en el libro.
    \BRitem[Motivación:] Seguir las normas de referencia de acuerdo al formato APA.
    \BRitem[Ejemplo Positivo:] Un libro cuyo número de autores es 5, se referencia a los autores de la siguiente manera:

    Apellido autor 1, iniciales del autor 1, Apellido autor 2, iniciales del autor 2 y Apellido de autor 3, iniciales del autor 3 (año de publicación)

    Ejemplo: Escartín, MªJ., Palomar, M. y Súarez, E. (1997)

    \BRitem[Ejemplo Negativo:] Un libro cuyo número de autores es 5, se referencia a los autores de la siguiente manera:

    Ejemplo: Escartín, MªJ., Palomar, M., Chavez, C., Salinas, L.A. y Súarez, E. (1997)
\end{BussinesRule}

\begin{BussinesRule}{BR24}{Debe existir al menos un criterio de evaluación para una Unidad de Aprendizaje.}
    \BRitem[Tipo:] Regla de integridad estructural.
    \BRitem[Clase:] Habilitadora.
    \BRitem[Nivel:] Control.
    \BRitem[Descripción:] Para cada Unidad de Aprendizaje debe existir al menos un criterio de evaluación cuyo porcentaje asociado sería de 100\%, sin embargo, puede tener tantos criterios como el docente decida.
    \BRitem[Motivación:] Que el docente que imparta dicha Unidad de Aprendizaje tenga al menos un referente para evaluar a sus alumnos.
    \BRitem[Ejemplo positivo:] La Unidad de Aprendizaje ``Ingeniería de Sofware'' tiene como único criterio de evaluación el desarrollo de un proyecto cuyo porcentaje equivale al 100\% de la calificación del alumno.
\end{BussinesRule}

\begin{BussinesRule}{BR25}{La suma de los porcentajes de cada evaluación debe ser igual a 100\%.}
    \BRitem[Tipo:] Regla de operación.
    \BRitem[Clase:] Habilitadora.
    \BRitem[Nivel:] Control.
    \BRitem[Descripción:] La suma total de los porcentajes de cada evaluación registrada por el docente, debe ser exactamente igual al 100\%.
    \BRitem[Motivación:] Que exista coherencia al momento de evaluar y que el alumno pueda identificar de dónde proviene su calificación.
    \BRitem[Ejemplo positivo:] Las evaluaciones registradas para la Unidad de Aprendizaje ``Ingeniería de Software'' son:
        \begin{itemize}
            \item 20\% Examen oral.
            \item 20\% Tareas.
            \item 60\% Proyecto final.
        \end{itemize}
        En total, los porcentajes suman el 100\% de la calificación del alumno.
\end{BussinesRule}

\begin{BussinesRule}{BR26}{Vigencia de una Unidad de Aprendizaje}
    \BRitem[Tipo:] Regla de integridad referencial.
    \BRitem[Clase:] Cronometrada.
    \BRitem[Nivel:] Control.
    \BRitem[Descripción:] La vigencia de una Unidad de Aprendizaje comienza en el año en el que está siendo diseñada o rediseñada hasta cinco años posteriores.
    \BRitem[Motivación:] Que los a los alumnos se les impartan las Unidades de Aprendizaje más actualizadas y no estén trabajando con cosas obsoletas.
    \BRitem[Ejemplo positivo:] Para una Unidad de Aprendizaje que está siendo diseñada o rediseñada el día 13/11/2018 cumple la regla:
        \begin{itemize}
            \item 13/11/2023.
            \item 2023.
        \end{itemize}
\end{BussinesRule}

\begin{BussinesRule}{BR27}{ Longitud máxima de matricula.}
    \BRitem[Tipo: ]Regla de inferencia de un hecho.
    \BRitem[Clase: ]Habilitadora.
    \BRitem[Nivel: ]Control.
    \BRitem[Descripción: ]La matricula tiene un tamaño máximo de 10 caracteres y es la matricula que tiene cada trabajador del IPN.
    \BRitem[Sentencia:]
    \BRitem[Motivación: ]Que la base de datos esté siempre en un estado consistente.
    \BRitem[Ejemplo Positivo:] . 2014171285
    \BRitem[Ejemplo Negativo: ]. 20111111111
\end{BussinesRule}

\begin{BussinesRule}{BR28}{ Longitud máxima de contraseña.}
    \BRitem[Tipo: ]Regla de inferencia de un hecho.
    \BRitem[Clase: ]Habilitadora.
    \BRitem[Nivel: ]Control.
    \BRitem[Descripción: ]La contraseña tiene un tamaño máximo de 6 caracteres y no debe llevar acentos.
    \BRitem[Sentencia:]
    \BRitem[Motivación: ]Que la base de datos esté siempre en un estado consistente.
    \BRitem[Ejemplo Positivo:] . abc123
    \BRitem[Ejemplo Negativo: ]. abé123333
\end{BussinesRule}


\begin{BussinesRule}{BR29}{Debe existir al menos un criterio de evaluación para la Relación de Prácticas.}
    \BRitem[Tipo:] Regla de integridad estructural.
    \BRitem[Clase:] Habilitadora.
    \BRitem[Nivel:] Control.
    \BRitem[Descripción:] Para cada Práctica debe existir al menos un criterio de evaluación cuyo porcentaje asociado sería de 100\%, sin embargo, puede tener tantos criterios como el docente decida.
    \BRitem[Motivación:] Que cada práctica tenga al menos un referente para evaluar a sus alumnos.
    \BRitem[Ejemplo positivo:] La Práctica 1 de la Unidad de Aprendizaje ``Ingeniería de Sofware'' tiene como único criterio de evaluación el desarrollo de un proyecto cuyo porcentaje equivale al 100\% de la calificación del alumno.
\end{BussinesRule}

\begin{BussinesRule}{BR30}{Información de los usuarios.}
    \BRitem[Tipo:] Regla de integridad.
    \BRitem[Clase:] Habilitadora.
    \BRitem[Nivel:] Control.
    \BRitem[Descripción:]Se debe de contar con la siguiente información para registrar un usuario:
    \begin{itemize}
        \item Nombre
        \item Apellido Paterno
        \item Apellido Materno

        Estos tres datos comparten la misma expresión regular:
        /[A-Za-záéíóúüñÁÉÍÓÚÜÑ]+( [A-Za-záéíóúüñÁÉÍÓÚÜÑ]+)*/

        \item Titulo. Se obtiene por medio de un select.
        \item Cargo. Se obtiene por medio de un select.
        \item Lugar de trabajo. Se obtiene por medio de un select.
        \item Contraseña. Expresión regular: /[^áéíóúüñÁÉÍÓÚÜÑ\s]+/

        \item Correo electrónico
    \end{itemize}
    \BRitem[Motivación:] Contar con la información necesaria de los usarios para el desarrollo de las actividades.
    \BRitem[Ejemplo positivo:] Información del usuario
    \begin{itemize}
        \item Nombre: Juan Manuel
        \item Apellido Paterno: Perez de
        \item Apellido Materno: los Santos
        \item Contraseña: juanit8banan4
        \item Correo electrónico: juanito@gmail.com
    \end{itemize}
    \BRitem[Ejemplo negativo:] Información del usuario
    \begin{itemize}
        \item Nombre: Juan Manuel Fernando
        \item Apellido Paterno: Perez de los Santos
        \item Apellido Materno: del Carmen
        \item Contraseña: 132:\_147\%\$
        \item Correo electrónico: juanito\&@gmail.com-mx/
    \end{itemize}
 \end{BussinesRule}

\begin{BussinesRule}{BR31}{Información de Programa Académico.}
    \BRitem[Tipo:] Regla de integridad.
    \BRitem[Clase:] Habilitadora.
    \BRitem[Nivel:] Control.
    \BRitem[Descripción:]Se debe de contar con la siguiente información para registrar un Programa Académico:
    \begin{itemize}
        \item Nombre
        Este dato tiene la expresión regular:
        [\wéíóúüñÁÉÍÓÚÜÑ]+( [\wáéíóúüñÁÉÍÓÚÜÑ]+)*

    \end{itemize}
    \BRitem[Motivación:] Contar con la información necesaria para la gestión de un Programa Académico.
    \BRitem[Ejemplo positivo:] Información del Programa Académico
    \begin{itemize}
        \item Nombre: Ingeniería en Sistemas Computacionales
    \end{itemize}
    \BRitem[Ejemplo negativo:] Información del Programa Académico
    \begin{itemize}
    \item Nombre: Ingeniería\% \_en\ Sistemas\ Computacionales\
    \end{itemize}
 \end{BussinesRule}

 \begin{BussinesRule}{BR32}{ Longitud máxima de nombre de Programa Académico.}
    \BRitem[Tipo: ]Regla de inferencia de un hecho.
    \BRitem[Clase: ]Habilitadora.
    \BRitem[Nivel: ]Control.
    \BRitem[Descripción: ]El nombre del Programa Acádémico tiene una longitud máxima de 150 basada en el modelo de datos.
    \BRitem[Sentencia:]
    \BRitem[Motivación: ]Que la base de datos esté siempre en un estado consistente.
    \BRitem[Ejemplo Positivo:] . Ingeniería en Sistemas Computacionales
    \BRitem[Ejemplo Negativo: ]. Ingeniería en Sistemas Computacionales Ingeniería en Sistemas Computacionales Ingeniería en Sistemas Computacionales Ingeniería en Sistemas Computacionales Ingeniería en Sistemas Computacionales
\end{BussinesRule}

\begin{BussinesRule}{BR33}{Información del Plan de Estudio.}
	\BRitem[Tipo:] Regla de integridad.
	\BRitem[Clase:] Habilitadora.
	\BRitem[Nivel:] Control.
	\BRitem[Descripción:]Se debe de contar con la siguiente información para registrar el Plan de Estudios:
	\begin{itemize}
		\item Programa Académico. Se obtiene por medio de un select.
		\item Año. Expresión regular:[0-9]\{4\}
		\item Modalidad. Se obtiene por medio de un select.
		\item Creditos totales TEPIC. Expresión Regular: \textbackslash d+(\textbackslash.\textbackslash d+)?
		\item Creditos totales SATCA. Expresión Regular: \textbackslash d+(\textbackslash.\textbackslash d+)?
		\item Total Horas/ Teoría. Expresión Regular: [1-9]\{2\}
		\item Total Horas/ Práctica. Expresión Regular: [1-9]\{2\}
	\end{itemize}

	\BRitem[Motivación:] Contar con la información necesaria del Plan de Estudio para el desarrollo de las actividades.
	\BRitem[Ejemplo positivo:] Información del Plan de Estudios
	\begin{itemize}
		\item Año: 2008
		\item Creditos totales TEPIC: 2.3
		\item Creditos totales SATCA: 1.8
		\item Total Horas/ Teoría: 16
		\item Total Horas/ Práctica: 9
	\end{itemize}
	\BRitem[Ejemplo negativo:] Información del Plan de Estudios
	\begin{itemize}
		\item Año: 18
		\item Creditos totales TEPIC: 302.5 6
		\item Creditos totales SATCA: 1 . 5
		\item Total Horas/ Teoría: 104
		\item Total Horas/ Práctica: 568
	\end{itemize}
\end{BussinesRule}

%\begin{BussinesRule}{}{}
%   \BRitem[Tipo:]
%   \BRitem[Clase:]
%   \BRitem[Nivel:]
%   \BRitem[Descripción:]
%   \BRitem[Motivación:]
    %\BRitem[Ejemplo positivo:]
%\end{BussinesRule}
