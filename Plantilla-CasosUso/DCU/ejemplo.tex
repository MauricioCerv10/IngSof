\chapter{Especificación de Casos de Uso}

% REGISTRAR BIBLIOGRAFÍA.
\begin{UseCase}{CU1}{Registrar referencias bibliográficas}{El usuario podrá registrar una o más referencias bibliográficas en el sistema.}
		\UCitem{Versión}{\color{Gray}1.0}
		\UCitem{Autor}{\color{Gray}Ramírez Martínez Janet Naibi}
		\UCitem{Supervisa}{\color{Gray}}
		\UCitem{Actor}{\hyperlink{Usuario}{Usuario}}
		\UCitem{Propósito}{Asignar las referencias bibliográficas sugeridas a distintas unidades de aprendizaje a través de un identificador único.}
		\UCitem{Entradas}{Las entradas para el registro de una referencia bibliográfica serán:
          \begin{itemize}
          	\item ISBN.
          	\item Título del libro.
            \item Año de publicación.
            \item Edición.
            \item Ciudad de publicación.
            \item Editorial.
          \end{itemize}
Las entradas para el autor del libro serán
		\begin{itemize}
        	\item Identificador.
			\item Nombre del autor.
            \item Primer apellido.
            \item Segundo apellido.
		\end{itemize}        	
        }
		\UCitem{Origen}{Teclado.}
		\UCitem{Salidas}{
        	\begin{itemize}
        		\item MSG1. Los campos marcados con (*) son obligatorios
                \item MSG2. Registro finalizado exitosamente.
                \item MSG3 El libro con el ISBN [Número de ISBN] ya existe.
                \item MSG5 ¿Seguro que desea cancelar el registro?
                \item MSG6 Por el momento no se puede registrar la bibliografía.
                
        	\end{itemize}
        }
		\UCitem{Destino}{Pantalla.}
		\UCitem{Precondiciones}{ El libro que se desea registrar, no debe existir previamente en el sistema.}
		\UCitem{Postcondiciones}{La bibliografía queda registrada en el sistema.}
		\UCitem{Errores}{}
		\UCitem{Estado}{Revisión.}
		\UCitem{Observaciones}{}
\end{UseCase}

%--------------------------- CU TRAYECTORIA PRINCIPAL -------------------------
\begin{UCtrayectoria}{Principal}
    
    \UCpaso[\UCactor] Presiona el botón \IUbutton{Registrar bibliografía} de la interfaz de usuario \IUref{IU1}{Página Principal}
    
    %\UCpaso El sistema busca el formulario \Formref{FM1}{Registrar bibliografía}. [Trayectoria A]
    
    \UCpaso Muestra la interfaz de usuario \IUref{IU2}{Registro de bibliografía}.
    \UCpaso[\UCactor] Ingresa el título del libro que desea registrar.
    \UCpaso[\UCactor] Ingresa el ISBN asociado al libro que está registrando. 
    \UCpaso[\UCactor] Ingresa el nombre de la editorial que publicó el libro.
    \UCpaso[\UCactor] Ingresa el año de publicación del libro que se está registrando.
    \UCpaso[\UCactor] Ingresa la edición del libro.
    \UCpaso[\UCactor] Ingresa la ciudad de publicación.
    \UCpaso[\UCactor] Ingresa el nombre, apellido paterno y apellido materno  del autor del libro que se está registrando. [Trayectoria A]
    
    %\UCpaso[\UCactor] Ingresa los datos que se le solicitan en el formulario \Formref{FM1}{Registrar bibliografía}. [Trayectoria B]
    
    \UCpaso[\UCactor] Termina la operación presionando el botón \IUbutton{Finalizar}. [Trayectoria B] [Trayectoria C]
    
    \UCpaso Verifica que todos los campos marcados como obligatorios hayan sido completamente contestados. [Trayectoria D]
    
    \UCpaso Valida que no exista un libro con el mismo ISBN. [Trayectoria E]
    
    \UCpaso Guarda la información de la bibliografía en la base de datos.
    
    \UCpaso El sistema muestra el mensaje \MSGref{MSG2}{Registro finalizado exitosamente}.
    
    \UCpaso[\UCactor] Cierra el mensaje presionando el botón \IUbutton{Aceptar}.
    
    \UCpaso Muestra la interfaz de usuario \IUref{IU1}{Página principal}.
\end{UCtrayectoria}

%------------------------ CU TRAYECTORIA ALTERNARIVA X -------------------------

\begin{comment}
\begin{UCtrayectoriaA}{A}{El sistema no encuentra ningún formulario para mostrar.}
	\UCpaso No encuentra ningún formulario para mostrar.
    \UCpaso El sistema muestra el mensaje \MSGref{MSG6}{Por el momento no se puede registrar la bibliografía}.
    \UCpaso[\UCactor] Cierra el mensaje presionando el botón \IUbutton{Aceptar}.
    \UCpaso Continua en el paso 1 de la trayectoria principal del \UCref{CU1}.
\end{UCtrayectoriaA}
\end{comment}

%------------------------ CU TRAYECTORIA ALTERNARIVA A -------------------------

\begin{UCtrayectoriaA}{A}{El usuario quiere agregar más autores para el libro que está registrando.}
	%\UCpaso[\UCactor] Presiona el botón \IUbutton{$\bigoplus$} que se encuentra a un lado del campo ``Nombre'' del formulario \IUref{IU2}{Registro de bibliografía}  [Trayectoria A.1]
    \UCpaso Verifica que no se exceda el límite de autores para un libro con base en la regla \BRref{BR4}{Un libro con más de 3 autores es sólo citado por los primeros 3}. [Trayectoria A.2]
    \UCpaso Agrega los campos para registrar a un nuevo autor.
    \UCpaso[\UCactor] Llena los campos: ``nombre'', ``apellido paterno'' y ``apellido materno'' del nuevo autor.
    \UCpaso Continua en el paso 10 de la trayectoria principal del \UCref{CU1}
\end{UCtrayectoriaA}

%------------------------ CU TRAYECTORIA ALTERNARIVA A.1 -----------------------

\begin{UCtrayectoriaA}{A.1}{El usuario presionó por error el botón \IUbutton{$\bigoplus$}.}
	%\UCpaso[\UCactor] Presiona el botón \IUbutton{$\ominus$} que se encuentra a un lado del campo ``Nombre'' del formulario \Formref{IU2}{Registro de bibliografía}
    \UCpaso Elimina los campos agregados.
    \UCpaso Continua en el paso 9 de la trayectoria principal del \UCref{CU1}
\end{UCtrayectoriaA}

%------------------------ CU TRAYECTORIA ALTERNARIVA A.2 -----------------------

\begin{UCtrayectoriaA}{A.2}{El usuario ya registró 3 autores para un libro}
	\UCpaso Oculta el botón \IUbutton{$\bigoplus$}.
    \UCpaso Continua en el paso 10 de la trayectoria principal del \UCref{1}
\end{UCtrayectoriaA}

%------------------------ CU TRAYECTORIA ALTERNARIVA B -------------------------

\begin{UCtrayectoriaA}{B}{El actor quiere cancelar el registro de la bibliografía.}
	\UCpaso[\UCactor] Presiona el botón \IUbutton{Cancelar}.
    \UCpaso Muestra el mensaje \MSGref{MSG5}{¿Seguro que desea cancelar el registro?}.
    \UCpaso[\UCactor] Confirma la operación presionando el botón \IUbutton{Si}.
    \UCpaso Muestra la interfaz de usuario \IUref{IU1}{Página principal}
\end{UCtrayectoriaA}
 

%------------------------ CU TRAYECTORIA ALTERNARIVA C -------------------------

\begin{UCtrayectoriaA}{C}{El actor presiona accidentalmente el botón Cancelar}
	\UCpaso[\UCactor] Presiona el botón \IUbutton{Cancelar}
    \UCpaso Muestra el mensaje \MSGref{MSG5}{¿Seguro que desea cancelar el registro?}.
    \UCpaso[\UCactor] Presiona el botón \IUbutton{No}.
    \UCpaso Cierra el mensaje.
    \UCpaso Continua en el paso 10 de la trayectoria principal del \UCref{CU1}.
\end{UCtrayectoriaA}

%------------------------ CU TRAYECTORIA ALTERNARIVA D -------------------------

\begin{UCtrayectoriaA}{D}{Uno o más campos obligatorios no fueron contestados.}
	\UCpaso Detecta uno o más campos sin contestar.
    \UCpaso Muestra el mensaje \MSGref{MSG1}{Los campos marcados con (*) son obligatorios}.
    \UCpaso[\UCactor] Cierra el mensaje presionando el botón \IUbutton{Aceptar}.
    \UCpaso Continua en el paso 3 de la trayectoria principal del \UCref{CU1}.
\end{UCtrayectoriaA}

%------------------------ CU TRAYECTORIA ALTERNARIVA E -------------------------

\begin{UCtrayectoriaA}{E}{El ISBN que se intenta registrar ya existe en el sistema.}
	\UCpaso Encuentra un libro registrado con el mismo ISBN que el libro que se está registrando.
    \UCpaso Muestra el mensaje \MSGref{MSG3}{El libro con el ISBN [Número de ISBN] ya existe.}
    \UCpaso[\UCactor] Cierra el mensaje presionando el botón \IUbutton{Aceptar}
    \UCpaso Continua en el paso 3 de la trayectoria principal del \UCref{CU1}.
\end{UCtrayectoriaA}