\begin{UseCase}{SP4-CU4}{Registrar Recurso Humano}{El usuario Jefe de Innovación Educativa ingresa los datos de los Recursos Humanos de su Unidad Académica.}
        \UCitem{Versión}{\color{Gray}1.1}
        \UCitem{Autor}{\color{Gray}Rivas Rojas Arturo}
        \UCitem{Supervisa}{\color{Gray}}
        \UCitem{Actor}{\hyperlink{JDIE}{Jefe de Innovación Educativa}}
        \UCitem{Propósito}{Ingresar al sistema la información necesaria de los Recursos Humanos.}
        \UCitem{Entradas}{Las entradas para el registro de un Recurso Humano serán:
          \begin{itemize}
            \item Nombre.
            \item Primer Apellido.
            \item Segundo Apellido.
            \item Cargo.
            \item Título.
            \item Lugar de Trabajo
          \end{itemize}
        }
        \UCitem{Origen}{Teclado.}
        \UCitem{Salidas}{
            \begin{itemize}
                \item \MSGref{MSG5}{Registro finalizado exitosamente.}
                \item \MSGref{MSG7}{Los catálogos necesarios no se han cargado, favor de intentarlo más tarde.}
                \item \MSGref{MSG29}{¿Está seguro que desea cancelar? Se perderán todos los avances sin guardar.}
                \item \MSGref{MSG44}{Este campo es requerido.}
                \item \MSGref{MSG25}{Servicios no disponibles por el momento.}
            \end{itemize}
        }
        \UCitem{Destino}{Pantalla.}
        \UCitem{Precondiciones}{
            \begin{itemize}
                \item El catálogo de Cargo debe de estar cargado en el sistema.
            \end{itemize}
        }
        \UCitem{Postcondiciones}{El Recurso Humano quedará registrado en el sitema, permitiendo consultarlo y relacionarlo a una Unidad de Aprendizaje.}
        \UCitem{Errores}{
            \begin{itemize}
                \item E1. No existen o no pudieron cargarse los catálogo necesarios.
            \end{itemize}
        }
        \UCitem{Estado}{Revisión.}
        \UCitem{Observaciones}{}
\end{UseCase}
%--------------------------- CU TRAYECTORIA PRINCIPAL -------------------------
\begin{UCtrayectoria}{Principal}
    \UCpaso[\UCactor] Presiona el botón \IUbutton{(+)Registrar Recurso Humano} de la interfaz de usuario \IUref{GRH-J}{Gestionar Recursos Humanos}
    \UCpaso Carga el catálogo de Cargo descrito en la \BRref{BR14}{Catálogos existentes}. [Trayectoria A]
    \UCpaso Carga el catálogo de Título descrito en la \BRref{BR14}{Catálogos existentes}. [Trayectoria A]
    \UCpaso Muestra la interfaz de usuario \IUref{RRH-J}{Registrar Recurso Humano}.
    \UCpaso[\UCactor] Ingresa el Nombre.
    \UCpaso[\UCactor] Ingresa el Primer Apellido.
    \UCpaso[\UCactor] Ingresa el Segundo Apellido.
    \UCpaso[\UCactor] Selecciona uno o múltiples Cargos.
    \UCpaso[\UCactor] Selecciona el Título.
    \UCpaso[\UCactor] Selecciona uno o múltiples Lugar de Trabajo.
    \UCpaso[\UCactor] Termina la operación presionando el botón \IUbutton{Registrar}. [Trayectoria B] [Trayectoria B.1]
    \UCpaso Valida los datos conforme al Modelo de datos. [Trayectoria C][Trayectoria D]
    \UCpaso Persiste los datos ingresados. [Trayectoria E]
    \UCpaso Muestra el mensaje \MSGref{MSG5}{Registro finalizado exitosamente}.
    \UCpaso[\UCactor] Cierra el mensaje presionando el botón \IUbutton{Aceptar}.
    \UCpaso Muestra la interfaz de usuario \IUref{GRH-J}{Gestionar Recursos Humanos}.
\end{UCtrayectoria}
%------------------------ CU TRAYECTORIA ALTERNARIVA A -------------------------
\begin{UCtrayectoriaA}{A}{Los catálogos de la \BRref{BR14}{Catálogos existentes} necesarios no se pudieron cargar.}
    \UCpaso Muestra el mensaje \MSGref{MSG7}{Los catálogos necesarios no se han cargado, favor de intentarlo más tarde}.
    \UCpaso[\UCactor] Cierra el mensaje presionando el botón \IUbutton{Aceptar}.
    \UCpaso Muestra la interfaz de usuario \IUref{GRH-J}{Gestionar Recursos Humanos}.
\end{UCtrayectoriaA}
%------------------------ CU TRAYECTORIA ALTERNARIVA B -------------------------
\begin{UCtrayectoriaA}{B}{El actor presiona el botón \IUbutton{Cancelar}}
    \UCpaso Muestra el mensaje \MSGref{MSG29}{¿Está seguro que desea cancelar? Se perderán todos los avances sin guardar}.
    \UCpaso[\UCactor] Cierra el mensaje presionando el botón \IUbutton{Si}.
    \UCpaso Muestra la interfaz de usuario \IUref{GRH-J}{Gestionar Recursos Humanos}.
\end{UCtrayectoriaA}
%------------------------ CU TRAYECTORIA ALTERNARIVA B.1 -------------------------
\begin{UCtrayectoriaA}{B.1}{El actor presiona accidentalmente el botón \IUbutton{Cancelar}}
    \UCpaso Muestra el mensaje \MSGref{MSG5}{¿Está seguro que desea cancelar el registro?}.
    \UCpaso[\UCactor] Cierra el mensaje presionando el botón \IUbutton{No}.
    \UCpaso Continúa en el paso 11 de la trayectoria principal del \UCref{SP4-CU4}.
\end{UCtrayectoriaA}
%------------------------ CU TRAYECTORIA ALTERNARIVA C -------------------------
\begin{UCtrayectoriaA}{C}{El Sistema detecta uno o más campos obligatorios sin contestar.}
    \UCpaso Muestra el mensaje \MSGref{MSG44}{Este campo es requerido}  debajo de cada campo obligatorio sin contestar.
    \UCpaso Continúa en el paso 5 de la trayectoria principal del \UCref{SP4-CU4}.
\end{UCtrayectoriaA}
%------------------------ CU TRAYECTORIA ALTERNARIVA D -------------------------
\begin{UCtrayectoriaA}{D}{El Sistema detecta inconsistencia en los datos}
    \UCpaso Muestra el mensaje \MSGref{MSG35}{ Escribe información válida} debajo del campo inconsistente.
    \UCpaso Continúa en el paso 4 de la trayectoria principal del \UCref{SP4-CU11}.
\end{UCtrayectoriaA}
%------------------------ CU TRAYECTORIA ALTERNARIVA E -------------------------
\begin{UCtrayectoriaA}{E}{Ocurre un error al momento de persistir los datos.}
    \UCpaso Muestra el mensaje \MSGref{MSG25}{Servicios no disponibles por el momento.}
    \UCpaso[\UCactor] Cierra el mensaje presionando el botón \IUbutton{Aceptar}.
    \UCpaso Muestra la interfaz de usuario \IUref{GRH-J}{Gestionar Recursos Humanos}.
\end{UCtrayectoriaA}