% REGISTRAR AUTOR.
\begin{UseCase}{SP1-CU6}{Registrar autor de la bibliografía}{El usuario podrá registrar al autor de una bibliografía.}
        \UCitem{Versión}{\color{Gray}1.0}
        \UCitem{Autor}{\color{Gray}Ramírez Martínez Janet Naibi}
        \UCitem{Supervisa}{\color{Gray}Cervantes Delgadillo Mauricio}
        \UCitem{Actor}{\hyperlink{Docente}{Docente}}
        \UCitem{Propósito}{Registrar al autor o a los autores de la bibliografía que está siendo registrada por el usuario.}
        \UCitem{Entradas}{Las entradas para el registro de un autor serán:
          \begin{itemize}
            \item Nombre.
            \item Primer Apellido.
            \item Segundo Apellido.
          \end{itemize}
        }
        \UCitem{Origen}{Teclado y Mouse}
        \UCitem{Salidas}{
            \begin{itemize}
                \item MSG4. Los campos marcados con (*) son obligatorios.
                \item MSG5. Registro finalizado exitosamente.
                \item MSG6. ¿Seguro que desea cancelar el registro?
                \item MSG28. El autor: "Nombre Primer Apellido Segundo Apellido" ya existe.
            \end{itemize}
        }
        \UCitem{Destino}{Pantalla.}
        \UCitem{Precondiciones}{El autor que se quiere registrar no debe existir previamente en el catálogo de autores que se le muestra al usuario en la \IUref{IU7}{Registrar bibliografía}.}
        \UCitem{Postcondiciones}{
            \begin{itemize}
                \item El autor queda registrado en la base de datos.
                \item El autor aparece en el catálogo que se le muestra al usuario en la \IUref{IU7}{Registrar bibliografía}.
            \end{itemize}}
        \UCitem{Errores}{}
        \UCitem{Estado}{Revisión.}
        \UCitem{Observaciones}{}
\end{UseCase}

%--------------------------- CU TRAYECTORIA PRINCIPAL -------------------------
\begin{UCtrayectoria}{Principal}

    \UCpaso[\UCactor] Ingresa el nombre del autor.

    \UCpaso[\UCactor] Ingresa el primer apellido del autor.

    \UCpaso[\UCactor] Ingresa el segundo apellido del autor.

    \UCpaso[\UCactor] Termina la operación presionando el botón \IUbutton{Registrar}. [Trayectoria A]

    \UCpaso Verifica que todos los campos marcados como obligatorios hayan sido completamente contestados. [Trayectoria B]

    \UCpaso Valida que no exista un autor registrado con el mismo nombre, primer apellido y segundo apellido. [Trayectoria C]

    \UCpaso Guarda la información del autor en la base de datos.

    \UCpaso El sistema muestra el mensaje \MSGref{MSG5.}{Registro finalizado exitosamente.}

    \UCpaso[\UCactor] Cierra el mensaje presionando el botón \IUbutton{Aceptar}.

    \UCpaso Muestra la interfaz de usuario \IUref{IU7}{Registrar bibliografía de la UA}.
\end{UCtrayectoria}

%------------------------ CU TRAYECTORIA ALTERNATIVA A -------------------------

\begin{UCtrayectoriaA}{A}{El usuario ya no requiere un nuevo autor.}
    \UCpaso[\UCactor] Presiona el botón \IUbutton{$\bigotimes$} del modal \IUref{IU7.1}{Registrar autor de la bibliografía}.
    \UCpaso Cierra el modal.
    \UCpaso Continua en el paso 4 de la trayectoria principal del \UCref{SP1-CU1}.
\end{UCtrayectoriaA}


%------------------------ CU TRAYECTORIA ALTERNATIVA B -------------------------

\begin{UCtrayectoriaA}{C}{Uno o más campos obligatorios no fueron contestados.}
    \UCpaso Detecta uno o más campos sin contestar.
    \UCpaso Muestra el mensaje \MSGref{MSG4.}{Los campos marcados con (*) son obligatorios.}
    \UCpaso[\UCactor] Cierra el mensaje presionando el botón \IUbutton{Aceptar}.
    \UCpaso Continua en el paso 4 de la trayectoria principal del \UCref{SP1-CU1}.
\end{UCtrayectoriaA}

%------------------------ CU TRAYECTORIA ALTERNARIVA C -------------------------

\begin{UCtrayectoriaA}{D}{El autor que se intenta registrar ya existe en el sistema.}
    \UCpaso Encuentra un autor registrado con el mismo nombre, primer apellido y segundo apellido que el autor que se está registrando.
    \UCpaso Muestra el mensaje \MSGref{MSG28.}{El autor: "Nombre Primer Apellido Segundo Apellido" ya existe.}
    \UCpaso[\UCactor] Cierra el mensaje presionando el botón \IUbutton{Aceptar}.
    \UCpaso Continua en el paso 1 de la trayectoria principal del \UCref{SP1-CU6}.
\end{UCtrayectoriaA}