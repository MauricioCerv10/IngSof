\begin{UseCase}{SP5-CU4}{Registrar Usuario.}{El \hyperlink{JDIC}{JDIC}, \hyperlink{JDIA}{JDIA} y  \hyperlink{JIE}{JIE} podrán crear las cuentas de los empleados que están a su cargo.}
        \UCitem{Versión}{\color{Gray}1.0}
        \UCitem{Autor}{\color{Gray}Hernández Ruiz Rafael}
        \UCitem{Supervisa}{\color{Gray}Ramos Diaz Enrique}
        \UCitem{Actor}{\hyperlink{JDIC}{JDIC}, \hyperlink{JDIA}{JDIA}, \hyperlink{JIE}{JIE}.}
        \UCitem{Propósito}{Contar con los usuarios involucrados en el proceso para que interactúen en el sistema. }
        \UCitem{Entradas}{
          \begin{itemize}
            \item Matricula del empleado.
            \item Apellido Paterno.
            \item Apellido Materno.
            \item Titulo.
            \item Cargo.
            \item Lugar de trabajo.
            \item Contraseña.
            \item Clic en el botón registrar.
        \end{itemize}}
        \UCitem{Origen}{Mouse y teclado.}
        \UCitem{Salidas}{
            \begin{itemize}
                \item \MSGref{MSG27}{Los catálogos necesarios no se han cargado, favor de intentarlo más tarde.}
                \item \MSGref{MSG25}{Servicios no disponibles por el momento.}
                \item \MSGref{MSG5}{Registro finalizado exitosamente.}
                \item \MSGref{MSG35}{Inconsistencia en los datos. Verifique los campos e intente de nuevo.}
                \item \MSGref{MSG36}{Error, el usuario ya existe en el sistema.}
            \end{itemize}
        }
        \UCitem{Destino}{Pantalla.}
        \UCitem{Precondiciones}{ \begin{itemize}
            \item Debe existir por lo menos un registro en el catálogo de cargos.
            \item Debe existir por lo menos un registro en el catalogo de lugares de trabajo.
        \end{itemize} }
        \UCitem{Postcondiciones}{La gestión del nuevo usuario.}
        \UCitem{Errores}{ \begin{itemize}
        \item El catálogo de cargos no se cargo correctamente.
        \item Hubo un problema al conectarse con la base de datos. \end{itemize}}
        \UCitem{Estado}{Gestión y registro.}
        \UCitem{Observaciones}{}

\end{UseCase}

\begin{UCtrayectoria}{Principal.}
    
  \UCpaso[\UCactor] Presiona en la interfaz de usuario \IUref{IU2}{Gestionar empleados} el botón \IUbutton{+}.
    \UCpaso  El sistema verifica la existencia de registros en los catálogos cargos  y  lugares de trabajo. [Trayectoria B] 
    \UCpaso El sistema carga la información del catalogo de empleado  según la \BRref{BR3}{Gestión de Usuarios} para el actor. [Trayectoria H]
    \UCpaso El sistema carga la pantalla  \IUref{IU8}{Registrar empleados}. [Trayectoria I]
    \UCpaso[\UCactor] Captura la información del empleado.[Trayectoria C] [Trayectoria D] [Trayectoria G]
    \UCpaso[\UCactor]  Presiona el botón de \IUbutton{Registrar}.
    \UCpaso El sistema verifica la consistencia de la información. [Trayectoria E] [Trayectoria F]
    \UCpaso El sistema registra al empleado. [Trayectoria H]
    \UCpaso  El sistema muestra el \MSGref{MSG5}{Registro finalizado exitosamente}.    
    \UCpaso[\UCactor] Cierra el mensaje presionando el botón \IUbutton{Aceptar}.
    \UCpaso El sistema regresa a la interfaz de usuario \IUref{IU2}{Gestionar empleados}.
\end{UCtrayectoria}


\begin{UCtrayectoriaA}{B}{No existen registros en los catálogos.}
    \UCpaso El sistema muestra el mensaje \hyperref[MSG7]{MSG7. Los catálogos necesarios no se han cargado, favor de intentarlo más tarde.}.
	\UCpaso[\UCactor] Cierra el mensaje presionando el botón \IUbutton{Aceptar}.
     \UCpaso El sistema regresa a la interfaz de usuario \IUref{IU2}{Gestionar empleados}.
\end{UCtrayectoriaA}

\begin{UCtrayectoriaA}{C}{El actor selecciona el cargo del empleado.}
    \UCpaso     El sistema ajustara las opciones de zona de trabajo de acuerdo con el cargo y el actor según la \BRref{BR4}{Cada Usuario debe estar relacionado a una zona de trabajo.}
    \UCpaso Continua en el paso 5 de la trayectoria principal del \UCref{SP5-CU4}.
\end{UCtrayectoriaA}

\begin{UCtrayectoriaA}{D}{El actor presiono el botón \IUbutton{Cancelar}.}
 \UCpaso El sistema regresa a la interfaz de usuario \IUref{IU2}{Gestionar empleados}.
\end{UCtrayectoriaA}

\begin{UCtrayectoriaA}{E}{El usuario ya estaba registrado en el sistema.}
 \UCpaso El sistema muestra el \MSGref{MSG36}{Error, el usuario ya existe en el sistema.} 
 \UCpaso[\UCactor] Cierra el mensaje presionando el botón \IUbutton{Aceptar}.
 \UCpaso Continua en el paso 4 de la trayectoria principal del \UCref{SP5-CU4}.
\end{UCtrayectoriaA}

\begin{UCtrayectoriaA}{F}{Los datos están incompletos o incorrectos.}
 \UCpaso El sistema muestra el \MSGref{MSG35}{Inconsistencia en los datos. Verifique los campos e intente de nuevo.}
 \UCpaso[\UCactor] Cierra el mensaje presionando el botón \IUbutton{Aceptar}.
 \UCpaso Continua en el paso 5 de la trayectoria principal del \UCref{SP5-CU4}.
\end{UCtrayectoriaA}

\begin{UCtrayectoriaA}{G}{El actor presiona el \IUbutton{+} que esta a un costado del cargo.}
 \UCpaso El sistema agrega otro selector de cargo y la información de este según la \BRref{BR3}{Gestión de Usuarios} para el actor.
 \UCpaso Continua en el paso 5 de la trayectoria principal del \UCref{SP5-CU4}.
\end{UCtrayectoriaA}

%------------------------ CU TRAYECTORIA ALTERNARIVA H -------------------------

\begin{UCtrayectoriaA}{H}{La base de datos no está disponible, o hubo un error de conexión.}
    \UCpaso El sistema muestra el mensaje \MSGref{MSG25}{Servicios no disponibles por el momento.}.
    \UCpaso[\UCactor] Cierra el mensaje presionando el botón \IUbutton{Aceptar}.
    \UCpaso El sistema regresa a la interfaz de usuario \IUref{IU2}{Gestionar empleados}.
\end{UCtrayectoriaA}

\begin{UCtrayectoriaA}{I}{El sistema no encuentra la interfaz a mostrar o existe un problema con el servidor.}
    \UCpaso No se puede mostrar la página solicitada.
    \UCpaso El sistema muestra el \MSGref{MSG24}{Por el momento la página solicitada no esta disponible}.
    \UCpaso[\UCactor] Cierra el mensaje presionando el botón \IUbutton{Aceptar}.
    \UCpaso El sistema regresa a la interfaz de usuario \IUref{IU2}{Gestionar empleados}.
\end{UCtrayectoriaA}
