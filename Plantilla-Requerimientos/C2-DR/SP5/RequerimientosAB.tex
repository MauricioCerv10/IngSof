% -------------- TABLA PARA REQUERIMIENTOS FUNCIONALES ---------------- % 
% Nomenclatura para la prioridad: 
%	A - Alta
%	M - Media
%	B - Baja

\textbf{Nota:} Se usa :
	\begin{itemize}
		\item \textbf{JDIC:} Jefe de departamento de Innovación curricular
		\item \textbf{JDIA:} Jefe de división de innovación académica
		\item \textbf{DES:} Directora de Educación Superior
	\end{itemize}

\begin{table}[h!]
	\begin{requerimientos}
		
		\FRitem{SP5-F1}{Registrar empleados del JDIA}{El sistema permitirá al usuario JDIA registrar usuarios JDIC.}{A}{}

		\FRitem{SP5-F2}{Registrar empleados del JDIC}{El sistema permitirá al usuario JDCI registrar usuarios analistas y jefes de innovación educativa.}{A}{}

		\FRitem{SP5-F3}{Consultar empleados}{El sistema permitirá al JDIA y al JDIC consultar la información y trabajo activo de los analistas y jefes de innovación educativa.}{M}{}

		\FRitem{SP5-F4}{Eliminar empleados del JDIA}{El sistema permitirá al JDIA eliminar usuarios JDIC.}{B}{}

		\FRitem{SP5-F5}{Eliminar empleados del JDIC}{El sistema permitirá al JDCI eliminar usuarios analistas y jefes de innovación educativa.}{B}{}

		\FRitem{SP5-F6}{Consultar asignaciones del analista}{El sistema le mostrará al usuario analista las tareas  que tiene asignadas.}{A}{}

		\FRitem{SP5-F7}{Consultar mapas curriculares}{El JDIA y el JDIC consultarán la lista de los mapas curriculares por nombre de la unidad académica o todos los mapas curriculares registrados.}{M}{}

		\FRitem{SP5-F8}{Ver mapa curricular}{El sistema mostrará al JDIA y al JDIC la información de un mapa curricular previamente seleccionado.}{A}{}

		\FRitem{SP5-F9}{Asignar analista}{El JDIC seleccionará (de la lista de usuarios analista, él incluido) el empleado (visualizando su nombre y el número de tareas que tiene asignadas) que revisará el mapa curricular previamente elegido.}{M}{}

		\FRitem{SP5-F10}{Atender mapa curricular por parte del JDIC}{El sistema permitirá al JDIC analizar, en conjunto con los analistas, uno o más mapas curriculares recibidos por las unidades académicas.}{A}{}

		\FRitem{SP5-F11}{Comunicar observaciones del mapa curricular a los analistas}{El sistema permitirá al JDIC contactar al analista encargado de un mapa curricular para comunicarle observaciones, correcciones, notas, etcétera, respecto a éste.}{M}{}

		\FRitem{SP5-F12}{Realizar modificaciones al mapa curricular}{El sistema permitirá a la JDIC realizar correcciones y ajustes a un mapa curricular, en conjunto con su analista asignado, para llegar a un acuerdo.}{A}{}

		\FRitem{SP5-F13}{Verificar nombre de unidades de aprendizaje en el mapa curricular}{El sistema permitirá a la JDIC y a la JDIA aprobar o rechazar el nombre que se le asignó a las unidades de aprendizaje en un mapa curricular.}{A}{}

		\FRitem{SP5-F14}{Iniciar revisión del mapa curricular}{Un mapa curricular enviado por la unidad académica debe tener todos sus campos completos y llenos para que se pueda iniciar el revisado por la DES en el sistema.}{A}{}

		\FRitem{SP5-F15}{Realizar notas y comentarios}{El sistema permitirá al JDIC y a los analistas redactar una o más correcciones por cada campo de un mapa curricular.}{A}{}

		\FRitem{SP5-F16}{Imprimir notas y correcciones}{El sistema permitirá imprimir el historial de correcciones de un mapa curricular, con el formato establecido por la DES.}{B}{}

		\FRitem{SP5-F17}{Resaltar palabras erróneas}{El sistema permitirá a la JDIC y analistas subrayar o marcar una o varias palabras o secciones erróneas en cada campo de un mapa curricular.}{M}{}


	\end{requerimientos}
    \caption{Requerimientos funcionales del sistema.}
    \label{tbl:}
\end{table}