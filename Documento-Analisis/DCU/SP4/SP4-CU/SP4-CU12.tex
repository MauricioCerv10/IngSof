\begin{UseCase}{SP4-CU12}{Revisar Mapa Curricular}{Los usuarios Jefes y Analista visualizan los Contenidos de un Mapa Curricular y de ser necesario agregan comentarios.}
        \UCitem{Versión}{\color{Gray}1.0}
        \UCitem{Autor}{\color{Gray}Rivas Rojas Arturo}
        \UCitem{Supervisa}{\color{Gray}}
        \UCitem{Actor}{\hyperlink{JDIE}{Jefe de Innovación Educativa}, \hyperlink{JDIA}{Jefe División de Innovación Académica}, \hyperlink{JDDIC}{Jefe de Departamento de Desarrollo e Innovación Curricular} y \hyperlink{A}{Analista}}
        \UCitem{Propósito}{Realizar la revisión y aprobación de un Mapa Curricular.}
        \UCitem{Entradas}{Las entradas para la revisión de un Mapa Curricular serán:
          \begin{itemize}
            \item Comentarios.
          \end{itemize}
        }
        \UCitem{Origen}{Teclado.}
        \UCitem{Salidas}{
            \begin{itemize}
				\item \MSGref{MSG16}{¿Está seguro de finalizar la revisión?}
				\item \MSGref{MSG18}{Revisión de Sección Finalizada.}
				\item \MSGref{MSG25}{Servicios no disponibles por el momento.}
            \end{itemize}
        }
        \UCitem{Destino}{Pantalla.}
        \UCitem{Precondiciones}{
            \begin{itemize}
                \item Para el Analista debe haberse concluido la tarea de registro del Mapa Curricular.
                \item Para los Jefes debe haberse concluido la tarea de revisión del Mapa Curricular por parte del Analista.
            \end{itemize}
        }
        \UCitem{Postcondiciones}{
            \begin{itemize}
                \item Después de que lo haya revisado un Analista lo puede revisar su Jefe.
                \item Después de que el Jefe de Innovación Educativa lo haya revisado y aprobado ahora lo podrán revisar en la DES.
            \end{itemize}
        }
        \UCitem{Errores}{}
        \UCitem{Puntos de Extensión}{}
        \UCitem{Estado}{Revisión.}
        \UCitem{Observaciones}{}
\end{UseCase}
%--------------------------- CU TRAYECTORIA PRINCIPAL -------------------------
\begin{UCtrayectoria}{Principal}
    \UCpaso[\UCactor] Presiona el botón \IUbutton{Consultar} de la Interfaz de usuario  \IUref{CT-J}{Consultar Tareas}.
    \UCpaso Muestra la interfaz de usuario \IUref{RMC-A}{Revisar Mapa Curricular}.
    \UCpaso[\UCactor] Selecciona el semestre de la Unidades de Aprendizaje que desea revisar. [Trayectoria A][Trayectoria A.1]
    \UCpaso Expande el semestre seleccionado mostrando las Unidades de Aprendizaje registradas en el sistema. [Trayectoria A]
    \UCpaso[\UCactor] Termina la operación presionando el botón \IUbutton{Finalizar}.
    \UCpaso Muestra el mensaje \MSGref{MSG16}{¿Está seguro de finalizar la revisión?}.
    \UCpaso[\UCactor] Cierra el mensaje presionando el botón \IUbutton{Si}. [Trayectoria B]
    \UCpaso Muestra el mensaje \MSGref{MSG18}{Revisión de Sección Finalizada.}.[Trayectoria C]
    \UCpaso[\UCactor] Cierra el mensaje presionando el botón \IUbutton{Aceptar}.
    \UCpaso Muestra la interfaz de usuario \IUref{CT-J}{Consultar Tareas}.
\end{UCtrayectoria}
%------------------------ CU TRAYECTORIA ALTERNARIVA A -------------------------
\begin{UCtrayectoriaA}{A}{El actor presiona el botón \IUbutton{lápiz}.}
    \UCpaso Muestra una ventana emergente solicitando el comentario.
    \UCpaso[\UCactor] Cierra la ventana presionando el botón \IUbutton{Aceptar}.
    \UCpaso Continúa en el paso 3 de la trayectoria principal del \UCref{SP4-CU12}.
\end{UCtrayectoriaA}
%------------------------ CU TRAYECTORIA ALTERNARIVA A.1 -------------------------
\begin{UCtrayectoriaA}{A.1}{El actor presiona accidentalmente \IUbutton{lápiz}.}
    \UCpaso Muestra una ventana emergente solicitando el comentario.
    \UCpaso[\UCactor] Cierra la ventana presionando el botón \IUbutton{Cancelar}.
    \UCpaso Continúa en el paso 3 de la trayectoria principal del \UCref{SP4-CU12}.
\end{UCtrayectoriaA}
%------------------------ CU TRAYECTORIA ALTERNARIVA B -------------------------
\begin{UCtrayectoriaA}{B}{El actor presiona accidentalmente el botón \IUbutton{Finalizar}}
    \UCpaso[\UCactor] Cierra el mensaje presionando el botón \IUbutton{No}.
    \UCpaso Continúa en el paso 3 de la trayectoria principal del \UCref{SP4-CU12}.
\end{UCtrayectoriaA}
%------------------------ CU TRAYECTORIA ALTERNARIVA C -------------------------
\begin{UCtrayectoriaA}{C}{El actor realizó comentarios}
    \UCpaso Persiste los datos. [Trayectoria C.1]
    \UCpaso Muestra el mensaje \MSGref{MSG18}{Revisión de Sección Finalizada.}.
    \UCpaso[\UCactor] Cierra el mensaje presionando el botón \IUbutton{Aceptar}.
    \UCpaso Muestra la interfaz de usuario \IUref{CT-J}{Consultar Tareas}.
\end{UCtrayectoriaA}
%------------------------ CU TRAYECTORIA ALTERNARIVA C.1 -------------------------
\begin{UCtrayectoriaA}{C.1}{Ocurre un error al momento de persistir los datos}
    \UCpaso Muestra el mensaje \MSGref{MSG25}{Servicios no disponibles por el momento.}.
    \UCpaso[\UCactor] Cierra el mensaje presionando el botón \IUbutton{Aceptar}.
    \UCpaso Muestra la interfaz de usuario \IUref{CT-J}{Consultar Tareas}.
\end{UCtrayectoriaA}